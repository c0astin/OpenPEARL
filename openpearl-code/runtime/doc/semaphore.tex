\chapter{Semaphore}
PEARL requires a semaphore implementation with simultaneous locking
mechanism. This is very unfamiliar in small embedded operation systems.
Despite of the fact that Linux would support this feature for semaphores 
thie mechanisme is implemented in the class {\em Semaphore} using one counting semaphore and a mutex. 
Each multitasking operating system provide these two native elements.
The Semaphore-class uses the common interfaces {\em CSemaCommon} and 
{\em MutexCommon}. Both classes need a plattform specific implementation 
with the names {\em CSema} and {\em Mutex}.

A single PEARL semaphore is
just a counter --- and for diagnosic purposes a pointer to the semaphore''s
name.

The semaphore operation request locks the class mutex to avoid race conditions 
with other tasks semaphore operations. In case of all semaphores value are 
larger than 0, they are decremented and the task continued.
In case of at least one requested single semaphore is 0, the task is added
to a wait queue ({\em PriorityQueue}).

On releasing a PEARL semaphore, the wait queue content is checked if a 
task may become continued. The wait queue is searched in order of the tasks 
priority. The single semaphore''s  values are updated as necessary.

To block a task the class must provided the methods {\em block} and 
{\em unblock}.

\section{Declaration}
The declaration of a PEARL semaphore is always done with a preset value.

\begin{PEARLCode}
DCL s0 SEMAPHORE PRESET(0) GLOBAL;
DCL s1 SEMAPHORE PRESET(1);
DCL s2 SEMAPHORE PRESET(2);
\end{PEARLCode}

The class {\em Semaphore} provides a ctor for definition 
of a semaphore variable.
\begin{description}
\item[Semaphore] takes 2 parameters; the semaphore's name and the preset value.
\end{description}

\begin{CppCode}
pearlrt::Semaphore _s0("_s0",0);
static pearlrt::Semaphore _s1("_s1",1);
static pearlrt::Semaphore _s2("_s2",0);
extern pearlrt::Semaphore _s2; // import
\end{CppCode}

\section{Operations}
The class {\em Semaphore} provides the \verb|static| operations 
{\em request()}, {\em release()} and {\em dotry()}\footnote{try is a reserved
keyword in C++}.

PEARL provides the simultaneous locking pattern on semphores.
This leads to the interface of the operations:
\begin{CppCode}
void Semaphore::request(Task *me, int nbrOfSemas, Semaphore **semas);
void Semaphore::release(Task *me, int nbrOfSemas, Semaphore **semas);
int Semaphore::dotry(Task *me, int nbrOfSemas, Semaphore **semas);
\end{CppCode}

\paragraph{Regard that TRY allows only one sempahore in PEARL.}

\section{Sample Usage}

\begin{PEARLCode}
REQUEST s0,s1,s2;
\end{PEARLCode}

\begin{CppCode}
{
   static Semaphore * semas[] = {&s0,&s1,&s2};
   Semaphore::request(me, sizeof(semas[])/sizeof(semas[0]), &semas);
}
\end{CppCode}

The new block allows the reuse of the identifier {\em semas} in multiple
{\em request}, {\em release} and {\em try} operations in the same 
module.
The static storage declaration provides the initialization before run time.
The size calculation should be executed by the C++-compiler.
