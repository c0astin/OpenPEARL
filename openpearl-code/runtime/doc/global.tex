\chapter{Multiple Modules and GLOBAL}

PEARL supports the separation of the application into seperate modules.
In OpenPEARL, the module name given by the \texttt{MODULE(name)} statement
defines the namespace in C++.
In order to avoid conflicts with namespaces in the standard libraries,
the supplied module name gets a prefix \texttt{pearl\_}.
All user supplied identifiers reside in the namespace of the module.

The complete runtime library uses the namespace \texttt{pearlrt}.

OpenPEARL needs some other variables for the organisation. 
They may be in the same namespace, if conflicts are avoided. This is 
achieved by prefixing all user supplied identifiers with a prefix \texttt{\_}.

\begin{PEARLCode}
MODULE(moduleA);

PROBLEM;
   DCL  a FIXED GLOBAL;
   SPC b FIXED GLOBAL (moduleB);
...
! in any PROC
   a := b;
...
\end{PEARLCode}

will produce a C++ code like:
\begin{CppCode}
#include "PearlIncludes.h"

namespace pearl_moduleA {
   pearlrt::Fixed<31> _a;
}
namespace pearl_moduleB {
   extern pearlrt::Fixed<31> _b;
}
namespace pearl_moduleA {
...
// in any PROC
   _a := moduleB::_b;
...
}
\end{CppCode}



