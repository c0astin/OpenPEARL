\section{Buildsystem}

The ESP32 is provided and supported by Espressif (www.espressif.com).
An IoT debelopment Framework (IDF) and a c/c++ buildsystem are provided. 

The IDF is structured in socalled {\em components}. An IDF-Project contains
the project specific files which are combined by the framework with the 
framework components. The build of an ESP-IDF application is done 
{\em out of place} --- in the build folder of the application folder.

The OpenPEARL buildsystem requires the invocation of the suitable 
g++ compiler/linker upon the c++ files, which result from the PRL to C++ 
translation of the OpenPEARL compiler.

To use the ESP-IDF, the steps for installation must be done as described in 
getting started guide from Espressif.
To check the installation an presence of all prerequisites you should build
 one the provided examples of the ESP-IDF as described there.

\subsection{Integration of the ESP-IDF with the OpenPEARL-Repostory}
The ESP-IDF becomes extended in high rate. Thus we do not copy a specific
version of the ESP-IDF in our repository.
The integration of OpenPEARL and the ESP-IDF is done via the following steps:
\begin{enumerate}
\item the toolchain of the ESP-IDF expects an environment variable
   for the location of the \texttt{esp-idf} folder.
   This location may be set via \texttt{make menuconfig}
   of the concrete project 
   (run \texttt{make menuconfig} in \texttt{esp-idf-project}) 
\item There is a need for a concrete project to build.
   This project is located in the folder \texttt{.../runtime/esp32/OpenPEARLProject}
   This folder contains the \texttt{MakeFiles.txt} and the default \textit{application}-component \texttt{main}.
\item The main folder contains 
  \begin{itemize}
  \item symlinks to the platform independent parts (common)
  \item FreeRTOS specific parts (PEARL)
  \item  some additionals files in (addOns)
  \item the ESP32 specific device drivers
  \item CMakeFiles.txt with a list of all contribuing files
  \item OpenPEARL needs some changes in the FreeRTOS implementation.
    The folder \texttt{patchFiles} contains the expected and patched 
    version of some esp-idf releases.
  \end{itemize}
\end{enumerate}

The OpenPEARL build system is based on \verb|make|. The EspIdf is now based 
on cmake.

\subsection{Integration of esp-idf to OpenPEARL}
The folder ..../esp32/OpenPEARLProject contains a complete OpenPEARL applicationwith no tasks but an PEARL-application module \texttt{testprog.prl} 
and its compiled outputs: 
\texttt{testproc.cc} and its \texttt{system.cc}

In the folder \texttt{.../OpenPEARLProject} we need:
\begin{description}
\item[\texttt{. ~/esp/esp/idf/export.sh}] to setup the path entries and
   environment variables for the esp-idf-
\item [\texttt{idf.py build}] the create the idf-project, with support of all
   esp-idf components, which are required for the testprog.prl
\item the interesting part is the library, which coresponds 
   to the main-component
\item we can use this library later for the make-bases OpenPEARL linkage, afer
  we removed the objects of \texttt{testprog} and \texttt{system}
\item during \texttt{make install} the compile and link commands
  are created with the help of some generated files
  \begin{itemize}
  \item esp\_idf\_build.mk contains lists of linker parameters and the setting of  the \texttt{CXX} variable for make
  \item cp.cmd contains copy commands for the shell to copy all required
   libraries and linker command files the the installation target
  \item these files may be recreated with the script \texttt{generateLinkerCommandAndCopyFiles.sh}, if there is need of change
  \end{itemize}
\end{description}


\subsection{Steps to use new esp-idf-components}
If a new esp32 specifig device becomes integrated in OpenPEARL, it is
possible that currently unsed esp-idf components are required. 
You will notice this by error messages from the c++-compiler and/or linker.

Update procedure:

\begin{enumerate}
\item enter folder \texttt{OpenPEARLProject}
\item extend \texttt{testprog.prl} with the new system elements
\item compile them with \texttt{prl -c -b esp32 testprog.prl}
\item enable idf-virtual environment with \texttt{. ~/esp/esp-idf/export.sh}
\item Build the OpenPEARLProject with \texttt{idf.py -v build} \newline
   The option \texttt{-v} is important
\item select the printed linker command (this is one of the lasts steps)
   and copy this command in a file \texttt{linker.txt.new}
\item remove all decoration stuff from linker.txt.new (see linker.txt as
   a reference)
\item replace linker.txt with your new version
\item run \texttt{./generateLinkerCommandAndCopyFiles.sh}
\item enter parent folder
\item run \texttt{make install} --- this should be ok now.
\end{enumerate}

\subsection{cc\_bin and run\_bin}
The overall script 'prl' for compiling and linking of an OpenPEARL application
needs platform specific commands to compile c++ files and link them with the 
platform specific libraries.

For each platform two files must be provided named 'cc\_bin.inc'
and 'run\_bin.inc'. They are include during the installation of the OpenPEARL
distribution in the overall 'prl' script.

\section{Kconfig parameters}
There are some parameters for the OpenPEARL configuration.
\begin{description}
\item[ESP32\_IDF\_PATH] must point to the installation location of yout ESP-IDF.
\item[ESP32\_FLASH\_INTERFACE] must identify the device name
   of your interface to the ESP32
\end{description}
More item may be available in future.

\section{FreeRTOS}
The ESP32-IDF provides a modified version of FreeRTOS which supports a kind of 
symetric multiprocessing.
This ESP-IDF component needs a \verb|FreeRTOSConfig.h| file.
OpenPEARL needs some setting in the \verb|FreeRTOSConfig.h|. 
This configuration is part of the OpenPEARL repository. The ESP-IDF provided
configuration file is replaced by the OpenPEARL specific version.

\section{Notes about updates in ESP-IDF}
If a new release of ESP-IDF should be used, several checks must be done:
\begin{itemize}
\item check changes in the ESP-IDF system
  \begin{itemize}
    \item location of  \verb|FreeRTOSConfig.h|
    \item  the FreeRTOS implementation of tasks.c and tasks.h may be 
       not compatible with the versions in pathFiles
  \end{itemize}
\item check changes in the components API and names
\item .. ??? ... hope that it works
\end{itemize}

\section{Partitions on the Flash Memory}
There are some partitions like:

\begin{verbatim}
# ESP-IDF Partition Table
# Name,   Type, SubType, Offset,  Size, Flags
nvs,      data, nvs,     0x9000,  0x4000,
otadata,  data, ota,     0xd000,  0x2000,
phy_init, data, phy,     0xf000,  0x1000,
factory,  0,    0,       0x10000, 1M,
ota_0,    0,    ota_0,  0x110000, 1M,
ota_1,    0,    ota_1,  0x210000, 1M,
\end{verbatim}
The bootloader starts the application in the partition \texttt{factory}.
The partition \texttt{nvs} store non volatile data.
The partition \texttt{ota-1} and \texttt{ota-2} are used for updates over the air.

\subsection{Resize the Application Partition}
For larger applications, the size of 1MB of the partition \texttt{factory} is too small.
The project \texttt{OpenPEARLProject} contains a custom 
partition table --- currently with a 2MB partition for the application.

\begin{verbatim}
make initial_flash
\end{verbatim}

\subsection{Non Volatile Storage Partition (nvs)}
The nvs is used be several examples of the espressif ide. 
The nvs-storage is used by OpenPEARL in the namespace \texttt{OpenPEARL} for
\begin{itemize}
\item Wifi-Parameters may be placed with the configutarion itemEsp32WifiConfig
\item \textit{others may follow}
\end{itemize}


\subsection{Flash Bootloader and Partition Table}
The bootloader an partition table are not flashed with the OpenPEARL application.
This must be done at least once for each ESP32-module before first use.


\section{Open Issues}
\subsection{Core Usage}
Currently only one core is used.
In version V4.2 there is a problem with the WiFi-setup using two cores which leads to a panic failure.

In future, the second core should be used also.
The automatic restart of an application task assumes that only one processor core ist 
available.
There are two possibilities to overcome this obstancle:
\begin{enumerate}
\item enshure that all application tasks are on the same core and use the other core for i/o-tasks
\item add a patch to FreeRTOS to restart an overrun task automatically at its termination.
\end{enumerate}

\subsection{Esp32Uart Device}
The ESP-IDF is subject of rapid changes. Therefore it is not 
useful to modify any provides code for usage of the internal devices.

Especially the uart is very complicated. Thus a separate FreeRTOS thread 
is used for the write operation. The read operation uses the 
provided mechanism with an event queue.

Suspend and terminate during i/o operations are treated by a simulation
of an receive event or semaphore unlocking for send done.


\subsection{missing Device Drivers}
\begin{enumerate}
\item Console
\item I2C
\item Ethernet
\item ...
\end{enumerate}


