\documentclass[oneside,10pt]{scrbook}
\usepackage{geometry}
\usepackage{graphicx}
\usepackage{import}
\usepackage{xcolor}
\usepackage{listings}
\usepackage{caption}
\usepackage{hyperref}
\usepackage{tikz}
\usetikzlibrary{calc,positioning,shapes.geometric}

\geometry{a4paper}
\setlength{\parindent}{0pt}

%\DeclareCaptionFont{white}{\color{white}}
%DeclareCaptionFormat{listing}{\colorbox[cmyk]{0.43, 0.35, 0.35,0.01}
\DeclareCaptionFormat{listing}{\colorbox{black!15}
  {\parbox{5cm}{#1#2#3}}}
\captionsetup[lstlisting]{format=listing,
%labelfont=white,textfont=white, 
singlelinecheck=false, margin=0pt, font={bf,footnotesize}}

\begin{document}
\title{OpenPEARL - Platform Manual for EPS32 Systems}
\author{R. M\"uller}

\lstnewenvironment{CppCode} {
    \lstset{numbers=left,
            title={C++},
            frame=tlrb,
	    breaklines = true,
	    belowcaptionskip=4pt
    }
}%
{}%

\lstnewenvironment{PEARLCode} {
    \lstset{numbers=left,
            title={PEARL},
            frame=tlrb,
	    breaklines = true,
	    belowcaptionskip=4pt
    }
}%
{}%

\pagestyle{myheadings}
\markboth {OpenPEARL Platform Manual ESP32}
          {OpenPEARL Platform Manual ESP32}

\maketitle

\tableofcontents

%\input{log.tex}
%\input{interrupt.tex}
%\input{inOutput.tex}

\chapter{Installation}
The installation must be done via the git repository of the OpenPEARL project.
There are several software packages required on your linux PC. 
The repository contains an install script (\verb|installPackages|),
which identifies your
linux distribution by content of the file \verb|/etc/os-release|.

For details of the additional packages study the installation script.

For details of the installation for the esp32-toolchain please 
consult the esp32 homepage from espressif.

\paragraph{Prerequisites}
\begin{itemize}
\item you have access to your linux system as user and root
\item you have an internet connection
\item you have a git client installed
\end{itemize}

\paragraph{Perform the following steps:}
\begin{enumerate}
\item login as normal user
\item create a working directory for the installation. \\
   e.g. /home/userx/OpenPEARL\\
    \verb|mkdir /home/userx/OpenPEARL|
\item set you current working directory to this point \\
    \verb|cd /home/userx/OpenPEARL|
\item obtain a read only copy the repository\\
    \verb|git clone git://git.code.sf.net/p/openpearl/code .|\\
    regard the point at the end of the command
\item set the working directory to the top of the real content
     in your working copy 
     (we are sorry about this unnecessary directory level): \\
    \verb|cd openpearl-code|
\item get administration priviledges\\
    \verb|sudo| or \verb|su|
\item run the script:\\
    \verb|./installPackages|\\
    and wait for completion of the installation.
    This takes about 6 minutes at an RaspberryPi 3B
\item still as administrator, execute\\
    \verb|make defconfig|\\
    \verb|make prepare|\\
    \verb|make install|\\
\end{enumerate}

\paragraph{Verify Installation:}
Work as normal user.


\begin{enumerate}
\item \verb|cd /home/userx/OpenPEARL/openpearl-code/demos|
\item build some of the provided demonstration programs\\
   \verb|prl -b esp32 <demo>.prl|\\
   where \verb|<demo>| is Hello or sched\_demo
\item run the demonstration programm with \\
   \verb|-r -b esp32 <demo> |\\
   The programs should print some messages and terminate themself.
\end{enumerate}


\chapter{Configuration}
\section{Available Options}
\begin{itemize}
\item location of esp-idf
\end{itemize}

For details see the {\em HELP} pages in the \verb|make menuconfig| 
entries.

\section{Rebuild of the OpenPEARL Environment}
The tuning the installation must be done as administrator:
\begin{enumerate}
\item \verb|cd /home/userx/OpenPEARL/openpearl-code|
\item change the option upon your demands:\\
      \verb|make menuconfig| 
\item \verb|make install|
\end{enumerate}

\section{Configuration Option for the individual Application}
\import{../../common/platformManual/}{configurationCommon.tex}
%\import{./}{log.tex}
%\import{./}{pearlrc.tex}

\chapter{Release Information}
\section{Supported Language Elements}
The OpenPEARL project aims to support all language elements  as
described in the language report.

This version does not support:\footnote{The numbers in braces refer the ticket numbers.}

\begin{itemize}
\item STRUCT
\item TYPE
\item no SIGNAL treatment (\#127)
%%% \item no INTERRUPT support (\#122)
%%% \item operator SIZEOF not supported (\#109)
%\item length of float constants must be clearified and is subject of 
%		change(\#120)
\item REF CHAR
\item REF used upon inititialized references does not throw a signal (\#217)
\item REF on TASK,DATION,SIGNAL,INTERRUPT not tested (\#196,\#197)
\item no support for multi modules (\#193)
\item I/O-related
   \begin{itemize}
   %\item no warning if multiple positions formats are inside a READ/WRITE statement (\#126)
   %\item E-format does not work properly (\#210)
   %\item CONVERT not supported (\#215)
   %\item additional X-elements in some PUT statements (\#221)
   %\item format elements PAGE,ADV,LINE,COL,POS,SOP,EOF not supported(\#216,\#222)
   \item CYCLIC only available on system dations of type DIRECT (\#226)
   \item repetition factor in i/o formats and variable array slices 
      does not work in i/o statements(\#232)
   \item TFU not supported (\#232,\#237)
%   \item loop index not accessible in i/o statements (\#239)
%   \item type of dation element in i/o statements not checked (\#251)
   \item no array slices as i/o element are supported (\#223,\#252)
   \item no R-Format (\#254)
   \end{itemize}
%\item no constants expressions treated by the compiler (\#248)
\item BIT-slices with size 1 do not work with variable slice index (\#243)
\item BIT and CHAR-slices not allowed on left hand side and in input statements
    (\#224)
%\item control characters in string constants does not work if 
%    there is a newline in the control character sequence(\#203) 

\item Semantic analysis for good diagnostic error messages.
     In case of problems error messages from the c++ compiler may appear. 
\item Semantic analysis for program quality analysis (e.g. unused variables)
\end{itemize}

\chapter{Usage of the OpenPEARL system}

The principle of operation is very similar to the usage of the \texttt{gcc}.

\begin{enumerate}
\item create source file
\item compile and link
\item run the application
\end{enumerate}

In order to explain the usage, we start with a copy of the hello.prl demonstration file.

\begin{enumerate}
\item copy \texttt{hello.prl} into your working directory (e.g. \texttt{/home/userx/prl}) \newline
   \texttt{cp /home/userx/OpenPEARL/openpearl/demos/hello.prl /home/userx/prl}
\item set the working directory to \newline
   \texttt{cd /home/userx/prl}
\item build the hello world application \newline
   \texttt{prl -b esp32 hello.prl}
\item run the hello world application \newline
   \texttt{./hello}
\end{enumerate}

For your own applications, just modify the source code of demo.prl according your demands.
You should not use \texttt{system.prl} as source file - this is a known problem of the current
build system.


\paragraph{Note:}
Please ignore the intermediate files like:
hello.cc, hello.log, hello.xml, system.cc



\chapter{Supported Devices}
\import{../../common/platformManual/}{devicesCommon.tex}
\import{./}{devices.tex}

\chapter{Supported Interrupt Sources}
\import{../../common/platformManual/}{interruptCommon.tex}
\import{./}{interrupts.tex}

\chapter{Available PEARL Signals}
The OpenPEARL signals are organized in some groups.

\import{../../common/}{signallist.tex}


\chapter{How-To: Add new System Elements}
\import{../../common/platformManual/}{addDeviceCommon.tex}
\import{./}{addDevice.tex}
\import{../../common/platformManual/}{addInterruptCommon.tex}
\import{./}{addInterrupt.tex}
\import{../../common/platformManual/}{addSignalCommon.tex}
\end{document}

