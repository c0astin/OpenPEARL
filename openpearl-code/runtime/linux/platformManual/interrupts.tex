
\section{UnixSignal}
Some UNIX signals may be used to trigger the PEARL application from outside.
These signals may be mapped to PEARL interrupts.

Synopis: \verb|UnixSignal(nbr)|

\begin{description}
\item [nbr] denotes the linux signal number.
The following signals are supported:

\begin{tabular}{|l|c|}
\hline
Signal & Number \\
\hline
SIGHUP & 1 \\
SIGINT & 2 \\
SIGQUIT & 3 \\
SIGTERM & 15 \\
SIGUSR1 & 16 \\
SIGUSR2 & 17 \\
\hline
\end{tabular}
\end{description}

Sample usage:
\begin{PEARLCode}
...
SYSTEM;
  ctrlc: UnixSignal(3);
PROBLEM;
   SPC ctrlc INTERRUPT;

   DCL shutdown BIT INIT('0'B1);

running: TASK;
   WHILE NOT shutdown REPEAT
      ...
   END;
END;

shutdownTask: TASK;
   ENABLE ctrlc;
   WHEN ctrlc RESUME;
   PUT '*** shutdown request received' TO console BY A, SKIP;
   PUT '*** wait until all tasks terminated themself' TO console BY A, SKIP;
   shutdown := '1'B1;
END;
...
\end{PEARLCode}

\paragraph{Note:}
\begin{itemize}
\item Note that there may be only one user defined name for each UNIX signal.
\end{itemize}
