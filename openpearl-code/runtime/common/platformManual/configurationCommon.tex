
\subsection{Log --- Logging}
The configuration element LOG allows the specification of the 
log output of the OpenPEARL application.

Sample Usage:
\begin{PEARLCode}
Log('EW') --- StdErr;
\end{PEARLCode}

The element LOG needs an output device,
which receives text messages without a file name.
Typical system devices are StdErr, StdOut or LogFile depending on the concrete 
platform. 

The log level is defined by the character string. Possible letters are:
\begin{description}
\item['E'] error messages from signals, even if they are caught
\item['W'] warning from the runtime library
\item['D'] debug messages from the runtime library
\item['I'] informative messages from the runtime library, like the 
    list of defined tasks at system start
\end{description}

The default log level is 'EW'.

\subsection{SendXML --- report exit status code as XML}
The configuration element SendXML allows to report the 
exit code at the end of the application to the standard output.
The exit code may be affected with the statment:

\begin{PEARLCode}
__cpp__('pearlrt::Control::setExitCode(1);');
\end{PEARLCode}

Sample usage:
\begin{PEARLCode}
SendXML;
\end{PEARLCode}

This feature is good for automatic testing on remote microcontroller systems.
A minitoring computer may flash and run test applications and parse the output 
for the starting xml-tag.

In order to use this feature with changing the source file, it is possible to
 pass an additional module to the \texttt{prl}-command via the option \texttt{-add}.

Example:
\texttt{prl HelloWorld.prl -add SendXML} 
needs a file SendXML.xml in the current folder. 
This file may be created with the following OpenPEARL-module.

\begin{PEARLCode}
MODULE(SendXML);
SYSTEM;
   SendXML;
MODEND;
\end{PEARLCode}

