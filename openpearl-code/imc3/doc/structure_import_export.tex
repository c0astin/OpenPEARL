\chapter{Structure of Module Import and Export Definition}

The compiler creates an xml-file per PEARL-module containing the
\begin{itemize}
\item system part information
\item problem part specifications and declarations
\end{itemize}

This chapter describes the interface to the compiler.

\section{XML-Document Structure}
The compiler has no knowledges about the nature of a system name.
Thus the system part is translated as it is --- only syntactical errors
are recognized. 
The document has the root tag \verb|<module>|, which hat the attribute
\verb|file| with the value source file name.  
All system elements are in the xml-subtree \verb|<system>|.
All problem part elements with attribute GLOBAL are located
in the xml-subtree \verb|problem|.

\begin{XMLCode}
<?xml version="1.0" encoding="UTF-8" ?>
<module file="demo.prl">
<system>
   <username .... >
   ....
   </username>
   <configuration .... >
   ....
   </configuration>
</system>
<problem>
  <spc .../>
  <dcl .../>
  ...
</problem>
</module>
\end{XMLCode}

\section{System Part Elements}
\subsection{User Names \texttt{<username>}}
If a system element defines a user name the \verb|<username>| tag is 
used. Each user name is accompanied by an system name with the tag 
\verb|<sysname>|, which may have parameters.
The sysname-tag may contain a list of associations, thus there may by a tree
of \verb|<association>| tags.
The user name tag contains the attribute \verb|line| with the value of the 
source code line number.

\subsection{System Name \texttt{<sysname>}}
The system name tag contains the attribute \verb|name| containing the 
name of the system element.

\subsection{Parameters \texttt{<parameters>}}
\label{sec_parameters}
The parameters are located in the \verb|<parameters>|-subtree.
The imc tool supports constants of type FIXED, CHAR and BIT.
The compiler detects  the type of the parameter
from the literal and passes the literal as content to the corresponding
\verb|<FIXED>|-, \verb|<CHAR>|- or \verb|<BIT>|-tag.

\subsection{Associations \texttt{<association>}}
An association may be ether a system name (with parameters) or a user name.
The  \verb|<association>|-tag contains the attribute \verb|name| with the 
given name as value. If parameters are specified for the association,
they are passed as \verb|<parameters>|-subtree in the same way as decribed 
in section \ref{sec_parameters}.

There is no check in the compilation phase done,
whether a name is a system name
or a user supplied name.

As defined in the grammar of OpenPEARL an association may lead to a 
connection provider which has another association.
Example:
\begin{PEARLCode}
SYSTEM;
  Log('EW') --- LogFile('pearl_log.txt') --- DISC('./',1);
\end{PEARLCode}


\subsection{Configuration Item}
Configuration elements in the system part are located in the
\verb|<configuration>|-element. 
The only difference to the \verb|<username>|-tag is the absence of the 
\verb|name|-attribute. Parameters and associations apply identical.

\subsection{Example for System Part Elements and their Specifications}
The following example contain two declaractions of the same I2C interface.
It depends on the platform definition wether this is allowed or not. For I2CBus 
interfaces this would be not allowed on the same device-node.

\begin{PEARLCode}
MODULE(demo);

SYSTEM;
lm75: LM75('48'B4) --- I2CBus('/dev/i2c-0');

lm75: LM75('48'B4) --- i2cbus1;
i2cbus1: I2CBus('/dev/i2c-1');

sig1: FixedRangeSignal;
int1: UnixSignal(15);

disc: Disc('/tmp/folder1', 10);

stdOut: StdOut;

Log('EW') --- StdOut;

PROBLEM,
SPC sig1 SIGNAL GLOBAL;
SPC int1 INTERRUPT GLOBAL;
SPC disc DATION OUT DIRECT SYSTEM ALL GLOBAL;
SPC stdOut DATION OUT SYSTEM ALPHIC GLOBAL;
SPC lm75 DATION IN SYSTEM FIXED(15) GLOBAL;

MODEND;
\end{PEARLCode}

\begin{XMLCode}
<?xml version="1.0" encoding="UTF-8" ?>
<module file="demo.prl">
<system>
   <username name="lm75" line="4">
      <sysname name="LM75">
      <parameters>
         <BIT>'48'B4</BIT>
      </parameters>
      <association name="I2CBus">
         <parameters>
            <CHAR>'/dev/i2c-0'</CHAR>
         </parameters>
      </association>
   </sysname>
</username>

<username name="lm75a" line="6">
   <sysname name="LM75">
      <parameters>
         <BIT>'49'B4</BIT>
      </parameters>
      <association name="i2cbus1">
      </association>
   </sysname>
</username>

<username name="i2cbus1" line="7">
   <sysname name="I2CBus">
      <parameters>
         <CHAR>'/dev/i2c-1'</CHAR>
      </parameters>
   </sysname>
</username>

<username name="sig1" line="9">
   <sysname name="FixedRangeSignal"/>
</username>

<username name="int1" line="10">
   <sysname name="UnixSignal">
      <parameters>
         <FIXED>15</FIXED>
      </parameters>
   </sysname>
</username>

<username name="disc" line="12">
   <sysname name="Disc">
      <parameters>
         <CHAR>'/tmp/folder1'</CHAR>
         <FIXED>10</FIXED>
      </parameters>
   </sysname>
</username>

<username name="stdOut" line="14">
   <sysname name="StdOut">
   </sysname>
</username> 

<configuration line="16">
   <sysname name="Log">
      <parameters>
         <CHAR>'EW'</CHAR>
      </parameters>
      <association name="StdOut">
      </association>
   </sysname>
</configuration>
</system>
<problem>
<spc type="signal" name="sig1" line="19"/>
<spc type="interrupt" name="int1" line="20" />
<spc type="dation" name="disc" line="21">
   <attributes> OUT,SYSTEM, DIRECT </attributes>
   <data>ALL</data>
</spc>
<spc type="dation" name="stdOut" line="22">
      <attributes> OUT, SYSTEM </attributes>
      <data>ALPHIC</data>
</spc>
<spc type="dation" name="lm75" line="23">
      <attributes> IN, SYSTEM </attributes>
      <data>FIXED(15)</data>
</spc>
</problem>
</module>
\end{XMLCode}

\section{Problem Part Elements}
\subsection{Specification of System Part Elements}
The PEARL source code defined the type of a system name to be ether a DATION, 
INTERRUPT or SIGNAL. According detected type, the compiler 
adds a xml-tag \verb|<spc>|-tag with attributes \verb|type| anf \verb|line|
containing the value \verb|"DATION"|, \verb|"INTERRUPT"| or \verb|"SIGNAL"|,
 respectively.

While interrupts and signals have no parameters, dations need more
specifications. The attributes are stores as comma separated list in the
\verb|<attributes>|-subelement, the transmission data is given as value
of the <data>-tag. For details aboutencoding of the type information
see chapter \ref{encoding}. 

\subsection{Declaration of Problem Part Elements}

Additional type information for the \verb|type| attribute are introduced.
Simple types are donated by their PEARL type with length or precision.
Array are denoted with their virtual dimention list prepended separated by
one space character.
REF of any kind are denoted with  \texttt{REF} pnd space prepended the the
type.

Structs are denoted by their encoding rules of
structures and structure elements  as described in \cite{runtime}.

Dations contains an attribute- and data-tag like in the system part notation.

\subsection{Specification of Problem Part Elements}
The specification of global  elements in the problem part are described 
in the same as their declaration.

\subsection{Example for Specification and Declarations in Problem Part}

\begin{PEARLCode}
PROBLEM;
   DCL x FIXED(7) GLOBAL;
   SPC y FIXED(17) GLOBAL;
   DCL d12 DATION OUT STRUCT [a FIXED(7), b FLOAT(53)]
     DIM(10,20)  GLOBAL CREATED(aSystemDation);
   DCL s1 SEMA PRESET(10) GLOBAL;
\end{PEARLCode}

\begin{XMLCode}
...
<problem>
  <dcl type="FIXED(7)" name="x" line=".."/>
  <spc type="FIXED(17)" name="y" line=".."/>
  <dcl type="dation" name="d12" line="10">
    <attributes>OUT DIM(10,20)</attributes>
    <data>S5A7B57</data>
  </dcl>
  <dcl type="SEMA" name="s1" line="..."/>
</problem>
\end{XMLCode}


\subsection{Encoding of Transfer Data Types and STRUCTs}
\label{encoding}
The primitive data types and their length are encoded 
by their PEARL notation.  E.g.\verb|FIXED(7)|. The encoding rules
for struct and arrays are expected as described in \cite{runtime}. 

Special values are ALL and ALPHIC.

\begin{description}
\item[ALL] fits to all data types
\item[ALPHIC] in problem part fits to ALPHIC, CHAR and ALL in system devices
\end{description}

This mapping is also used for problem part global elements, except for dations,
which need more information to be checked.

\subsection{TFU}
The treatment of \texttt{TFU} differs, whether the system dation
requires \texttt{TFU} or not.

If the system dation requires \texttt{TFU}  the user dations created
 upon this system dation must use TFU.
The specification of a dation does not contain any information about TFU.

Example:

\begin{PEARLCode}
SYSTEM;
  con: Console;
PROBLEM;
  SPC con     DATION INOUT ALPHIC SYSTEM;
  DCL console DATION INOUT ALPHIC DIM(*,80) TFU FORWARD CREATED(con);
  DCL consol2 DATION INOUT ALPHIC DIM(*,90) TFU FORWARD CREATED(con);
  DCL noTFU   DATION INOUT ALPHIC DIM(*,80)     FORWARD CREATED(con);
\end{PEARLCode}

The name \texttt{noTFU} should cause an error message,
since the system dation \texttt{Console} requires \texttt{TFU}.

The compiler  has no information about the capabilities and details
of thesystem part elements.

To check the necessity of \texttt{TFU} and requested sizes the compiler 
exports all relevant information for the checks about \texttt{TFU}
In addition to the declaracti0on and specification of the element \texttt{con}
a separate section is exported like:

\begin{XMLCode}
 <problem>
  ...
  <tfuusage>
    <userdation name="console" lineno="23", col="7">
       <systemdation name="con"/>
       <usedTFU size="80"/>
    </userdation>
    <userdation name="console2" lineno="24", col="7">
       <systemdation name="con"/>
       <usedTFU size="90"/>
    </userdation>
    <userdation name="noTFU" lineno="25", col="7">
       <systemdation name="con"/>
    </userdation>
  </tfuusage>
\end{XMLCode}

The compiler can calculate the record size for typeOfTransmission ALL,ALPHIC or simpleType. For typeStructure - the compiler does not known the allignment
restrictions from the target depending g++-compilation.
If the type of transmission is a STRUCT - the size will be set to 0
and the check must be done at system start,
when the userdation becomes instanciated.

With this information the imc may check
\begin{itemize}
\item if the record size  of \texttt{console} fits with the requirements of 
    \texttt{con}
\item if the userdation uses \texttt{TFU} at all
\end{itemize} 

Application specific error messages can becom procuded, like:

\lstset{breaklines=true}
\begin{lstlisting}
demo.prl:24:7: error: TFU record of dation 'console2' exeeeds limit of 'Console'
demo.prl:25:7: error: dation 'noTFU' must use TFU, which is required by 'Console'
\end{lstlisting}


\section{Multi-Module Support}
The imc supports basic operation for multiple modules.
The module name defines a kind of namespace for all global symbols.

It is possible to ignore the module name with the command line parameter 
\texttt{-noNameSpace}.

The compiler does not provide the information for multiple modules.

The spc-tag may have an attribute \texttt{GLOBAL} with the name of
the module which should provide the symbol. The attribute \texttt{GLOBAL} 
default for the module name for convenience to import system part elements
in the same module.

