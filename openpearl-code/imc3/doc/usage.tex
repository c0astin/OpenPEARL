\chapter{Usage}

\section{Invocation}
The inter module checker is invoked with
\begin{verbatim}
imc -b <target> -o <output> pearlModules...
\end{verbatim}

The list of pearl modules is passed without extension. Pathnames are 
possible.

\section{Options}
\begin{description}
\item[-b] The option -b is  recommended. The given parameter defines the
platform definition XML file, which is expected ether in the current 
working directory (first)
or in installation directory of OpenPEARL.
\item [-o] defaults to system.cc in the current working directory. This is
  the name of the source file for the system part.
\item[-std=OpenPEARL] is default. Multiple system parts are allowed and the 
    module name defines the namespace.
\item[-std=PEARL90] allows only one system part. The module name is only used 
  for documentation. The C++ code goes into the namespace \texttt{pearlApp}.
\item[-I ] defines the installation path of the platform definition files
\item[--verbose] enables more messages 
\item[--help] prints a help message with all supported options.
\end{description}




