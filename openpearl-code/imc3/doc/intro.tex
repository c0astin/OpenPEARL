\chapter{Introduction}
A PEARL program consists of several separatly compilable modules.
The interface between the modules is defined by SPC and DCL statements with the
attribute GLOBAL.
The GLOBAL statement has an optioal parameter for the module which defines the symbol.
In OpenPEARL, this parameter default to the actual module name. This suits for the specification
of SYSTEM part elements of the same module. 
The declaration of plattform dependent elements like INTERRUPTs, 
DATIONs and SIGNALs must be done in the SYSTEM-part.

For proper operation the types of the specified and declared elements must
fit. 
For system elements, we need a mechanism to obtain the type in an
extensible way. This is solved by the use of an interface definition as 
xml-file for each system element.

This tool checks all global definitions and specifications and checks for
\begin{itemize}
\item proper types, including the types of system elements
\item multiple declarations
\item missing declarations
\item unused declarations
\end{itemize}

\section{Structure}
As shown in \cite{mueller2016} the structure of operation of the \OpenPEARL{}
compilation system is like:
\newcommand*\circled[1]{\tikz[baseline=(char.base)]{
   \node[shape=circle,fill=white,draw,minimum size = 0.5cm, inner sep=2pt] (char) {#1};}}

\begin{figure}[bpht]

  \begin{tikzpicture}[
    >=stealth,
    node distance=3.5cm,
    file/.style={
      cylinder,
      cylinder uses custom fill,
      %cylinder body fill=yellow!50,
      %cylinder end fill=yellow!50,
      shape border rotate=90,
      minimum width=3cm,
      aspect=0.25,
      draw
    },
        fileold/.style={
          cylinder,
          cylinder uses custom fill,
          cylinder body fill=gray!20,
          cylinder end fill=gray!20,
          shape border rotate=90,
          minimum width=3cm,
          aspect=0.25,
          draw
        },
    block/.style = { rectangle,
    				draw=black,
    				 thick, 
                    %fill=blue!20,
                    text centered, minimum width=5em,
                    %rounded corners,
                    inner sep = 2pt,
                    minimum height=4em },
    blockold/.style = { rectangle,
       				draw=black,
       				 thick, 
                       fill=gray!20,
                       text centered, minimum width=5em,
                       %rounded corners,
                       inner sep = 0.5em,
                       minimum height=4em }                          
  ]
    \node[fileold] (prl) at (0,10) {\shortstack{PEARL\\Sourcecode}};
    \node[blockold] (sprachumsetzer) at (5,10) {\circled{1} Compiler};
    \node[fileold,below of=prl,yshift=0.5cm] (problemcc) {\shortstack{C++ Code\\for PROBLEM-\\part}};
    \node[file,below of=sprachumsetzer,yshift=0.5cm] (modulxml) {\shortstack{XML Description
                                                            \\of \\
                                                             export+import}};
    \draw[->] (prl) -- (sprachumsetzer);
    \draw[->] (sprachumsetzer) -- (problemcc);
    \draw[->] (sprachumsetzer) -- (modulxml);

    \node[file, yshift=-1cm,right of=modulxml] (hardwarexml) {\shortstack{XML description \\of predefined\\system names}};
    \node[block, below of=modulxml,yshift=0.5cm] (imc) {\circled{3} InterModuleChecker};

    \draw [->] (modulxml) --  (imc);
    \draw [->] (hardwarexml) -- (imc);
    
    \node[file, below of=imc,yshift=0.5cm] (systemcc) {\shortstack{C++ Code\\ for SYSTEM- \\part}};
    \node[fileold, below of=systemcc, yshift=1.5cm, xshift=1cm] (runtimelib) {\shortstack{runtime\\library}};
    \node[blockold, left of=systemcc,xshift=-2cm] (gcc)
    {\circled{2}
    {\shortstack{gcc\\tool chain}}};
    \draw[->] (imc) -- (systemcc);
    \draw[->] (systemcc) -- (gcc);
    \draw[->] (runtimelib) -- (gcc);
    \draw[->] (problemcc) -- (gcc);
    
    \node[fileold, below of=gcc,yshift=0.5cm] (pearlapp) {\shortstack{PEARL\\application}};
    \draw[->] (gcc) -- (pearlapp);
  \end{tikzpicture}
\caption{Structure of the \OpenPEARL{} build system.}
\label{aufbau}
\end{figure}



The first stage compiler creates an C++-representation of the PEARL-module
together with an xml-representation of the export and import interface.

The definition of system names may be done by the user to order to supply 
additional system devices.
Each system name must be accompanied by the developer with a xml-representation
of the system element. The imc tool extracts the required information
from these xml-files and verifies a proper system configuration.

\section{Credits}
This tool  was influenced by the master thesis of 
Stephan Hertig \cite{msc_hertwig}, the semester project of
M.Bauer, T-Schaz, J.Weber, T.Welte and J.Wirth \cite{openpearlss16} and
the master thesis of M. Beyer \cite{msc_beyer}.


