\chapter{Parallel Activities}  % 9
\label{sec_parallel_activities}
\index{Task}

Typical for a program to control a technical process are
\begin{itemize}
\item asynchronous processes as program parts which run parallel in time 
and independent from one another, and which are initiated by
spontaneous events or at certain (scheduled) times, as well as
\item the synchronisation of such processes at certain program points,
e.g., to be able to exchange data with one another.
\end{itemize}

To program such processes, in PEARL tasks, interrupts and
synchronisation objects are used.

A task is of a program part which is executed under the control of the
operating system. The task body, like a procedure body consists of PEARL 
declarations and statements. Before using it, a task must be declared; 
upon this, an identifier is assigned to it, which is then mentioned 
in tasking operations such as start or resumption.

Since usually only one processor is available to the tasks of a program,
they must compete for its use --- but also for the access to other
resources (such as I/O devices) which have to be shared. The
operating system, assigns the resources taking account of
the tasks' priority. Thus, a positive integer can be allocated to a task
as priority, smaller numbers meaning higher priority. The thus
determined task priorities are used by the operating system to control
the allocation of resources. If a task, e.g., possesses the only
processor and requests another, exclusively occupied resource, the
operating system withdraws the processor from this task and allocates it to
the task of highest priority among all runable  tasks waiting for the
processor. To runnable tasks of the same priority, the processor is
allocated due to the Round-Robin-Strategy.

Such a priority controlled re-allocation of a processor takes place each
time, when one of its operating system functions is called, e.g., upon
the occurrence of an interrupt or upon executing statements for task 
control, for synchronisation and for input and output.

\section{Declaration and Specification of Tasks (TASK)}   % 9.1
\label{sec_task_dcl}
\index{TASK@\textbf{TASK}|textbf}
\index{Task!Declaration}
\index{Task!Specification}

The declaration of tasks takes place analogously to the declaration of
procedures.
%%% In contrast to 
Like procedures, tasks may only be declared at
module level --- i.e., not within procedure or task bodies. Furthermore,
parameters are not permitted. However, since in a task body all PEARL
objects declared at module level may be used and --- as far as it is
possible --- changed, the data exchange with a task can take place via
data which are declared at module level. Particularly the access of
several tasks to the same data, however, should carefully be
synchronised with the means described in \ref{sec_task_sync}. In the case of the data 
exchange by I/O statements, the input/output functions
already take over the synchronisation.

Example:

A task \code{protocol} deposits protocol texts in the variable 
\code{text} to be
delivered to a terminal via the task \code{output}.
 (The needed synchronisation
statements are explained in \ref{sec_task_sync}.)

\begin{lstlisting}
PROBLEM;
   DCL text CHAR(60);
protocol: TASK; 
   text := protocolText(27);
   ...
END; ! protocol 

output: TASK;
  PUT Text TO printer BY LIST;
  ...
END; ! output
...
\end{lstlisting}

The general form of a task declaration reads:

\begin{grammarframe}
\inputBNF{TaskDeclaration.bnf}
\index{TASK@\textbf{TASK}}
\index{MAIN@\textbf{MAIN}}
\inputBNF{PriorityAttribute.bnf}
\index{PRIO@\textbf{PRIO}}
\index{PRIORITY@\textbf{PRIORITY}|see {PRIO}}
\inputBNF{TaskBody.bnf}
\end{grammarframe}

If no priority is given in a task declaration, priority 255 is assumed.
%%%\begin{added}
The priority value must be in the range from 1 to 255. 
In case the priority value is not inside this range the 
\code{PrioOutOfRangeSignal} is induced.
Each target plattform must support at least the priorities 1 and 255.
In case a requested priority is not avaliable on the current 
\OpenPEARL{} plattform
the \code{PrioNotMappableSignal} is induced.
%%%\end{added}


The general form of a task specification reads:

\begin{grammarframe}
\inputBNF{TaskSpecification.bnf}
\index{SPC@\textbf{SPC}!Task}
\end{grammarframe}


Upon program or system start, the tasks marked with \code{MAIN} are started
according to their priority. 
%%%All MAIN tasks must be declared in the same module.
%%%\begin{added}
As mentioned in section \ref{sec_program_execution}, \OpenPEARL{} 
accepts tasks with \code{MAIN} attribute in different modules.
%%%\end{added}

The global attribute is explained in \ref{sec_references}.

\subsection{References to Tasks (REF TASK)}   % 9.1.1
\label{sec_ref_task}
\index{Task!Reference}

The control of all task activities is also possible via references to
objects of type \code{TASK}. The declaration of corresponding reference
variables can take place like this:

\begin{grammarframe}
\inputBNF{TaskReferenceDeclaration.bnf}
\index{DCL@\textbf{DCL}!Task Reference}
\index{REF@\textbf{REF}!Task Reference}
\end{grammarframe}

The general form of the specification of task reference variables reads:

\begin{grammarframe}
\inputBNF{TaskReferenceSpecification.bnf}
\index{SPC@\textbf{SPC}!Task Reference}
\end{grammarframe}

The following short example shows an application possibility for task
reference variables:

Example:

Both segments from two PEARL programs show how various tasks can be
started from a procedure, once via their names, once via references.

\begin{lstlisting}
PROBLEM;
   SPC (consumer_1, consumer_2, consumer_3) TASK GLOBAL;
   ...
startTask: PROC (index FIXED);
   CASE index 
   ALT
      ACTIVATE consumer_1;
   ALT
      ACTIVATE consumer_2;
   ALT
      ACTIVATE consumer_3;
   FIN;
END;
...
\end{lstlisting}

\begin{lstlisting}
PROBLEM;
   SPC (consumer_1, consumer_2, consumer_3) TASK GLOBAL;
   DCL consumers (3) INV REF TASK GLOBAL
      INIT (consumer_1, consumer_2, consumer_3);
...
startTask: PROC (index FIXED);
   IF index LE UPB consumers AND index GE LWB consumers
   THEN
      ACTIVATE consumers (index);
   FIN;
END; ! startTask
...
\end{lstlisting}

\subsection{Determining Task Addresses}  % 9.1.2

The address of a task can be obtained by calling the predefined
function \code{TASK} or simply by assigning a task identifier to a task
reference variable.

Example:

\begin{lstlisting}
SPC temperatureMeasurement TASK GLOBAL;
DCL ptrTask REF TASK;
...
ptrTask := TASK (temperatureMeasurement);
ptrTask := temperatureMeasurement;
\end{lstlisting}

Both assignments are equivalent, however, the first form should be used,
because it documents the address assignment in a better way.

The address of the running task is received by invoking the function
\code{TASK} without stating a parameter.

Example:

\begin{lstlisting}
...
ptrTask := TASK;  ! provides address of the task containing
                  ! this statement.
...
\end{lstlisting}

The general form of task reference assignment reads:

\begin{grammarframe}
\inputBNF{RefTaskAssignment.bnf}
\index{Assignement!Task Reference}
\end{grammarframe}

Besides using task reference variables in task statements, they can
also be used for identifying tasks (cf. \code{IS} and \code{ISNT} operators).

Example:

\begin{lstlisting}
IF ptrTask IS TASK(hello)
THEN
   ...
FIN;
IF ptrTask IS TASK
THEN
   /* points to running task */
   ...
FIN;
\end{lstlisting}


\subsection{Determining Task Priorities}   % 9.1.3

The (current) priority of a task can be obtained by the predefined
function \code{PRIO}\index{PRIO@\textbf{PRIO}}.
\index{Predefined Functions!\textbf{PRIO}}
 Without stating a parameter, it provides the priority of
the running task, and with a task name as parameter, it returns the
priority of that task.

Example:

\begin{lstlisting}
DCL currPrio  FIXED(15);
DCL currTaskA FIXED}(15);
SPC taskA     TASK GLOBAL;
...
currPrio  := PRIO;
prioTaskA := PRIO (taskA);
CONTINUE TaskA PRIO currPrio + 1;
\end{lstlisting}

\section{Clock Time Adjustment}

%%%\begin{added}
The internal clock is usually realized upon a realtime clock and 
periodic ticks, which defines the clock resolution. 

One problem arises if the absolute time does not
match the internal time  due to e.g.
daylight saving setting or other influences like temperature  drift.
An automatic reexecution of the task scheduling statements
 (see \ref{sec_tasking_statements})
may be wrong in some cases. 

To adress this problem, 
an element in the system part is provided in \OpenPEARL{}
to induce an interrupt, when a difference between a reference clock and
the \OpenPEARL{} specific time is detected.
In applications, which are sensitive
to the absolute time, this  interrupt should be treated 
and the task scheduling
statements reexecuted fitting to the requirements of the application.

Example:

\begin{lstlisting}
SYSTEM;
   clockChangeInterrupt: ClockDifference(3600, -20.5, +300.0) ---
                             Lpc17xxRTC;
\end{lstlisting}

If the difference of the hardware clock in \code{Lpc17xxRTC} is detected
 to be 
more than 20.5 seconds in the past, or more than 300 seconds in the future,
the \OpenPEARL{} clock base is changed and the interrupt triggered.
The check is repeated all 3600 seconds.

%%% should be treated in the plattform manual
%%%Another problem arises, if an usually small processor system provides no
%%%realtime clock. Thus no scheduling condition based on absolute times 
%%% can be satisfied. In this case the build process rejects the creation
%%%of the PEARL application, if AT or UNTIL
%%% (see \ref{sec_tasking_statements})
%%% is used.

%%%\end{added}


\section{Statements for Controlling Tasks}    % 9.2
\label{sec_tasking_statements}

A task can be started, terminated, suspended, continued, resumed and
descheduled.

%%%\subsection{Start Condition}   % 9.2.1
\subsection{Starting a Task (ACTIVATE)}   % 9.2.1
\label{sec_task_activate}
\index{ACTIVATE@\textbf{ACTIVATE}}
\index{TASK@\textbf{TASK}!Activation}

Unlike procedures, tasks are not executed immediately by calls, but in
dependence of time instants and interrupts. Thus, a task must first be
{\bf scheduled} for the start by a task control statement of the
syntax

\begin{grammarframe}
\inputBNF{TaskStarting.bnf}
\end{grammarframe}

The start itself is managed by a real time operating system. If
the \hlink{StartCondition} defined in the control statement is fulfilled,
the task first becomes ``runnable''; but it is not started, i.e.,
transferred into the state ``running'', until it has become the runnable
task of highest priority. %%% (cf. also 9.2.2).

The start condition becomes ineffective 
Examples:

\kw{AT}\code{ 20:0:0 }\kw{ACTIVATE}\code{ statistics;}\\
The task ``\code{statistics}'' is not to be started until the
future time ``8 o'clock p.m.''.

\kw{ACTIVATE}\code{ statistics;}\\ 
the task ``\code{statistics}'' is to be started immediately.

With the start condition, tasks can be scheduled for cyclic (repeated)
start, in accordance with the syntax definition

\begin{grammarframe}
\inputBNF{StartCondition.bnf}
\index{AT@\textbf{AT}}
\index{AFTER@\textbf{AFTER}}
\index{WHEN@\textbf{WHEN}}

\inputBNF{Frequency.bnf}
\index{ALL@\textbf{ALL}}
\index{UNTIL@\textbf{UNTIL}}
\index{DURING@\textbf{DURING}}
\end{grammarframe}

%%%% begin added
The evaluation of all expressions in the \hlink{StartCondition} is performed 
at the moment when the \hlink{TaskStarting} statement is reached. 
%%%% end added

Examples:

\kw{ALL}\code{ t }\kw{ACTIVATE}\code{ regulator;}\\
The task \code{regulator} becomes runnable at each integer
multiple of the time interval \code{t}.

\kw{WHEN}\code{ alarm }\kw{ACTIVATE}\code{ shutDown;}\\
The task \code{shutDown} becomes runnable each time when the
interrupt \code{alarm} occurs.

\code{AT} \index{AT@\textbf{AT}} 
 Expression$\S $Time determines, at which time the task shall be
executed for the first time, \code{AFTER}
\index{AFTER@\textbf{AFTER}} Expression$\S $Duration determines,
from when on this is to happen relative to the execution of the task
control statement, and \code{WHEN} \index{WHEN@\textbf{WHEN}}
 Name$\S $Interrupt determines that this is
to happen when the denoted interrupt occurs, possibly delayed by the
duration denoted in \code{AFTER} Expression$\S $Duration.
 If \code{AT}, \code{AFTER} and \code{WHEN} 
are lacking, the task becomes runnable immediately after the execution
of the task control statement.

If the task is to be repeated after equal time intervals, the time between
two executions is to be determined by \code{ALL}\index{ALL@\textbf{ALL}}
 Expression$\S $Duration. 
To limit the repeated executions, a time can be determined by 
\code{UNTIL}\index{UNTIL@\textbf{UNTIL}} Expression$\S $Time, or 
a duration with
 \code{DURING}\index{DURING@\textbf{DURING}} Expression$\S $Duration,
respectively, after which no more task starts are repeated.

If the start condition commences with \code{WHEN}, the task becomes runnable
each time the denoted interrupt occurs, considering further components
of the start condition. A schedule determined by \code{ALL} becomes
re-effective, relative to the occurrence of the interrupt, whereas the
previous schedule becomes ineffective.

%%%\begin{added}
In combinations of \code{WHEN} and \code{AFTER} each interrupt occurance restarts
the delay period for the tasks activation.
This is shown in figure  \ref{whenAfter}.
The restart of the timer is also true for the frequency.
%%%\end{added}

\begin{figure}
\setlength{\unitlength}{1.00mm}
\linethickness{0.4pt}
\begin{picture}(140.00,30.0)
\footnotesize
\thicklines{}
\setlength\fboxsep{0pt}

\put(0,18){IruptTask}
\put(0,8){Interrupt occurs}
\put(20,05){\vector(1,0){120}}
\put(135,0){time}
\put(29.2,8){$\uparrow$}
% ################
\put(30,18){\colorbox{gray!20}{\framebox(10,5){delay}}} % 
\put(40.20,18){\framebox(10,5){T1}}

\put(52.2,8){$\uparrow$}
\put(53,18){\colorbox{gray!20}{\framebox(10,5){delay}}} % 
\put(54.2,8){$\uparrow$ (a)}
\put(55,22){\colorbox{gray!20}{\framebox(10,5){delay}}} % 
\put(65.20,22){\framebox(10,5){T1}}

\put(79.2,8){$\uparrow$}
\put(80,18){\colorbox{gray!20}{\framebox(10,5){delay}}} % 
\put(90.20,18){\framebox(25,5){T1}}
\put(93.2,8){$\uparrow$ (b)}
\put(94,22){\colorbox{gray!20}{\framebox(10,5){delay}}} % 
\put(115.45,22){\framebox(10,5){T1}}
\multiput(104,27)(2,0){5}{\line(1,0){1}}
  \put(114.5,27){\vector(1,0){1}}
\put(110.2,8){$\uparrow$ (c)}
\end{picture}
\caption{The task IruptTask becomes activated by an interrupt. 
The activation itself is delayed for 1 second. 
If the interrupt re-occurs during the delay period - the delay
period is retriggered (a).
If the interrupt occurs while the task is still active, the delay
timer is started again (b). If the task is still active at the end
of the delay time
the activation is buffered once and executed immediately after the 
task's termination (c).}
\label{whenAfter}
\end{figure}

%%%\removedtext{(i.e., the task associated with
%%%it is not executed any longer)} % verwirrende Formulierung
,
\begin{itemize}
\item if it commences with \code{AFTER} or \code{ALL} and 
an end condition determined
by \code{UNTIL} or \code{DURING} is reached, or if it has the form 
\code{AFTER} Expression$\S $Duration and the denoted Duration reckoning from the
execution of the task control statement has passed,
\item because a statement to deschedule a task is executed 
(see \ref{sec_prevent}),
\item 
 upon execution of a new task control statement of the same type 
%%%\addedtext{
(\code{ACTIVATE}).
%%%}.
The previous start condition is replaced by the new one
, or, if no new
condition is denoted, deleted.
\end{itemize}

Example:
\begin{lstlisting}
...
AT 12:0:0 ACTIVATE protocol;
ALL 2 HRS ACTIVATE protocol; 
...
\end{lstlisting}

These two schedulings cannot be valid at the same time. On the contrary,
start condition  \code{ALL 2 HRS} replaces start condition 
\code{AT 12:0:0}, unless
time ``12 o'clock'' does not occur before executing the second
statement.

%%%\subsection{Starting a Task (ACTIVATE)}   % 9.2.2
%%%
%%%The general form of the statement to start a task (i.e., to schedule a
%%%task to be started) reads:
%%%
%%%\begin{front}
%%%TaskStart ::=\\
%%%\x [ StartCondition ] {\bf ACTIVATE} Name$\S $Task [ Priority ];
%%%
%%%Priority ::=\\
%%%\x \{ {\bf PRIORITY $\mid$ PRIO} \} Expression$\S $WithPositiveIntegerAsValue
%%%\end{front}
%%%\begin{grammar}
%%%\inputBNF{TaskStarting.bnf}
%%%\inputBNF{PriorityExpression.bnf}
%%%\end{grammar}
%%%\begin{discuss}
%%%TaskStarting already inserted in sec start condition
%%%\end{discuss}
%%%%%\begin{discuss}
%%%%%Executing such a start statement causes that the indicated task...
%%%%%
%%%%%such streichen?
%%%%%\end{discuss}

Executing %%%%%such
 a start statement causes that the indicated task
applies for the allocation of a processor immediately (form without
start condition) or at the instant determined by the start condition ---
competing with all other tasks being runable at the moment
of the start (being runable) as well. Thus, upon an immediate start,
the started task competes particulary with the starting task, if only
one processor is provided.

Possibly, the indicated task has already been started and not been
terminated when executing the start statement. If in this case no
\hlink{StartCondition} is denoted, an error message occurs.
%%%%%\begin{discuss}
%%%%%so ok? 

However, if the start statement contains a \hlink{StartCondition},
 the indicated task is
%%%continued
not changed, and its re-start is scheduled (buffered) according to the
\hlink{StartCondition}, replacing a possibly existing scheduling by the new
one.

Possibly the task is still active when the \hlink{StartCondition}
 should activate
 the task.
 In this case the task will be re-started as soon as the task terminates.
This buffering is limited to 1 re-start action.
% as described in FullPEARL(1977)

This behavior is shown in figure~\ref{task-sched} and \ref{task-sched1}. The rectangles denote the time period, where the tasks are active. The arrows mark
the time of interaction.

%%%%%\end{discuss}

\begin{figure}
\setlength{\unitlength}{1.00mm}
\linethickness{0.4pt}
\begin{picture}(90.00,20.0)
\footnotesize
\thicklines{}
\put(0,8){Interrupt occurs}
\put(0,14){IruptTask}
\put(20,05){\vector(1,0){70}}
\put(85,0){time}
\put(39.2,8){$\uparrow$ (a)}
\put(40,12){\framebox(10,5){T1}}
\put(59.2,8){$\uparrow$}
\put(60,12){\framebox(10,5){T1}}
\put(63,8){$\uparrow$}
\put(65,8){$\uparrow$ (b)}
\put(70.3,12){\framebox(10,5){T1}}

\end{picture}
\caption{The task IruptTask becomes activated by an interrupt (a). 
If the interrupt occurs while the task is still active, the 
activation is buffered once and executed immediately after the 
task's termination (b).}
\label{task-sched}
\end{figure}

\begin{figure}
\setlength{\unitlength}{1.00mm}
\linethickness{0.4pt}
\begin{picture}(110.00,20.0)
\footnotesize
\thicklines{}
\put(0,10){CyclTask}
\put(0,0){1 second tick}
\put(10,05){\vector(1,0){100}}
\put(105,0){time}
\multiput(29.2,0)(20,0){4}{$\uparrow$}
\put(30,10){\framebox(10,5){T1}}

\put(50,10){\framebox(25,5){T1}}
\put(75.3,10){\framebox(5,5){T1}}

\put(90,10){\framebox(5,5){T1}}
\end{picture}
\caption{The task CyclTask should become activated all 1 sec. 
The execution time varies. If the task is still active at the 
next activation time, the 
activation is buffered once and executed immediately after the 
task's termination.}
\label{task-sched1}
\end{figure}

A possibly stated priority overwrites the priority stated in the
declaration of the indicated task. 
%%%\begin{added}
The new priority is valid only 
for this activation. Subsequent activations will use the priority, which
is specified in the task declaration.
%%%\end{added}

Example:

\kw{AT}\code{ 22:00:00 }\kw{ALL}\code{ 2 HRS }\kw{UNTIL}
\code{ 2:00:00  }\kw{ACTIVATE} t1;\\
will produce the activation of the task \code{t1} at the following time, 
depending on the time of the execution of the scheduled activation statement.

\begin{tabular}{l|l}
current time & activation times \\
\hline
12:00:00 & 22:00:00, 0:00:00 (tomorrow), 2:00:00 (tomorrow) \\
23:00:00 & 22:00:00 (tomorrow), 0:00:00 (day after tomorrow) ,\\
  &  2:00:00 (day after tomorrow) \\ 
\end{tabular}

Note that \OpenPEARL{} treats time intervals inclusive the borders.
%%%%%\begin{added}
%%%%%noch eingefuegt ...

To avoid problems from jitter, it is recommended that the 
activation freqency and border expression have a sufficient spacing.
%%%%%\end{added}

%%%%%\begin{discuss}
%%%%%ist das die richtige Stelle? Gilt auch bei CONTINUE!
%%%%%
%%%%%
Figure \ref{sched_act} illustrates the possibilities to start a task
immediately or under various start conditions.
%%%\end{added}

\setlength{\unitlength}{1.00mm}
\begin{figure}
\begin{picture}(145,145)
\newsavebox{\normalstmt}
\savebox{\normalstmt}(40,8)[bl]{
   \put(0,0){\line(1,0){40}}
   \put(0,8){\line(1,0){40}}
   \put(0,0){\line(0,1){8}}
   \put(40,0){\line(0,1){8}}
}
\newsavebox{\compare}
\savebox{\compare}(30,8)[bl]{
   \put(0,4){\line(1,1){4}}
   \put(0,4){\line(1,-1){4}}
   \put(4,0){\line(1,0){22}}
   \put(4,8){\line(1,0){22}}
   \put(26,0){\line(1,1){4}}
   \put(26,8){\line(1,-1){4}}
}

% show bounding frame
%\put(0,0){\line(1,0){145}}
%\put(0,0){\line(0,1){145}}
%\put(145,145){\line(0,-1){145}}
%\put(145,145){\line(-1,0){145}}

\put(60,140){\makebox(0,0){{\bf start by statement}}}
\put(110,140){\makebox(0,0){{\bf start by interrupt}}}
\put(110,130){\makebox(0,0){{\bf WHEN} Name\S Interrupt}}
\put(090,126){\usebox{\normalstmt}}
\put(110,126){\vector(0,-1){8}}
\put(60,126){\vector(0,-1){8}}
% --------------------

\put(60,114){\makebox(0,0){delayed start ?}}
\put(45,110){\usebox{\compare}}
\put(75,114){\line(1,0){4}}
\put(74,110){y}
\put(79,114){\line(0,-1){8}}
\put(60,106){\vector(0,-1){4}}
\put(60,106){\vector(1,0){50}}
\put(30,114){\line(1,0){15}}
\put(42,116){n}
\put(30,114){\line(0,-1){28}}
\put(30,86){\vector(1,0){55}}

\put(110,114){\makebox(0,0){delayed start ?}}
\put(095,110){\usebox{\compare}}
\put(110,110){\vector(0,-1){8}}
\put(112,107){y}
\put(125,114){\line(1,0){15}}
\put(127,116){n}
\put(140,114){\line(0,-1){28}}
\put(140,86){\vector(-1,0){55}}

\put(15,98){\makebox(0,0){\shortstack{delayed\\start}}}

\put(60,98){\makebox(0,0){\shortstack{{\bf AT} time\\e.g. 12:00:00}}}
\put(40,94){\usebox{\normalstmt}}
\put(60,94){\line(0,-1){4}}

\put(110,98){\makebox(0,0){\shortstack{{\bf AFTER} delay\\e.g. 5 SEC}}}
\put(090,94){\usebox{\normalstmt}}
\put(110,94){\line(0,-1){4}}

\put(60,90){\line(1,0){50}}
\put(85,90){\vector(0,-1){8}}

% ---------------------------------
\put(85,78){\makebox(0,0){periodic start ?}}
\put(70,74){\usebox{\compare}}
\put(85,74){\vector(0,-1){8}}
\put(87,71){y}
\put(30,78){\line(1,0){40}}
\put(67,80){n}
\put(30,78){\line(0,-1){60}}
\put(30,18){\vector(1,0){55}}

\put(85,62){\makebox(0,0){\shortstack{{\bf ALL} period\\(e.g.  5 SEC)}}}
\put(65,58){\usebox{\normalstmt}}
\put(85,58){\vector(0,-1){8}}

\put(15,62){\makebox(0,0){\shortstack{periodic\\start}}}

\put(85,46){\makebox(0,0){end condition ?}}

\put(70,42){\usebox{\compare}}
\put(70,46){\vector(-1,0){40}}
\put(67,48){n}
\put(85,42){\line(0,-1){4}}
\put(87,39){y}
\put(60,38){\line(1,0){50}}
\put(110,38){\vector(0,-1){4}}
\put(60,38){\vector(0,-1){4}}

% -----------------------
\put(60,30){\makebox(0,0){\shortstack{{\bf UNTIL} time\\(e.g. 12:30:00)}}}
\put(40,26){\usebox{\normalstmt}}
\put(60,26){\line(0,-1){4}}

\put(110,30){\makebox(0,0){\shortstack{{\bf DURING} duration\\(e.g. 20 MIN)}}}
\put(090,26){\usebox{\normalstmt}}
\put(110,26){\line(0,-1){4}}

\put(15,30){\makebox(0,0){\shortstack{end\\criteria}}}

\put(60,22){\line(1,0){50}}
\put(85,22){\vector(0,-1){8}}

\put(85,10){\makebox(0,0){{\bf ACTIVATE} Name\S Task}}
\put(65, 6){\usebox{\normalstmt}}
\end{picture}
\caption{Possibilities to start a task immediately or under 
various conditions. The period of the element \code{DURING} starts after
the {\em delayed start} has completed.}
\label{sched_act}
\end{figure}
% Bild. 9-9 einfuegen !!!

% Fussnote (im Bild): After occurrence of the interrupt, the task is
% re-scheduled under the related start condition. The old scheduling becomes
% ineffective.
%%%\begin{added}
Possible problem situations:
\begin{description}
\item[\kw{TaskRunningSignal}] is induced, if the task is still active when 
an \code{ACTIVATE} statement without start-condition is executed.
\item[\kw{IllegalSchedulingSignal}] is induced, if at least one the values of 
  \code{AFTER}, \code{ALL} or \code{DURING} is non positiv.
\end{description}
The first activation will be on the next day, if the current time is larger 
than the value of the \code{AT} parameter.
If the time specified by \code{UNTIL} is less than the time of \code{AT},
 the period will end the day after the first scheduled activation. 
This is true
independent of the way of specifying the first activation.
%%%\end{added}

A task is terminated
\begin{itemize}
\item when it reaches the terminating \code{END} statement of its body, or
\item by executing a statement to terminate tasks related to the task
(see \ref{sec_terminate}).
\end{itemize}

Example:

The task \code{pressureMeasurement} shall measure all 5 seconds
 the pressure in a tank and transfer it to the task \code{checking}.
 If the pressure increases significantly fast, the measurement
 shall by taken all second with increased priority. 
The task \code{checking} is started immediately by task \code{initial}.

\begin{lstlisting}
PROBLEM;

initial: TASK MAIN;
   ACTIVATE checking;
   /* further initialisations */
 
END; ! initial
       
checking: TASK PRIORITY 6;
   ALL 5 SEC ACTIVATE pressureMeasurement;
   /* taking over the measured values
      if the pressure rises: */
         ALL 1 SEC ACTIVATE pressureMeasurement PRIO 2;
         ...
END; ! checking
 
pressureMeasurement: TASK PRIO 5;
   /* measuring and passing over to checking */
END; ! pressureMeasurement
...
\end{lstlisting}



\subsection{Terminating a Task (TERMINATE)}    % 9.2.3
\label{sec_terminate}

The premature termination of a task is achieved by following
statement:

\begin{grammarframe}
\inputBNF{TaskTerminating.bnf}
\index{TERMINATE@\textbf{TERMINATE}}
\index{TASK@\textbf{TASK}!Termination}
\end{grammarframe}

If the denotation Name$\S $Task is lacking, the statement refers to
the task in whose body it is contained.

All resources occupied by the terminated task (including processor) are
withdrawn from it. Synchronisation variables blocked by the task,
however, are not released automatically.

%%%\begin{added}
Possible problem situations:
\begin{description}
\item[\kw{TaskTerminatedSignal}] is induced, if the task is in terminated state
 when a \code{TERMINATE} statement is executed.
\end{description}
%%%\end{added}

\subsection{Suspending a Task (SUSPEND)}   % 9.2.4
\label{sec_suspend}

By executing the statement

\begin{grammarframe}
\inputBNF{TaskSuspending.bnf}
\index{SUSPEND@\textbf{SUSPEND}}
\index{TASK@\textbf{TASK}!Suspending}
\end{grammarframe}

the indicated task --- or the executing task, respectively, if
Name$\S $Task is lacking --- is suspended, i.e., its execution is
postponed. The processor allocated to it is withdrawn from it --- but
not all the other resources occupied by it.

A suspended task can only be continued by executing a continue
statement (see \ref{sec_continue}).

%%%\begin{added}
Possible problem situations:
\begin{description}
\item[\kw{TaskTerminatedSignal}] is induced, if the task is not activated when 
a \code{SUSPEND} statement is executed.
\item[\kw{TaskSuspendedSignal}] is induced, if the task is already suspended when
a \code{SUSPEND} statement is executed.
\end{description}
%%%\end{added}


\subsection{Continuing a Task (CONTINUE)}    % 9.2.5
\label{sec_continue}

A suspended task can be continued immediately, at a certain point in
time, after a certain duration or after the occurrence of an interrupt by
the following statement:

\begin{grammarframe}
\inputBNF{TaskContinuing.bnf}
\index{CONTINUE@\textbf{CONTINUE}}
\index{TASK@\textbf{TASK}!Continuation}

\inputBNF{SimpleStartCondition.bnf}
\index{AT@\textbf{AT}}
\index{AFTER@\textbf{AFTER}}
\index{WHEN@\textbf{WHEN}}
\end{grammarframe}

%%%% begin added
The evaluation of all expressions in the \hlink{SimpleStartCondition} is performed 
at the moment when the \hlink{TaskContinuing} statement is reached. 
%%%% end added


If Name$\S $Task is stated, the statement causes that the indicated task
competes for the processor immediately (form without start condition) or
at the instant determined by the start condition --- possibly with the
given priority replacing the declared one.

The form without Name$\S $Task causes that the executing task re-competes
for the processor at the instant determined by the start condition ---
possibly with the given priority replacing the declared one.

Example:

The task \code{acquisition} shall cause an output and not continue until
the occurrence of the interrupt \code{continuation}, but then with higher
priority.

\begin{lstlisting}
acquisition: TASK PRIO 8;
   ! Acquisition of data
   WHEN continuation CONTINUE PRIO 5;
   ! Output
   SUSPEND;
   ! Acquisition of data
END; ! Acquisition
\end{lstlisting}

In Figure~\ref{zustaende},
the state transitions of a task possible by executing \code{CONTINUE}
instructions are illustrated graphically. For the case of
``scheduled continuations'', the state transitions for 
\code{SUSPEND} statements are
given additionally.
%%%\begin{added}
Note that fulfilling the continuation condition annihilates the
\hlink{SimpleStartCondition}.
%%%\end{added}

%%%\begin{added}
Possible problem situations:
\begin{description}
\item[\kw{TaskTerminatedSignal}] is induced, if the task is not activated when 
a \code{CONTINUE} statement is executed, without a \hlink{SimpleStartCondition}.
\item[Priority update] occurs at any \code{CONTINUE PRIO xxx} statement,
 except if the tasks state is terminated.
\item[executing a scheduled \code{CONTINUE}] never induces a signal. The action will
   be discarded, if the tasks state is not suspended. The priority will be 
   updated, if the tasks state is not terminated.
\end{description}
%%%\end{added}


\begin{figure}
\unitlength 1.00mm
\linethickness{0.4pt}
\begin{picture}(158.00,117.67)
\footnotesize
\thicklines{}
%\circle(15.00,46.00){18.00}
\multiput(15.00,55.00)(0.98,-0.11){2}{\line(1,0){0.98}}
\multiput(16.97,54.78)(0.31,-0.11){6}{\line(1,0){0.31}}
\multiput(18.84,54.14)(0.19,-0.12){9}{\line(1,0){0.19}}
\multiput(20.53,53.10)(0.12,-0.12){12}{\line(1,0){0.12}}
\multiput(21.95,51.72)(0.11,-0.17){10}{\line(0,-1){0.17}}
\multiput(23.03,50.06)(0.12,-0.31){6}{\line(0,-1){0.31}}
\multiput(23.72,48.21)(0.09,-0.65){3}{\line(0,-1){0.65}}
\multiput(24.00,46.25)(-0.08,-0.99){2}{\line(0,-1){0.99}}
\multiput(23.83,44.27)(-0.12,-0.38){5}{\line(0,-1){0.38}}
\multiput(23.24,42.38)(-0.11,-0.19){9}{\line(0,-1){0.19}}
\multiput(22.25,40.67)(-0.11,-0.12){12}{\line(0,-1){0.12}}
\multiput(20.91,39.21)(-0.16,-0.11){10}{\line(-1,0){0.16}}
\multiput(19.28,38.08)(-0.26,-0.11){7}{\line(-1,0){0.26}}
\multiput(17.45,37.34)(-0.65,-0.11){3}{\line(-1,0){0.65}}
\put(15.50,37.01){\line(-1,0){1.98}}
\multiput(13.52,37.12)(-0.38,0.11){5}{\line(-1,0){0.38}}
\multiput(11.61,37.66)(-0.22,0.12){8}{\line(-1,0){0.22}}
\multiput(9.87,38.60)(-0.14,0.12){11}{\line(-1,0){0.14}}
\multiput(8.38,39.90)(-0.12,0.16){10}{\line(0,1){0.16}}
\multiput(7.21,41.50)(-0.11,0.26){7}{\line(0,1){0.26}}
\multiput(6.41,43.31)(-0.09,0.49){4}{\line(0,1){0.49}}
\put(6.03,45.26){\line(0,1){1.98}}
\multiput(6.09,47.24)(0.10,0.38){5}{\line(0,1){0.38}}
\multiput(6.57,49.16)(0.11,0.22){8}{\line(0,1){0.22}}
\multiput(7.47,50.92)(0.11,0.14){11}{\line(0,1){0.14}}
\multiput(8.72,52.45)(0.14,0.11){11}{\line(1,0){0.14}}
\multiput(10.29,53.67)(0.22,0.11){8}{\line(1,0){0.22}}
\multiput(12.08,54.51)(0.58,0.10){5}{\line(1,0){0.58}}
%\end
%\circle(15.00,72.00){18.00}
\multiput(15.00,81.00)(0.98,-0.11){2}{\line(1,0){0.98}}
\multiput(16.97,80.78)(0.31,-0.11){6}{\line(1,0){0.31}}
\multiput(18.84,80.14)(0.19,-0.12){9}{\line(1,0){0.19}}
\multiput(20.53,79.10)(0.12,-0.12){12}{\line(1,0){0.12}}
\multiput(21.95,77.72)(0.11,-0.17){10}{\line(0,-1){0.17}}
\multiput(23.03,76.06)(0.12,-0.31){6}{\line(0,-1){0.31}}
\multiput(23.72,74.21)(0.09,-0.65){3}{\line(0,-1){0.65}}
\multiput(24.00,72.25)(-0.08,-0.99){2}{\line(0,-1){0.99}}
\multiput(23.83,70.27)(-0.12,-0.38){5}{\line(0,-1){0.38}}
\multiput(23.24,68.38)(-0.11,-0.19){9}{\line(0,-1){0.19}}
\multiput(22.25,66.67)(-0.11,-0.12){12}{\line(0,-1){0.12}}
\multiput(20.91,65.21)(-0.16,-0.11){10}{\line(-1,0){0.16}}
\multiput(19.28,64.08)(-0.26,-0.11){7}{\line(-1,0){0.26}}
\multiput(17.45,63.34)(-0.65,-0.11){3}{\line(-1,0){0.65}}
\put(15.50,63.01){\line(-1,0){1.98}}
\multiput(13.52,63.12)(-0.38,0.11){5}{\line(-1,0){0.38}}
\multiput(11.61,63.66)(-0.22,0.12){8}{\line(-1,0){0.22}}
\multiput(9.87,64.60)(-0.14,0.12){11}{\line(-1,0){0.14}}
\multiput(8.38,65.90)(-0.12,0.16){10}{\line(0,1){0.16}}
\multiput(7.21,67.50)(-0.11,0.26){7}{\line(0,1){0.26}}
\multiput(6.41,69.31)(-0.09,0.49){4}{\line(0,1){0.49}}
\put(6.03,71.26){\line(0,1){1.98}}
\multiput(6.09,73.24)(0.10,0.38){5}{\line(0,1){0.38}}
\multiput(6.57,75.16)(0.11,0.22){8}{\line(0,1){0.22}}
\multiput(7.47,76.92)(0.11,0.14){11}{\line(0,1){0.14}}
\multiput(8.72,78.45)(0.14,0.11){11}{\line(1,0){0.14}}
\multiput(10.29,79.67)(0.22,0.11){8}{\line(1,0){0.22}}
\multiput(12.08,80.51)(0.58,0.10){5}{\line(1,0){0.58}}
%\end
%\circle(71.00,23.00){18.01}
\multiput(71.00,32.01)(0.98,-0.11){2}{\line(1,0){0.98}}
\multiput(72.97,31.79)(0.31,-0.11){6}{\line(1,0){0.31}}
\multiput(74.84,31.14)(0.19,-0.12){9}{\line(1,0){0.19}}
\multiput(76.53,30.11)(0.12,-0.12){12}{\line(1,0){0.12}}
\multiput(77.95,28.73)(0.11,-0.17){10}{\line(0,-1){0.17}}
\multiput(79.03,27.07)(0.12,-0.31){6}{\line(0,-1){0.31}}
\multiput(79.73,25.21)(0.09,-0.65){3}{\line(0,-1){0.65}}
\multiput(80.00,23.25)(-0.08,-0.99){2}{\line(0,-1){0.99}}
\multiput(79.84,21.28)(-0.12,-0.38){5}{\line(0,-1){0.38}}
\multiput(79.25,19.38)(-0.11,-0.19){9}{\line(0,-1){0.19}}
\multiput(78.26,17.67)(-0.11,-0.12){12}{\line(0,-1){0.12}}
\multiput(76.92,16.21)(-0.16,-0.11){10}{\line(-1,0){0.16}}
\multiput(75.29,15.08)(-0.26,-0.11){7}{\line(-1,0){0.26}}
\multiput(73.45,14.34)(-0.65,-0.11){3}{\line(-1,0){0.65}}
\put(71.50,14.01){\line(-1,0){1.98}}
\multiput(69.52,14.12)(-0.38,0.11){5}{\line(-1,0){0.38}}
\multiput(67.61,14.66)(-0.22,0.12){8}{\line(-1,0){0.22}}
\multiput(65.87,15.60)(-0.14,0.12){11}{\line(-1,0){0.14}}
\multiput(64.38,16.90)(-0.12,0.16){10}{\line(0,1){0.16}}
\multiput(63.20,18.49)(-0.11,0.26){7}{\line(0,1){0.26}}
\multiput(62.41,20.31)(-0.10,0.49){4}{\line(0,1){0.49}}
\put(62.03,22.25){\line(0,1){1.98}}
\multiput(62.08,24.23)(0.10,0.38){5}{\line(0,1){0.38}}
\multiput(62.57,26.15)(0.11,0.22){8}{\line(0,1){0.22}}
\multiput(63.46,27.92)(0.11,0.14){11}{\line(0,1){0.14}}
\multiput(64.72,29.45)(0.14,0.11){11}{\line(1,0){0.14}}
\multiput(66.28,30.67)(0.22,0.11){8}{\line(1,0){0.22}}
\multiput(68.07,31.52)(0.59,0.10){5}{\line(1,0){0.59}}
%\end
%\circle(71.00,88.00){18.00}
\multiput(71.00,97.00)(0.98,-0.11){2}{\line(1,0){0.98}}
\multiput(72.97,96.78)(0.31,-0.11){6}{\line(1,0){0.31}}
\multiput(74.84,96.14)(0.19,-0.12){9}{\line(1,0){0.19}}
\multiput(76.53,95.10)(0.12,-0.12){12}{\line(1,0){0.12}}
\multiput(77.95,93.72)(0.11,-0.17){10}{\line(0,-1){0.17}}
\multiput(79.03,92.06)(0.12,-0.31){6}{\line(0,-1){0.31}}
\multiput(79.72,90.21)(0.09,-0.65){3}{\line(0,-1){0.65}}
\multiput(80.00,88.25)(-0.08,-0.99){2}{\line(0,-1){0.99}}
\multiput(79.83,86.27)(-0.12,-0.38){5}{\line(0,-1){0.38}}
\multiput(79.24,84.38)(-0.11,-0.19){9}{\line(0,-1){0.19}}
\multiput(78.25,82.67)(-0.11,-0.12){12}{\line(0,-1){0.12}}
\multiput(76.91,81.21)(-0.16,-0.11){10}{\line(-1,0){0.16}}
\multiput(75.28,80.08)(-0.26,-0.11){7}{\line(-1,0){0.26}}
\multiput(73.45,79.34)(-0.65,-0.11){3}{\line(-1,0){0.65}}
\put(71.50,79.01){\line(-1,0){1.98}}
\multiput(69.52,79.12)(-0.38,0.11){5}{\line(-1,0){0.38}}
\multiput(67.61,79.66)(-0.22,0.12){8}{\line(-1,0){0.22}}
\multiput(65.87,80.60)(-0.14,0.12){11}{\line(-1,0){0.14}}
\multiput(64.38,81.90)(-0.12,0.16){10}{\line(0,1){0.16}}
\multiput(63.21,83.50)(-0.11,0.26){7}{\line(0,1){0.26}}
\multiput(62.41,85.31)(-0.09,0.49){4}{\line(0,1){0.49}}
\put(62.03,87.26){\line(0,1){1.98}}
\multiput(62.09,89.24)(0.10,0.38){5}{\line(0,1){0.38}}
\multiput(62.57,91.16)(0.11,0.22){8}{\line(0,1){0.22}}
\multiput(63.47,92.92)(0.11,0.14){11}{\line(0,1){0.14}}
\multiput(64.72,94.45)(0.14,0.11){11}{\line(1,0){0.14}}
\multiput(66.29,95.67)(0.22,0.11){8}{\line(1,0){0.22}}
\multiput(68.08,96.51)(0.58,0.10){5}{\line(1,0){0.58}}
%\end
%\circle(130.00,23.00){18.00}
\multiput(130.00,32.00)(0.98,-0.11){2}{\line(1,0){0.98}}
\multiput(131.97,31.78)(0.31,-0.11){6}{\line(1,0){0.31}}
\multiput(133.84,31.14)(0.19,-0.12){9}{\line(1,0){0.19}}
\multiput(135.53,30.10)(0.12,-0.12){12}{\line(1,0){0.12}}
\multiput(136.95,28.72)(0.11,-0.17){10}{\line(0,-1){0.17}}
\multiput(138.03,27.06)(0.12,-0.31){6}{\line(0,-1){0.31}}
\multiput(138.72,25.21)(0.09,-0.65){3}{\line(0,-1){0.65}}
\multiput(139.00,23.25)(-0.08,-0.99){2}{\line(0,-1){0.99}}
\multiput(138.83,21.27)(-0.12,-0.38){5}{\line(0,-1){0.38}}
\multiput(138.24,19.38)(-0.11,-0.19){9}{\line(0,-1){0.19}}
\multiput(137.25,17.67)(-0.11,-0.12){12}{\line(0,-1){0.12}}
\multiput(135.91,16.21)(-0.16,-0.11){10}{\line(-1,0){0.16}}
\multiput(134.28,15.08)(-0.26,-0.11){7}{\line(-1,0){0.26}}
\multiput(132.45,14.34)(-0.65,-0.11){3}{\line(-1,0){0.65}}
\put(130.50,14.01){\line(-1,0){1.98}}
\multiput(128.52,14.12)(-0.38,0.11){5}{\line(-1,0){0.38}}
\multiput(126.61,14.66)(-0.22,0.12){8}{\line(-1,0){0.22}}
\multiput(124.87,15.60)(-0.14,0.12){11}{\line(-1,0){0.14}}
\multiput(123.38,16.90)(-0.12,0.16){10}{\line(0,1){0.16}}
\multiput(122.21,18.50)(-0.11,0.26){7}{\line(0,1){0.26}}
\multiput(121.41,20.31)(-0.09,0.49){4}{\line(0,1){0.49}}
\put(121.03,22.26){\line(0,1){1.98}}
\multiput(121.09,24.24)(0.10,0.38){5}{\line(0,1){0.38}}
\multiput(121.57,26.16)(0.11,0.22){8}{\line(0,1){0.22}}
\multiput(122.47,27.92)(0.11,0.14){11}{\line(0,1){0.14}}
\multiput(123.72,29.45)(0.14,0.11){11}{\line(1,0){0.14}}
\multiput(125.29,30.67)(0.22,0.11){8}{\line(1,0){0.22}}
\multiput(127.08,31.51)(0.58,0.10){5}{\line(1,0){0.58}}
%\end
%\circle(130.00,88.00){18.01}
\multiput(130.00,97.01)(0.98,-0.11){2}{\line(1,0){0.98}}
\multiput(131.97,96.79)(0.31,-0.11){6}{\line(1,0){0.31}}
\multiput(133.84,96.14)(0.19,-0.12){9}{\line(1,0){0.19}}
\multiput(135.53,95.11)(0.12,-0.12){12}{\line(1,0){0.12}}
\multiput(136.95,93.73)(0.11,-0.17){10}{\line(0,-1){0.17}}
\multiput(138.03,92.07)(0.12,-0.31){6}{\line(0,-1){0.31}}
\multiput(138.73,90.21)(0.09,-0.65){3}{\line(0,-1){0.65}}
\multiput(139.00,88.25)(-0.08,-0.99){2}{\line(0,-1){0.99}}
\multiput(138.84,86.28)(-0.12,-0.38){5}{\line(0,-1){0.38}}
\multiput(138.25,84.38)(-0.11,-0.19){9}{\line(0,-1){0.19}}
\multiput(137.26,82.67)(-0.11,-0.12){12}{\line(0,-1){0.12}}
\multiput(135.92,81.21)(-0.16,-0.11){10}{\line(-1,0){0.16}}
\multiput(134.29,80.08)(-0.26,-0.11){7}{\line(-1,0){0.26}}
\multiput(132.45,79.34)(-0.65,-0.11){3}{\line(-1,0){0.65}}
\put(130.50,79.01){\line(-1,0){1.98}}
\multiput(128.52,79.12)(-0.38,0.11){5}{\line(-1,0){0.38}}
\multiput(126.61,79.66)(-0.22,0.12){8}{\line(-1,0){0.22}}
\multiput(124.87,80.60)(-0.14,0.12){11}{\line(-1,0){0.14}}
\multiput(123.38,81.90)(-0.12,0.16){10}{\line(0,1){0.16}}
\multiput(122.20,83.49)(-0.11,0.26){7}{\line(0,1){0.26}}
\multiput(121.41,85.31)(-0.10,0.49){4}{\line(0,1){0.49}}
\put(121.03,87.25){\line(0,1){1.98}}
\multiput(121.08,89.23)(0.10,0.38){5}{\line(0,1){0.38}}
\multiput(121.57,91.15)(0.11,0.22){8}{\line(0,1){0.22}}
\multiput(122.46,92.92)(0.11,0.14){11}{\line(0,1){0.14}}
\multiput(123.72,94.45)(0.14,0.11){11}{\line(1,0){0.14}}
\multiput(125.28,95.67)(0.22,0.11){8}{\line(1,0){0.22}}
\multiput(127.07,96.52)(0.59,0.10){5}{\line(1,0){0.59}}
%\end
\put(14.67,72.00){\makebox(0,0)[cc]{\tekst 20mm,explicitly suspended;}}
\put(14.67,46.33){\makebox(0,0)[cc]
{\tekst 15mm,explicitly suspended and scheduled for continuation;}}
\put(70.67,88.00){\makebox(0,0)[cc]
{\tekst 12mm,runable and scheduled for continuation;}}
\put(130.00,88.00){\makebox(0,0)[cc]{\tekst 12mm,runable;}}
\put(71.00,23.33){\makebox(0,0)[cc]
{\tekst 12mm,running and scheduled for continuation;}}
\put(130.00,22.67){\makebox(0,0)[cc]{\tekst 12mm,running;}}
\put(13.33,7.00){\makebox(0,0)[cc]{``active''}}
\put(74.00,7.00){\makebox(0,0)[cc]{``executing''}}
\put(16.34,88.00){\makebox(0,0)[cc]{``suspended''}}
\put(41.33,34.67){\makebox(0,0)[cc]{\tekst 12mm,SUSPEND;}}
\put(45.00,54.66){\makebox(0,0)[cc]{\tekst 25mm,ContinuationCondition fulfilled;}}
\put(41.33,76.33){\makebox(0,0)[cc]{\tekst 12mm,SUSPEND t$_i$;}}
\put(100.67,19.00){\makebox(0,0)[cc]{\tekst 20mm,ContinuationCondition\\
CONTINUE PRIO p;}}
\put(144.67,50.67){\makebox(0,0)[cc]{\tekst 12mm,CONTINUE\\~[PRIO p];}}
\put(145.33,98.00){\makebox(0,0)[cc]{\tekst 20mm, CONTINUE t$_i$\\PRIO p;}}
\put(100.67,100.00){\makebox(0,0)[cc]
{\tekst 25mm,ContinuationCondition\\CONTINUE t$_i$ PRIO p;}}
\bezier{92}(123.33,16.33)(114.00,9.00)(104.67,15.00)
\bezier{80}(137.67,28.00)(143.33,36.00)(143.67,46.33)
%\bezvec{120}(97.33,103.67)(78.67,106.67)(76.33,96.00)
\put(76.33,96.00){\vector(-1,-3){0.2}}
\bezier{120}(97.33,103.67)(78.67,106.67)(76.33,96.00)
%\end
%\bezvec{80}(97.00,23.00)(92.33,28.33)(79.67,25.33)
\put(79.67,25.33){\vector(-4,-1){0.2}}
\bezier{80}(97.00,23.00)(92.33,28.33)(79.67,25.33)
%\end
\bezier{60}(121.67,92.00)(113.67,92.67)(108.33,97.00)
%\bezvec{104}(36.67,73.00)(26.67,65.33)(20.67,53.33)
\put(20.67,53.33){\vector(-1,-2){0.2}}
\bezier{104}(36.67,73.00)(26.67,65.33)(20.67,53.33)
%\end
\bezier{84}(62.00,24.67)(46.33,27.33)(44.00,32.33)
%\bezvec{76}(39.00,38.33)(34.67,45.00)(24.00,43.67)
\put(24.00,43.67){\vector(-1,0){0.2}}
\bezier{76}(39.00,38.33)(34.67,45.00)(24.00,43.67)
%\end
\bezier{72}(62.33,85.67)(50.00,81.67)(46.00,79.00)
%\bezvec{116}(144.33,55.33)(145.00,69.33)(136.67,81.67)
\put(136.67,81.67){\vector(-2,3){0.2}}
\bezier{116}(144.33,55.33)(145.00,69.33)(136.67,81.67)
%\end
\bezier{56}(139.00,87.33)(146.67,88.33)(146.67,94.33)
%\bezvec{96}(144.00,102.33)(135.00,109.00)(133.00,96.67)
\put(133.00,96.67){\vector(-1,-4){0.2}}
\bezier{96}(144.00,102.33)(135.00,109.00)(133.00,96.67)
%\end
\thinlines{}
\put(79.00,55.67){\oval(158.00,110.67)[]}
\put(106.50,55.50){\oval(97.00,105.00)[]}
\put(120.00,103.00){\makebox(0,0)[cc]{\tekst 20mm,Start condition\\fulfilled;}}
\bezier{48}(122.00,117.67)(118.67,113.00)(119.67,107.00)
%\bezvec{24}(120.67,99.33)(120.33,98.00)(124.00,95.33)
\put(124.00,95.33){\vector(1,-1){0.2}}
\bezier{24}(120.67,99.33)(120.33,98.00)(124.00,95.33)
%\end
\put(130.33,116.33){\makebox(0,0)[cc]{\tekst 12mm,from state\\``scheduled'';}}
\put(100.33,89.00){\makebox(0,0)[cc]{\tekst 20mm,ContinuationCondition\\fulfilled;}}
\bezier{72}(80.00,90.33)(86.33,95.33)(95.33,92.00)
%\bezvec{68}(105.33,85.33)(108.33,82.33)(120.67,85.67)
\put(120.67,85.67){\vector(4,1){0.2}}
\bezier{68}(105.33,85.33)(108.33,82.33)(120.67,85.67)
%\end
\bezier{48}(23.67,49.00)(30.67,49.00)(35.33,51.67)
%\bezvec{312}(48.33,59.33)(77.33,72.67)(122.67,82.00)
\put(122.67,82.00){\vector(4,1){0.2}}
\bezier{312}(48.33,59.33)(77.33,72.67)(122.67,82.00)
%\end
\put(105.00,66.00){\makebox(0,0)[cc]{\tekst 20mm,ContinuationCondition\\fulfilled;}}
\bezier{160}(77.33,29.33)(92.67,43.67)(101.67,60.67)
%\bezvec{88}(107.00,70.67)(112.33,78.00)(124.67,79.67)
\put(124.67,79.67){\vector(4,1){0.2}}
\bezier{88}(107.00,70.67)(112.33,78.00)(124.67,79.67)
%\end
\put(122.33,46.33){\makebox(0,0)[cc]
{\tekst 12mm,t$_i$ is highest priority\\runable task;}}
\put(134.33,63.67){\makebox(0,0)[cc]
{\tekst 20mm,t$_i$ not any longer\\highest priority\\runable task;}}
\bezier{132}(128.00,79.00)(115.33,68.67)(120.00,52.67)
%\bezvec{36}(123.67,40.33)(126.67,31.67)(126.67,31.67)
\put(126.67,31.67){\vector(1,-3){0.2}}
\bezier{36}(123.67,40.33)(126.67,31.67)(126.67,31.67)
%\end
\bezier{120}(134.00,31.67)(140.67,43.67)(135.67,59.33)
%\bezvec{36}(133.00,70.67)(131.33,74.33)(131.33,79.00)
\put(131.33,79.00){\vector(0,1){0.2}}
\bezier{36}(133.00,70.67)(131.33,74.33)(131.33,79.00)
%\end
\put(63.67,48.33){\makebox(0,0)[cc]
{\tekst 12mm,t$_i$ is highest priority\\runable task;}}
\put(80.67,61.00){\makebox(0,0)[cc]
{\tekst 20mm,t$_i$ not any longer\\highest priority\\runable task;}}
\bezier{100}(66.67,79.67)(60.00,69.67)(61.67,56.67)
%\bezvec{44}(63.67,41.67)(64.33,37.33)(66.33,31.33)
\put(66.33,31.33){\vector(1,-3){0.2}}
\bezier{44}(63.67,41.67)(64.33,37.33)(66.33,31.33)
%\end
\put(16.83,62.33){\oval(31.00,64.67)[]}
%------
\bezier{100}(73.67,31.67)(80.00,44.33)(80.00,55.67)
%\bezvec{60}(80.33,66.67)(80.33,75.00)(76.33,80.33)
\put(76.33,80.33){\vector(-1,2){0.2}}
\bezier{60}(80.33,66.67)(80.33,75.00)(76.33,80.33)
%\end
\end{picture}

\caption{State transition diagram of task $t_i$ when executing
CONTINUE}
\label{zustaende}
\end{figure}


\subsection{Delaying a Task (RESUME)}    % 9.2.6
\label{sec_resume}

If the running task shall release the processor allocated to it for a
certain duration or until the occurrence of a certain point in time or an
interrupt, respectively, (if it is to be delayed), the following
statement is to be executed:

\begin{grammarframe}
\inputBNF{TaskResuming.bnf}
\index{RESUME@\textbf{RESUME}}
\index{Task!Delay}
\end{grammarframe}

This statement if equivalent to the non-devisable combination of
the statements

\code{SimpleStartCondition }\kw{CONTINUE};\\
\kw{SUSPEND};

After having executed the statement 
%%%TaskDelay
\code{RESUME}, the start condition is
ineffective, i.e., the execution is once.

%%%\begin{added}
Note that \code{RESUME} annihilates a \hlink{SimpleStartCondition}
for the current task, due to the implicit \code{CONTINUE}-statement.
%%%\end{added}

Example:

The task \code{control} shall turn on a device and check 10 seconds later,
whether the device works as desired.

\begin{lstlisting}
control: TASK;
   ! turning on the device
   AFTER 10 SEC RESUME;
   ! ckecking the function of the device
   ...
END; ! control
\end{lstlisting}

\subsection{De-scheduling a Task (PREVENT)}    % 9.2.7
\label{sec_prevent}

Sometimes it is necessary to cancel the schedules existing for a task,
i.e., to take care that the start conditions related to this task become
ineffective. This can be achieved with the following statement:

\begin{grammarframe}
\inputBNF{TaskPreventing.bnf}
\index{PREVENT@\textbf{PREVENT}}
\index{Task!De-schedule}
\end{grammarframe}

This statement does not terminate the corresponding task; if
Name$\S $Task is not given, the statement effects the running task.

Example:

The procedure control called by a higher level task, gives a move
request to an AGV (Automatic Guided Vehicle) which has to send the
 ready message \code{ready} within
30 seconds in order to start the task \code{supply}.
If the ready message has not
arrived within 30 seconds, the task \code{malfunction} shall be started, and
the probably still possible, but delayed start of \code{supply} shall be
de-scheduled. In the normal case --- \code{ready} arrives within 30 seconds
--- the already scheduled start of \code{malfunction} must be de-scheduled.

\begin{lstlisting}
PROBLEM;
   SPECIFY ready INTERRUPT;
 
control: PROC (x FIXED,  /* X coordinate */
               y FIXED); /* Y coordinate */
    /* converting the coordinates taken over into a bit string,
       output of the bit string to the AGV  */
    WHEN ready ACTIVATE supply;
    AFTER 30 SEC ACTIVATE malfunction;
    ... 
END;  ! control
   
supply: TASK PRIO 3;
   PREVENT malfunction;
   ... 
END;  ! supply
   
malfunction: TASK PRIO 2;
   PREVENT supply;
   ... 
END;  ! malfunction 
\end{lstlisting}

\section{Synchronising Tasks}    % 9.3
\label{sec_task_sync}
\index{Task!Synchronisation}

Generally, tasks are run independently from one another. However, it may
happen that several tasks work on parts of a more complex, overall
problem and have to use certain resources, particularly data, jointly.

Example:

\begin{figure}[h]
\centering
\setlength{\unitlength}{0.00087500in}%
\begin{picture}(4148,2325)(249,-1705)
\put(1351,344){\line( 0,-1){405}}
\put(1351,-61){\line( 1, 0){900}}
\put(3151,-61){\line( 1, 0){900}}
\put(4051,-61){\line( 0, 1){450}}
\put(3151,-61){\vector( 0,-1){225}}
\put(2251,-61){\vector( 0,-1){225}}
\put(2701,-736){\vector( 0,-1){675}}
\thicklines
\put(2026,-736){\framebox(1350,450){}}
\put(1351,524){\makebox(0,0)[b]{\smash{\SetFigFont{10}{12.0}{rm}message}}}
\put(4051,524){\makebox(0,0)[b]{\smash{\SetFigFont{10}{12.0}{rm}message}}}
\put(2701,-601){\makebox(0,0)[b]{\smash{\SetFigFont{10}{12.0}{rm}buffer}}}
\put(4276,119){\makebox(0,0)[lb]{\smash{\SetFigFont{10}{12.0}{rm}producer2}}}
\put(1126,119){\makebox(0,0)[rb]{\smash{\SetFigFont{10}{12.0}{rm}producer1}}}
\put(2926,-1141){\makebox(0,0)[lb]{\smash{\SetFigFont{10}{12.0}{rm}output}}}
\put(2746,-1681){\makebox(0,0)[b]{\smash{\SetFigFont{10}{12.0}{rm}Protocol printer}}}
\end{picture}

\caption{Example for task co-ordination}
\label{puffer}
\end{figure}

According to Figure~\ref{puffer} the tasks \code{producer1} and \code{producer2}
 produce messages which they want to
store in a common buffer; by the task \code{output}, a message is taken from
the buffer and displayed in a formatted way on a protocol printer.

For a correct execution of the operations, they have to be co-ordinated
as follows:
\begin{itemize}
\item The task \code{output} creates only an output, if a message was buffered, which
means that it possibly waits that the buffer is filled by \code{producer1} or
\code{producer2}.
\item \code{producer1} and \code{producer2} can only buffer a message, if no
previous message is being sent out by \code{output}, i.e., they must possibly
be suspended until \code{output} has issued the previous message completely.
\item The tasks \code{producer1} and \code{producer2} to buffer a
message have to be mutually exclusive.
\end{itemize}

To control such co-ordination tasks, two types of synchronisation
variables are provided, namely semaphore and bolt variables. If the
access to jointly used resources (e.g., data, procedures, devices) is to
be co-ordinated, one synchronisation variable is associated to each
resource; then the using tasks execute a locking statement on the
associated synchronisation variable before and a release statement
after using a resource.

The association of synchronisation variables to resources does not take
place by PEARL statements; it is only assumed and is realised by using a
certain synchronisation variable always in combination with a certain
resource, only. Such an association is not restricted to data,
procedures, devices, etc., sometimes it must also be done for program
parts (e.g., for executing sub-tasks), which shall all be
terminated until a certain other program part (e.g., the main task) is to be
continued.

\subsection{Semaphore Variables (SEMA) and Statements (REQUEST,
RELEASE, TRY)} % 9.3.1
\label{sec_semaphores}

% Zeilenumbruch mitten im Wort RELEASE
\index{Task!Semaphore}

Semaphore variables can have non-negative integers as values. These
numbers represent the states ``free'' or ``locked'': Zero means state
``locked'' (the semaphore variable ``locks''), positive numbers mean the
state ``free''.  These states can only be changed by special request or
release statements.

Semaphore variables are declared as follows:

\kw{DCL}\code{ Identifier\_or\_IdentifierList }\kw{SEMA}\code{;}

Example:

\kw{DCL}\code{ (on, off) }\kw{SEMA;}

After its declaration, a semaphore variable has the state ``locked''.

If a semaphore variable explicitly is to assume the state ``locked'',
the following locking statement is to be executed:

\kw{REQUEST} Name$\S $Sema;
\index{REQUEST@\textbf{REQUEST}|textbf}

The effect of this statement depends on the actual value of the
indicated semaphore variable:
\begin{itemize}
\item If the value is greater than zero, it is decremented by 1.
\item If the value equals zero, it stays the same; the executing task,
however, is suspended and placed in a queue which is (implicitly)
associated to the semaphore variable (the task is blocked).
\end{itemize}

The execution of the release statement

\kw{RELEASE} Name$\S $Sema;
\index{RELEASE@\textbf{RELEASE}|textbf}

causes that the value of the indicated semaphore variable is incremented
by 1. Furthermore, the tasks in the queue of the semaphore variable are
released; they repeat their request statement in the order of their
priorities.

Example:

With these means, the problem of buffering messages can be solved as
follows:

For entering messages into the buffer, a semaphore variable
\code{intoBuffer}, for the output out of the buffer, a semaphore variable
\code{outOfBuffer} is declared. Upon declaration, \code{intoBuffer} and
\code{outOfBuffer} are implicitly initialised with ``locked''. The task
\code{consumer}  is started from outside. Before it starts the tasks 
\code{producer\_1} and
\code{producer\_2}, \code{intoBuffer} must receive the state ``free''.

\begin{lstlisting}
PROBLEM;
   DCL (intoBuffer, outOfBuffer) SEMA;

consumer: TASK;
   RELEASE intoBuffer;
   ACTIVATE producer_1;
   ACTIVATE producer_2;
   REPEAT
      REQUEST outOfBuffer;
         ! Output to the protocol printer
      RELEASE intoBuffer;
   END; ! Loop
END; ! consumer

producer_1: TASK;
  REPEAT
     ! Preparing the message 
     REQUEST intoBuffer;
        ! Buffering
     RELEASE outOfBuffer;
   END; ! Loop
END; ! producer_1

producer_2: TASK;
   REPEAT
   ! Preparing the message
      REQUEST intoBuffer;
         ! Buffering
      RELEASE outOfBuffer;
   END; ! Loop
END; ! producer_2
\end{lstlisting}

Explanation:\\
Output waits first for the release of code{outOfBuffer}, i.e., the
buffering of a message, because \code{outOfBuffer} can only be released by
\code{producer\_1} or \code{producer\_2}.
After the output has taken place, \code{consumer}
releases \code{intoBuffer}, i.e., admits buffering, and waits for a new
request. \code{producer\_1} and \code{producer\_2} can only buffer a message, if state
``free'' is defined for \code{intoBuffer}; after buffering a message, they
release \code{outOfBuffer} resulting in the continuation of \code{consumer}.

The state of a semaphore variable can be obtained with the monadic
operator

\kw{TRY} Name$\S $Sema;
\index{TRY@\textbf{TRY}|textbf}

As result, the operator provides a \code{BIT(1)} value to be used in all
logical expressions. If the semaphore variable is ``free'', the operator
executes a \code{REQUEST} statement and provides value \code{'1'B} as result. If the
semaphore variable has already state ``locked'', no REQUEST statement
is executed, and value \code{'0'B} is provided. Therefores a possible blocking 
of the running task can be avoided when using the \code{TRY} operator!

Example:

In a dialogue system the function \code{function1}
 may not be called twice at the same time, although the
blocking of a task calling \code{function1} has to be avoided.

\begin{lstlisting}
PROBLEM;
   DCL func_1 SEMA; 

dialogue: TASK;
   ...
   IF TRY func_1
   THEN
      ! Semaphore variable func_1 had state 'free' 
      CALL function1;      ! execute function
      RELEASE func_1;      ! release semaphore 
   ELSE
      ! Semaphore variable func_1 had state 'locked'
      PUT 'function 1 not possible at present' TO terminal;
  FIN;
  ...
END; ! dialogue 
\end{lstlisting}

%%%% begin added 2022-03-11 (rm)
\paragraph{Note:} OpenPEARL accepts in TRY also a list of semaphores like the operations 
REQUEST and RELEASE. 
With \code{-std=PEARL90} an error message is emitted, if a list of semphore is detected.
%%%% end added 

Due to erroneous synchronisations, tasks can constantly block one
another; such a deadlock, e.g., could arise due to the following program
organisation:

\begin{multicols}{2}
\begin{lstlisting}
/* Task t1  */
REQUEST s1; 
   1st segment 
REQUEST s2; 
   2nd segment 
RELEASE s2; 
RELEASE s1;
\end{lstlisting}
\columnbreak
\begin{lstlisting}
/* Task t2  */
REQUEST s2; 
   1st segment 
REQUEST s1; 
   2nd segment 
RELEASE s1; 
RELEASE s2;
\end{lstlisting}
\end{multicols}

If tasks \code{t1} and \code{t2} are in their first segments at the same time, a
deadlock occurs, for on the one hand, \code{t1} is blocked by the execution of
statement ``\code{REQUEST s2;}'', because \code{t2} has already built up the lock
with the semaphore variable \code{s2}, and on the other hand, \code{t2} is blocked by
the execution of statement ``\code{REQUEST s1;}''. If no other task executes
a release, the blockings remain permanently.

To avoid such situations, lists of semaphore variables can be stated for
locking statements.

By executing the statement

\kw{REQUEST} Name$\S $Sema [ , Name$\S $Sema ]$^{...}$ ;
\index{REQUEST@\textbf{REQUEST}!Multiple Semaphores}

the running task is suspended, if at least one of the indicated semaphore
variables is in state ``locked''; the task is not continued until none
of the semaphore variables is locking any longer. If none of the
indicated semaphore variables is locking, the executing task is
continued after decrementing the values of each of these semaphore variables by 1.

The statement

\kw{RELEASE} Name$\S $Sema [ , Name$\S $Sema ]$^{...}$;
\index{RELEASE@\textbf{RELEASE}!Multiple Semaphores}

has the effect, as if a release statement were executed for each of
the indicated semaphore variables without any interruption.

In the last example, the deadlock could be avoided by the following
program organisation:

\begin{multicols}{2}
\begin{lstlisting}
/* Task t1  */
REQUEST s1,s2; 
   1st segment 
   2nd segment 
RELEASE s2; 
RELEASE s1;
\end{lstlisting}
\columnbreak
\begin{lstlisting}
/* Task t2  */
REQUEST s1,s2; 
   1st segment 
   2nd segment 
RELEASE s1; 
RELEASE s2;
\end{lstlisting}
\end{multicols}

The order of the semaphore variables in the list is of no significance.

After the \code{TRY} operator, only one single semaphore variable may be
given, a list of semaphore variables is not allowed.

The general forms for the declaration of semaphore variables as well as
locking and release statements read:

\begin{grammarframe}
\inputBNF{SemaDeclaration.bnf}
\index{DCL@\textbf{DCL}!Semaphore}
\index{SEMA@\textbf{SEMA}}
\index{PRESET@\textbf{PRESET}}
\inputBNF{SemaStatement.bnf}
\inputBNF{TRYOperator.bnf}
\end{grammarframe}


Thus, also arrays of semaphore variables are possible; the various
elements are addressed by indices in the request and release
statements.

Semaphore variables must be declared at module level. Upon declaration,
semaphore variables can also be initialised explicitly by \code{PRESET}
statement; the given values are assigned to the corresponding
semaphore variables according to their order.

Example:

Upon declaration, values 3 and 5 shall be assigned to the semaphore
variables \code{s1} and \code{s2}, respectively.

\kw{DCL}\code{ (s1, s2) }\kw{SEMA PRESET}\code{(3, 5);}

The general form for the specification of semaphore variables reads:

\begin{grammarframe}
\inputBNF{SemaSpecification.bnf}
\end{grammarframe}

%%%\begin{added}
Possible problem situations:

\begin{description}
\item[\kw{SemaOverFlowSignal}] is induced, if too many release operations
   are executed and the range of a \code{FIXED(32)} would exceed.
\item[\kw{REQUEST} \code{s1,s1}] is successful, if the internal value of the 
   semaphore is larger than 1.

\end{description}

%%%\end{added}
\subsection{Bolt Variables (BOLT) and Statements (ENTER, LEAVE,
RESERVE, FREE)} % 9.3.2
\label{sec_bolt}
\index{Task!Bolt}

We assume that in different tasks the same data are used for
calculation, but not modified (e.g., comparison quantities for
supervision processes); additionally, another task shall re-establish
these data (e.g., establishing new supervision data). It shall be
granted that the processes for modifying and using data mutually exclude
each other; in contrast, the simultaneous use of data (by various tasks)
is desired.

These problems can principally be solved with the described locking and
release statements; the formulation, however, is relatively
complicated, and the run time for the execution considerable. Thus, four
more statements working on variables of data type bolt are provided.

A bolt variable can take on the states ``locked'', ``lock possible'', or
``lock not possible'', depending on whether the associated resource
is used exclusively, is free or is used simultaneously.

Bolt variables, e.g., are declared as follows:

\kw{DCL}\code{ Identifier\_or\_IdentifierList }\kw{BOLT}\code{;}

After its declaration, a bolt variable has the state ``lock possible''.

Let the bolt variable \code{b} be associated to a resource. When entering a
critical section for the exclusive use of this resource, the access by
other tasks is locked by executing the statement

\kw{RESERVE}\code{ b;}

When leaving this critical section, the release takes place by the
statement

\kw{FREE}\code{ b;}

The critical sections for the simultaneous use of the resource are
initiated and terminated, respectively, by executing the statements

\kw{ENTER}\code{ b;} or \kw{LEAVE}\code{ b;}

The exact description of the effect of theses statements considers the
state and the modification of the bolt variables:
\begin{description}
\item[Effect of \kw{RESERVE}\code{ b;}]
\index{RESERVE@\textbf{RESERVE}|textbf}
If \code{b} has state ``lock possible'', \code{b} assumes state ``locked''; otherwise,
the executing task is suspended and placed in a queue associated with \code{b}.
\item[Effect of \kw{FREE}\code{ b;}]
\index{FREE@\textbf{FREE}|textbf}
If \code{b} is in state ''locked'' \code{b} assumes state ``lock possible''. Furthermore, all tasks waiting in the
queue of \code{b} due to a \code{RESERVE} statement are released; the tasks repeat
their locking statements in the order of their priorities. If no task
is waiting due to a \code{RESEREVE} statement, all other waiting tasks, due to 
their \code{ENTER}-instruction, are released, in turn repeating the \code{ENTER} 
statement in the order of their priorities.

\item[Effect of \kw{ENTER}\code{ b};]
\index{ENTER@\textbf{ENTER}|textbf}
If \code{b} has state ``locked'', or if a task is in the queue of \code{b} due to a
\code{RESERVE} statement, the executing task is suspended and placed in the
queue of \code{b}. Otherwise, \code{b} assumes state ``lock not possible'', to prohibit
exclusive access; furthermore, the (internally noted) number Z of
``using'' tasks is incremented by 1.\\
(I.e., tasks executing ``\code{ENTER b}'' are continued not before the tasks
requiring exclusive access to the resource.)
\item[Effect of \kw{LEAVE}\code{ b};]
\index{LEAVE@\textbf{LEAVE}|textbf}
If Z = 1, this statement causes the \code{b} assumes the state ''lock possible''.
Otherwise, Z is decremented by 1, and \code{b} keeps the state ``lock not
possible''.
\end{description}

 
Example:

A task \code{measurement} continuously acquires values of comparison
quantities needed by the tasks \code{control} and \code{disposition} for
calculations from a process to be supervised. It shall be granted that
\code{measurement} changes the comparison quantities only if they are not
used; in contrast, \code{control} and \code{disposition} shall use the
comparison quantities simultaneously.

For this, a bolt variable \code{valueAccess} is declared; in the bodies of the three
tasks the critical sections of the modification or the use of the
comparison quantities are initiated or terminated, respectively, as
follows:

\begin{multicols}{2}
\begin{lstlisting}
/*  body of measurement:  */

...
RESERVE valueAccess;
   ! modification
FREE valueAccess;
...
\end{lstlisting}
\columnbreak
\begin{lstlisting}
/*  bodies of control
          and disposition: */
...
ENTER valueAccess;
   ! use
LEAVE valueAccess;
...
\end{lstlisting}
\end{multicols}

All statements for bolt variables are also defined for lists of bolt
variables --- analogously to the request and release statements for
semaphore variables.

Generally, bolt variables can be declared and used as follows:

\begin{grammarframe}
\inputBNF{BoltDeclaration.bnf}
\index{BOLT@\textbf{BOLT}}
\index{DCL@\textbf{DCL}!Bolt}

\inputBNF{BoltSpecification.bnf}
\index{SPC@\textbf{SPC}!Bolt}
\inputBNF{BoltStatement.bnf}
\index{ENTER@\textbf{ENTER}}
\index{LEAVE@\textbf{LEAVE}}
\index{RESERVE@\textbf{RESERVE}}
\index{FREE@\textbf{FREE}}
\index{BOLT@\textbf{BOLT}!\textbf{ENTER}}
\index{BOLT@\textbf{BOLT}!\textbf{LEAVE}}
\index{BOLT@\textbf{BOLT}!\textbf{RESERVE}}
\index{BOLT@\textbf{BOLT}!\textbf{FREE}}
\end{grammarframe}

%Therefore, also arrays of bolt variables can be declared.
Also arrays of bolt variables can be declared.

Bolt declarations must take place at module level.

%%%%%\begin{added}
Possible problem situations:

\begin{description}
\item[\kw{BoltStateSignal}] is induced, if a \code{LEAVE} or \code{FREE} operation
   is executed and no \code{ENTER} or \code{RESERVE} operation (respectively) 
   was executed before.
\item [\kw{BoltEnterOverflowSignal}] if more than the range of \code{FIXED(32)} enter
   operations were executed in sequence.
\item[\kw{RESERVE} \code{b,b}] will never succeed, since a
    critical section cannot be locked twice.
\end{description}

%%%%%\end{added}


\section{Interrupts and Interrupt Statements}   % 9.4
\label{sec_interrupts}
\index{Task!Interrupt}
%\index{INTERRUPT@\textbf{INTERRUPT}}
\index{INTERRUPT@\textbf{INTERRUPT}!Statement}
\index{IRPT@\textbf{IRPT}|see {INTERRUPT}}

\subsection{Declarations of Interrupts and Software Interrupts}    % 9.4.1

An interrupt is a message of the controlled process via an interrupt
input line of the interrupt controller to the operating system, which,
after occurrence of the interrupt has to initiate the reaction planned
by the programmer for that case, e.g., ``Upon occurrence of interrupt
\code{ready}, task \code{supply} is to be started''.

The interrupt channels provided in a computer system are described in the
user manual, giving their (system) names. The interrupt channels required
by a PEARL program are declared in the system part, where user names can
be associated to them. Under these user names, they are specified as
interrupts in the problem part, to be able to use them in the task
control statements (see \ref{sec_tasking_statements}) and the interrupt statements.

Example:

\begin{lstlisting}
MODULE(moduleName); 
SYSTEM; 
   ready: Hard_Int(7); ! Hard_Int is the system name 

PROBLEM; 
   SPECIFY ready INTERRUPT;
   ...
initialisation: TASK;
   ... 
   WHEN ready ACTIVATE supply; \\
   ...
END; ! initialisation 
  
..
supply: TASK PRIORITY 2;
   ! task body 
END; ! supply
...    
\end{lstlisting}

The general form of an interrupt declaration and specification reads:

\begin{grammarframe}
\inputBNF{SystemInterruptDeclaration.bnf}
\index{INTERRUPT@\textbf{INTERRUPT}!System Part}

\inputBNF{InterruptSpecification.bnf}
\index{INTERRUPT@\textbf{INTERRUPT}!Specification}
\index{SPC@\textbf{SPC}!Interrupt}
\end{grammarframe}

Most modern computers and operating systems, however, do not provide the
possibility to address interrupt channels anymore, because hardware
oriented device control and supervision are taken over by programmable
logic controllers. The possibilities for interrupts provided by PEARL,
however, are also usable for so-called {\em software interrupts}
introduced into the system part under their own system names. The user manual
gives information about the system name and usage possibilities.

\subsection{Interrupt Statements (TRIGGER, ENABLE, DISABLE)}    % 9.4.2
\label{sec_task_interrupt}

First, interrupts are disabled by their declarations and have to be
enabled by an enable statement.

If the effect of an enabled interrupt, e.g., upon malfunctions in the
technical process, shall be suppressed, i.e., start conditions for task
statements related to it shall stay ineffective although the interrupt
occurs, a disable statement is to be executed for this purpose, which
in turn can be revoked again by executing an enable statement.

\begin{grammarframe}
\inputBNF{InterruptStatement.bnf}
\index{ENABLE@\textbf{ENABLE}}
\index{DISABLE@\textbf{DISABLE}}
\index{TRIGGER@\textbf{TRIGGER}}
\end{grammarframe}

The validity range of the disable statement starts with its
execution and ends upon execution of an enable statement.

Example:

\begin{lstlisting}
PROBLEM;

   SPC alarm INTERRUPT;
   ...
init: TASK MAIN;
   WHEN alarm ACTIVATE recovery;
   ENABLE alarm;
   ... 
END; ! init

recovery: TASK PRIO 1;
   DISABLE alarm;
   ! checking the reason
   ! initiate reaction
   ENABLE alarm;
END; ! recovery
...
\end{lstlisting}

The test of real time programs requires sometimes to simulate the effect
of interrupts, particularly if the technical process has not been
connected to the computer, yet. For such simulations, the trigger
statement is provided.

Example:
\begin{multicols}{2}
\begin{lstlisting}
MODULE(moduleIrpt);
SYSTEM;         
        
   ready: Soft_Int; 
   ...            
PROBLEM;       
   ...            
   WHEN ready ACTIVATE control; 
   ...              
MODEND;            
\end{lstlisting}
\columnbreak
\begin{lstlisting}
MODULE(test);
PROBLEM;
     
SPC ready IRPT GLOBAL(moduleIrpt);

   ...
   /* e.g., at random times */
   TRIGGER ready;
... 
MODEND;
\end{lstlisting}
\end{multicols}

