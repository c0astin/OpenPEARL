% labels and refs added
% grammar for 'SYSTEM' added
% signals in put/get formats added; behavior clearified
% 2020-03-11: rm: rounding in E-format added
%	signal at B2/3/4 formats added
%	D-format accept sign at input and set optional '-' at output
% 2020-03-26: rm: StringSelection in i/o statements clearified
\chapter{Input and Output}   % 10
\label{sec_input_output}
The input and output statements enable the transfer of data from the
working memory of the computer to an external data station (output) and
vice versa, the transfer of data from an external data station to the
working memory (input). Primarily, data stations are standard peripheral
devices (printer, console, disk, magnetic tape, keyboard, etc.) or
process peripherals (sensors, actuators, embedded controllers, etc.).

On these {\em system defined} data stations, {\em user defined} data
stations can be created in the program to store data, e.g., on disks,
magnetic tapes, printers, etc.

In the I/O statements of the problem part, the data stations are
addressed under freely selectable, logical user names; the features of
the data stations are to be specified before mentioning these names.

In conventional programming languages, the association of the logical
identifiers (user names) for data stations used in the program to the
devices of a particular computer system is done by additional control
definitions when setting up a compiler program (job control
definitions). In PEARL, these associations take place in standardised
notation by declarations in the system part.

\section{System Part}    % 10.1
\label{dation_system_part}

The system part serves for describing the used I/O configurations of a
PEARL program. A system name must be associated with all devices of a
computer system which can directly be addressed with PEARL I/O
statements. A list of these devices and their system names can be
found in the PEARL user manual of the corresponding target system. In
the system part, freely selectable user names must be assigned to the
system names of the needed devices. The devices can only be addressed in
I/O statements of the problem part via these user names. 

\begin{grammarframe}
\input{SystemPart.bnf}
\end{grammarframe}
%%%\begin{added}

Devices may be connected directly to the computer or they are connected 
via a bus like I2C to the computer. In this case, the bus is represented
by a separate user name declaration. The identifier of the bus may
be used in conjuntion with the connection symbol (-\,-\,-) to
a bus component. If the bus component is only used once in the system part,
it may be specified directly. In other cases, a user supplied name
must be introduced and used.


Thus, a system part consists of a number of device associations of the
following form:
%%%\end{added}


\begin{grammarframe}
\input{UserNameDeclaration.bnf}
\end{grammarframe}

The declaration of interrupts and signals are defined in
 chapter \ref{sec_interrupts} and \ref{signals}.

\begin{grammarframe}
\input{SystemDationDeclaration.bnf}

%\input{SystemSignalDeclaration.bnf}
%\input{SystemInterruptDeclaration.bnf}

\input{Association.bnf}
\input{ListOfConstants.bnf}
\input{ConstantParameter.bnf}
\end{grammarframe}
 
%%%\begin{added}
Especially for standalone running microcontrollers the system part
may configure the behavior of the PEARL environment, like the
definition of an error output channel. 

\begin{grammarframe}
\input{ConfigurationElement.bnf}
\end{grammarframe}

%%%\begin{removed}
%%%Example:
%%%\begin{tabbing}
%%%\x dateout: \= \kill
%%%{\bf Module} (example); \> \\
%%%{\bf SYSTEM}; \> \\
%%%\x tty0:    \> Lpc17xxUart(0, 9600, 8, 1, 'E');\\
%%%\x  \> Log('WE') -\,-\,- tty0; ! send errors and warnings to uart 0\\
%%%\x ...
%%%\end{tabbing}
%%%\end{removed}

%%%\end{added}
%%%\begin{added}

For the example of a system part as given below, a typical linux computer
system is considered, with the following provided system names in \OpenPEARL{}:

\begin{tabular}{ll@{\x}p{8cm}}
SystemName & Parameter & Device \\ \hline
StdIn  &  - & standard input \\
StdOut  &  - & standard output \\
Disc       & '/home', 10 & files for reading and writing are located in the given folder. The maximum of concurrent open files also specified. \\
RPiDigitalIn & ... & Digital input on an Rapberry Pi: Details are described in 
		the plattform users guide \\
RPiDigitalOut & ... & Digital output on an Rapberry Pi: Details are described in 
		the plattform users guide \\
\end{tabular}

PEARL example for the system part:

\begin{lstlisting}
MODULE (example); 
SYSTEM; 
   ! system dations 
   termin:   StdIn; 
   termout:  StdOut;
   folder:   Disc('/tmp' ,10) ;  
   counter:  RPiDigitalIn(2,1); 
   switch:   RPiDigitalOut(3,1); 
   motor:    RPiDigitalOut(8,4); ! Bit 8-5
   ic2bus1:  I2CBus('/dev/i2c-0', 100000);
   temp:     LM75('48'B4) --- i2cbus1;
   temp1:    LM75('49'B4) --- i2cbus1;

   ! configuration elements 
   Log('EW') --- LogFile('pearl_log.txt') --- folder; 
PROBLEM;
 ...
\end{lstlisting}

%%%\end{added}

%%%\begin{added}
The denotation of bits as used in \code{RPiDigitalIn} and \code{RPiDigitalOut}
in the system part is traditionally
defined by the starting bit number and number of bits.

The configuration item for logging denotes that errors and warnings ('EW')
should be gathered in the file '\code{pearl\_log.txt}' in the disc folder specified
by the previous defined \code{folder} on the disc drive.


Regard that some system deviced may be instanciated multiple times like \code{RPiDigitalOut}
--- others like \code{I2CBus} may only instanciated once.
 This feature and 
the available devices and the associated system names must be taken from
the corresponding user manual of the \OpenPEARL{} platform 
specific implementation.

In order to use such elements by multiple system names an 
\code{Identifier}$\S $UsernameOfConnectionProvider should be used  as shown with the
declarations of \code{temp} and \code{temp1}.
The system structure is shown graphically in figure \ref{fig:systemStructure}.

\begin{figure}
\setlength{\unitlength}{1 mm}%
\begin{picture}(140,70)
\thicklines
\put(110,5){\colorbox{gray!20}{\framebox(30,60){}}}
\put(115,60){Raspberry Pi}
\put( 90,50){\colorbox{white}{\framebox(30,5){StdIn}}}
\put( 90,43){\colorbox{white}{\framebox(30,5){StdOut}}}
\put( 90,36){\colorbox{white}{\framebox(30,5){Disc}}}
\put( 90,29){\colorbox{white}{\framebox(30,5){RPiDigitalIn}}}
\put( 90,22){\colorbox{white}{\framebox(30,5){RPiDigitalOut}}}
\put( 90, 8){\colorbox{white}{\framebox(30,5){I2CBus}}}

\put(60,50){\framebox(20,5){termin}}
\put(80,52.5){\line(1,0){11}}
\put(60,43){\framebox(20,5){termout}}
\put(80,45.5){\line(1,0){11}}
\put(60,36){\framebox(20,5){folder}}
\put(80,38.5){\line(1,0){11}}
\put(30,36){\framebox(20,5){LogFile}}
\put(50,38.5){\line(1,0){10}}
\put(0,36){\framebox(20,5){Log}}
\put(20,38.5){\line(1,0){10}}
\put(60,29){\framebox(20,5){counter}}
\put(80,31.5){\line(1,0){11}}
\put(60,22){\framebox(20,5){switch}}
\put(80,24.5){\line(1,0){11}}
\put(60,15){\framebox(20,5){motor}}
\put(80,17.5){\line(2,1){11}}
\put(60,8 ){\framebox(20,5){ic2bus}}
\put(80,10.5){\line(1,0){11}}
\put(00,15){\framebox(20,5){temp}}
\put(20,17.5){\line(1,0){10}}
\put(30,15){\framebox(20,5){LM75}}
\put(50,17.5){\line(2,-1){10}}
\put(00, 8){\framebox(20,5){temp1}}
\put(20,10.5){\line(1,0){10}}
\put(30,8){\framebox(20,5){LM75}}
\put(50,10.5){\line(1,0){10}}
\end{picture}
\caption{System structure of example above}
\label{fig:systemStructure}
\end{figure}
%%%\end{added}

\section{Specification and Declaration of Data Stations (DATION) in 
the Problem Part}   % 10.2
\label{sec_dation_problem_part}

\subsection{System Data Stations}   % 10.2.1
\index{DATION@\textbf{DATION}!System Data Station}
System dation are defined in the system part. They are specified 
in the problem part.

Before using system defined data stations, they must be specified in the
problem part, defining their associated user names. 
%%%\acceptedtext{The class attribute {\bf SYSTEM} must be set in the
%%%specification statement}.

The general form reads:

\begin{grammarframe}
\input{DationSpecification.bnf}
\index{SPC@\textbf{SPC}!System Data Station}
\input{TypeDationSystem.bnf}
\end{grammarframe}

The different attributes in a system dation specification describe the
fundamental features of the physical device to be communicated with. The
features of a device and the possible attributes are described for all
devices in the user manual.

For the declarations from the above example, this can take place as:

\begin{lstlisting}
SPC termin  DATION IN    SYSTEM ALPHIC,
    termout DATION OUT   SYSTEM ALPHIC,
    folder  DATION INOUT SYSTEM ALL,   
    counter DATION IN    SYSTEM BIT(1),
    switch  DATION OUT   SYSTEM BIT(1),
    motor   DATION OUT   SYSTEM BIT(4);
\end{lstlisting}
%%%\end{accepted}

System dations may require  \code{Typology} as well as \code{AccessAttribute}.

\subsection{User Defined Data Stations}    % 10.2.2

By a dation declaration, a logical, so-called user defined data station
(or user dation) is created on a physical device (system dation). The
allocation to a device is determined by the 
\code{CREATED} \index{CREATED@\textbf{CREATED}} attribute.
The attributes of the system dation must not conflict with the attributes
of the user defined data station.

All input/output statements described in the following refer to user
dations --- the direct declaration of system dations is forbidden in OpenPEARL.


\begin{grammarframe}
\input{DationDeclaration.bnf}
\index{DCL@\textbf{DCL}!User Data Station}
\end{grammarframe}

Hence, no arrays of data stations may be declared. However, it is
possible to declare arrays of references to data stations to enable also
indexed addressing of data stations.

The various attributes enable the detection of contradictions between
the features of data stations and their way of using in I/O statements
already at compilation time.

\begin{grammarframe}
\input{TypeDation.bnf}
\end{grammarframe}

Each data station is source and/or sink of a data transmission. The
corresponding feature must be given upon declaration:

\begin{grammarframe}
\input{SourceSinkAttribute.bnf}
\end{grammarframe}

\code{IN} \index{IN@\textbf{IN}}
 means that this data station is a source for data, i.e., it may only
appear in such data transmission statements which transmit these data
{\em into} the working memory (e.g., digital inputs, keyboard).

Data stations specified with \code{OUT} \index{OUT@\textbf{OUT}} 
may only be used as sinks for outputs
from the working memory (e.g., printer).

A data station with attribute \code{INOUT} \index{INOUT@\textbf{INOUT}}
 allows for data transmissions in
both directions (e.g., disk).

The data transmissions take place with the computer internal format of
the data or by means of conversion between computer internal and
external format. For this, PEARL provides three different kinds of I/O
statements:
\begin{itemize}
\item The \code{READ}/\code{WRITE} statements for transmission in computer internal
format (e.g., for disk data, see \ref{sec_read_write}).
\item The \code{PUT}/\code{GET} statements for transmission with conversion between
internal format and representation in the character set available on the
data station (e.g., for printer output, see \ref{sec_get_put}).
\item The \code{TAKE}/\code{SEND} statements for transmitting process data (see
\ref{sec_take_send}).
\end{itemize}

The data transmission to or from a data station can only take place in
one of the given ways.

The selection is made when declaring the data station by means of the
class attribute stating to which of the three classes the data belong:
\begin{itemize}
\item If the \code{READ}/\code{WRITE} statements are to be used, i.e., if the data
station records data in computer internal form, the type of data to be
transmitted is stated as class attribute, e.g., \code{FIXED}, or \code{FLOAT}, %%%(53),
or \code{BIT (16)}, or \code{ALL} \index{ALL@\textbf{ALL}} (for various types).
\item If the data are represented alphanumerically on the data stations
(case \code{PUT}/\code{GET}), the data station gets the class attribute 
\code{ALPHIC} \index{ALPHIC@\textbf{ALPHIC}}.
\item Data stations for transmissions with the \code{TAKE}/\code{SEND} statements
have class attribute \code{BASIC} \index{BASIC@\textbf{BASIC}}.
\end{itemize}

The general form of the class attribute reads:

\begin{grammarframe}
\input{ClassAttribute.bnf}
\input{TypeOfTransmissionData.bnf}
\input{SimpleType.bnf}
\input{IOCompoundType.bnf}
\input{IOStructure.bnf}
\input{IOStructureComponent.bnf}
\end{grammarframe}

Hence, the type of the transmission data may also be a multiply
structured structure or a newly defined type, but no component being of
type reference.

Example:

\begin{lstlisting}
...
TYPE KindStructure STRUCT 
      [ (index, number) FIXED,
         weight         FLOAT,
          ...
      ];
DCL  kindFile DATION INOUT KindStructure ...,
     tab      DATION INOUT FIXED ...;
\end{lstlisting}

The attribute \code{ALL} includes all other possibilities of type of
transmission data.
No type checking occurs in this case.

Example:

On a disk memory drive with system name
%%%\removedtext{PSP31}\acceptedtext{Disc}
\code{Disc}
  and user name \code{Disk}, a file
File1 for the input of \code{FIXED} quantities and a file \code{File2} for the
input/output of \code{FLOAT} quantities with computer internal format are to be
created.

\begin{lstlisting}
SYSTEM;
   disk: Disc('/home',10);
PROBLEM;
   SPC disk  DATION INOUT SYSTEM ALL ...;
   DCL file1 DATION IN    FIXED ... CREATED (disk);
   DCL file2 DATION INOUT FLOAT ... CREATED (disk);
\end{lstlisting}

The source/sink attribute and the class attribute must be defined for
each data station, in contrast to the now described attributes for the
structure and access possibilities of a data stations.

The smallest data set transmitted to or from a data station is called
data element. Its type is determined by the class attribute. Several
data elements can be combined into a record (synonymously, line), and
several records to a segment (syonymously, page), i.e., all elements
make up a 1-, 2-, or 3-dimensional array. For this, the number of data
elements in a line, the number of lines in a page, and the number of
pages must be denoted in the structure attribute:

\begin{grammarframe}
\input{Typology.bnf}
\input{TypologyDeclaration.bnf}
\input{TypologySpecification.bnf}
\index{DIM@\textbf{DIM}}
\end{grammarframe}

%%%\begin{added}

The Constant-FIXED-Expressions of each dimension size
 must have a positive value.
%%%\end{added}

%%%pi must be substituded in next paragraph
%%%The pi denotation furthest to the right always denotes the number of
%%%elements per line, the next (possibly lacking) pi denotation denotes
%%%the number of lines per page, and the following (possibly lacking) pi
%%%denotation denotes the number of pages.
The furthest dimension value to the right always denotes the number of
elements per line,
 the next (possibly lacking) value denotes
the number of lines per page, and the next (possiby lacking)
value denotes the number of pages.
%%%\end{discuss}

The denotation \code{*} means that
the corresponding number is not limited. E.g., a data station \code{printer}
with 120 characters per line, 60 lines per page, and any number of pages
can obtain the structure \code{(*, 60, 120)}.

The following combinations are possible:
\begin{itemize}
\item 3-dimensional typology\\
\code{DIM (}number of pages, number of lines, number of elements\code{)}, or\\
\code{DIM (*}, number of lines, number of elements\code{)}
\item 2-dimensional typology\\
\code{DIM (}number of lines, number of elements\code{)}, or\\
\code{DIM (*}, number of elements\code{)}
\item 1-dimensional typology\\
\code{DIM (}number of elements\code{)}, or\\
\code{DIM (*)}
\end{itemize}

\begin{modified}
Furthermore, the typology indicates, how many data elements are
transmitted {\em at least} upon the execution of {\em one} data
transmission statement:
\begin{itemize}
\item If the attribute \code{TFU} is lacking, single data elements,
lines or pages can be transmitted.
\item The \code{TFU} attribute \index{TFU@\textbf{TFU}}
 means that only lines are transferable.
If the actual number of data elements in a data transmission statement
is less than the number of data elements of a line, this line
%%% or page, respectively,
 is implicitly padded with spaces (\code{ALPHIC} data
stations) or zeros (\code{BASIC} data stations).


Example:

\kw{DCL}\code{ printer }\kw{DATION OUT ALPHIC DIM}\code{(*, 60, 120) }\kw{TFU}\code{ ... ;}\\
\code{...} \\
\kw{PUT}{ \code 'PEARL' }\kw{TO}\code{ printer;}

This \code{PUT} statement has the effect that the five characters P, E, A, R,
L are written out in one (new) line by \code{printer}.
\item \code{TFU} may only be used if the number of elements of a record is defined. 
Obviously, it is not allowed for a typology like \code{DIM(*)}.
\item It is possible to use  \code{SKIP} additionally to indicate the end of line.
Multiple \code{SKIP}-statements will produce empty lines on output.
\item On input, a complete line will be processed.  Any data after the last
read input  element will be discarded. A single \code{SKIP} statement at the end 
of the line is ignored in the format list. Multiple \code{SKIP} statements will
discard input lines.
\item \code{TFU} enshures the correct record structure. Obviously
\code{STREAM} is not allowed in combination with \code{TFU}.
%%%\begin{removed}
%%%\item TFU MAX corresponds to TFU.
%%%\end{removed}
\end{itemize}
\end{modified}

\begin{added}
Details about the behavior of \code{TFU}:
\begin{itemize}
\item If the system dation requires transfer units, the sice is specified
   in the system dation definition. The record length of the user dation must not
   exceed this length. 
\item If the system dation does not require \code{TFU}, the behavior is simulated
  by the user dation.
\item If a user dation is declared with \code{TFU}, the record length is defined by
  \begin{itemize}
   \item the position of the end-of-record marker (e.g. NL) for ALPHIC dations
      with access attribute FORWARD
   \item the number of data elements in a line and the size of the data elements for
      all other non basic  dation types. No end-of-record markers are used.
  \end{itemize}
\item On output:
  \begin{itemize}
  \item  If the record length of the system dation is larger than
   the record length of the user dation, the remaining elements
    are padded with zeros or spaces according the dation type.
  \item The \code{TFU} buffer is transfered to the device:
  \begin{itemize}
   \item at the end of the output statement
   \item if a positioning statement leaves the current record
  \end{itemize}
  \end{itemize}
\item On input:
  \begin{itemize}
  \item If the record length of the system dation is larger than
   the record length of the user dation, the remaining elements must be equal
   to the padding element.
   If the input record is shorter, the remaining elements are ignored.
\item A new TFU record is read from the device,
   the first data input format element in the input statement
   after a positioning statement which leaves the current record is treated.
\end{itemize}
\end{itemize}
\end{added}

The possible ways of accessing a data station are determined by the
access attribute:

\begin{grammarframe}
\input{AccessAttribute.bnf}
\end{grammarframe}

\code{DIRECT} \index{DIRECT@\textbf{DIRECT}}
means that (based on a transmitted data element) any data element
can directly be accessed, giving the position of the element
(see \ref{sec_read_write}, \ref{sec_get_put}).

%%% FORBACK eliminated
The attribute \code{FORWARD} \index{FORWARD@\textbf{FORWARD}}
%%% and FORBACK
 mean sequential access; i.e., the
access may (based on a transmitted element) only take place in the order
determined by the structure,
% --- for FORWARD only forward, for FORBACK in both directions ---
possibly giving the relative position of the wanted
element to the element just transmitted 
(see \ref{sec_read_write}, \ref{sec_get_put}).

%\begin{added}
%OpenPEARL does not support \code{FORBACK} dations.
%\end{added}

\code{NOCYCL} \index{NOCYCL@\textbf{NOCYCL}}, \code{CYCLIC}\index{CYCLIC@\textbf{CYCLIC}},
 \code{STREAM}\index{STREAM@\textbf{STREAM}} and 
\code{NOSTREAM}\index{NOSTREAM@\textbf{NOSTREAM}} are treated in the context of the
I/O statements in 
\ref{sec_read_write} and \ref{sec_get_put}.

Example:

On a disk drive with system name 
%%%\removedtext{PSP31}\addedtext{Disc}
\code{Disc}
 and user name \code{disc} a file for a
table with 300 lines and 5 columns (elements per line) is to be created.
Let the table elements be floating point numbers; the access takes place
to elements directly and only reading.

\begin{lstlisting}
MODULE(modulename);

SYSTEM;
disc: Disc('/home', 10);
... 
PROBLEM;
  SPC disc DATION INOUT SYSTEM ALL;
  DCL table DATION IN FLOAT DIM(300, 5) DIRECT GLOBAL CREATED (disc);
 ...
\end{lstlisting}

Examples for the declaration of data stations for inputs/outputs in
character form are described in 
\ref{sec_read_write} and \ref{sec_get_put}.

%%%\begin{accepted}
The characteristics of the system dation must fit to the specification:
\begin{itemize}
\item \code{ALPHIC} specification is allowed upon system dations of type \code{ALPHIC},
   \code{ALL} or \code{CHAR}
\item \code{FORWARD} specification is compatible with \code{DIRECT} attribute
   of the system dation
\item \code{DIM} specifications must not exceed the size of the system dation ---
      more artificial dimensions may be specified
\end{itemize}
%%%\end{accepted}

The import of a user dation differs to the declaration in the following points:
\begin{itemize}
\item instead of a \code{TypologyDeclaration} the \code{TypologySpecification} must be used
\item \code{CREATED} must not be used
\end{itemize}

Example with the declaration of \code{table} above:
\begin{lstlisting}
  ! assume table is defined in module tableDef
  SPC table DATION IN FLOAT DIM(,) DIRECT GLOBAL(tableDef);
\end{lstlisting}

\section{Opening and Closing of Data Stations (OPEN, CLOSE)}   % 10.3
\label{sec_dation_open_close}
\index{OPEN@\textbf{OPEN}}

Before a data station may be used for the first time in a data
transmission statement, it must be opened by the open statement:

\begin{grammarframe}
\input{OpenStatement.bnf}
\index{OPEN@\textbf{OPEN}!\textbf{BY}}
\index{OPEN@\textbf{OPEN}!\textbf{IDF}}
\index{BY@\textbf{BY}!\textbf{OPEN}}
\index{IDF@\textbf{IDF}}
\end{grammarframe}

When executing the open statement, a data station with typology is
positioned at its beginning.
The open parameters serve to handle data stations containing
identifyable files. E.g., a system defined data station \code{Disc} can possess
a file \code{TAB1}, which is also maintained after terminating the program
under this name. Later on, the same or another program can create a user
defined data station \code{table} on \code{disc}, identified with file \code{TAB1} in the
open statement.


\begin{grammarframe}
\input{OpenParameter.bnf}
\end{grammarframe}


%%%\begin{added}
The open parameters may not conflict --- e.g. \code{OLD} and \code{NEW} are not allowed
in the same statement.

%%%\end{added}
%%%\begin{removed}
%%%subsets nicht definiert!

%%%The open parameters of the open statement must belong to different
%%%subsets.
%%%\end{removed}

Meaning of the parameters:
\begin{itemize}
\item \code{IDF} (Name$\S $CharacterVariable $\mid$ CharacterStringConstant)
\index{IDF@\textbf{IDF}}\\
The value of the specified character variable or the specified character
string constant is the name of the file to be identified with the data
station named Name$\S $Dation.
\item \code{RST} (Name$\S $ErrorVariable-FIXED)
\index{RST@\textbf{RST}!\textbf{OPEN}}\\
If an error occurs during the \code{OPEN} execution, the specified variable is
assigned an error number unequal to zero; in the error-free case, it is
set to zero (see \ref{sec_dation_rst}).
\item \code{OLD}\index{OLD@\textbf{OLD}}\\
If there is a file with \code{IDF} name, it its allocated on the notated data
station. Otherwise, or if \code{IDF} is lacking, an error message is given, or
the \code{RST} variable is set with the error number.
\item \code{NEW}\index{NEW@\textbf{NEW}}\\
A file with \code{IDF} name is created and identified with the notated data
station. If there is already a file with this name, or if \code{IDF} is
lacking, an error message is given, or the \code{RST} variable is set with the
error number.
%%%\begin{added}
If the dation has a fixed topology like \code{DIM(10,10)} the content is
filled with spaces or null bytes for \code{ALPHIC} or typed dations, respectively.
%%%\end{added}
\item \code{ANY}\index{ANY@\textbf{ANY}}\\
If there is already a file with \code{IDF} name, it is identified with the
notated data station. Otherwise, a new file is created for it. If \code{IDF}
is lacking, a new file is created under a name determined by the system
and identified with the denoted data station.
\item \code{CAN}\index{CAN@\textbf{CAN}} (from ``cancel'')\\
The file is no longer to be made accessible after executing the close
statement (see below).
\item \code{PRM}\index{PRM@\textbf{PRM}} (from ``permanent'')\\
The file is still there after executing the close statement, and again
accessible with the same name after re-executing an open statement.
%%%\begin{discuss}
%%% removed, since multiple OPENs are forbidden now (22.3.2017)
%%%\item EXCLUSIVE\index{EXCLUSIVE@\textbf{EXCLUSIVE}} \\
%%%The dation (not file) may be opened only once.
%%%Subsequent attempts to open the already
%%%opened  dation produce an error message, or the RST variable is set with the
%%%error number.
%%%\end{discuss}
\end{itemize}

%%%\begin{removed}
%%%If open parameters are lacking, ANY and PRM are assumed.
%%%\end{removed}

%%%\begin{added}
If the open parameter about the previous state of the file (e.g. \code{OLD}, \code{NEW}, \code{ANY})
is missing \code{ANY} is assumed.
If the parameter about the state of the file after closing (e.g. \code{CAN}, \code{PRM})
is missing in \code{OPEN} and \code{CLOSE} statements \code{PRM} is assumed.
%%%\end{added}

When executing the close statement, a file is closed; i.e., it is
usable again not before the execution of an open statement.

\index{CLOSE@\textbf{CLOSE}}
\begin{grammarframe}
\input{CloseStatement.bnf}
\index{CLOSE@\textbf{CLOSE}!CloseStatement}
\index{BY@\textbf{BY}!\textbf{CLOSE}}
\end{grammarframe}

The settings for closing a data station made in the open statement can
be overwritten by a close statement:

%%%\begin{front}
%%%CloseParameter ::=\\
%%%\x {\bf CAN $\mid$ PRM $\mid$ RST} (Name$\S $ErrorVariable-FIXED)
%%%\end{front}
\begin{grammarframe}
\input{CloseParameter.bnf}
\index{RST@\textbf{RST}!\textbf{CLOSE}}
\index{CAN@\textbf{CAN}!\textbf{CLOSE}}
\index{PRM@\textbf{PRM}!\textbf{CLOSE}}
\end{grammarframe}

Generally, the following rules hold:
\begin{itemize}
%%%\begin{added}
\item The close parameters may not conflict --- e.g. ether \code{CAN} or \code{PRM} is allowed
%%%\end{added}
\item Not every task executing an access to a data station must execute
an \code{OPEN} or \code{CLOSE} statement.
%%% removed: OPEN allowed only once
%%%\item However, at least one open statement must be executed concerning
%%%the access to a data station.
%%% removed: OPEN allowed only once
%%%\item The same number of close statements and open statements must
%%%be executed to close the data station.
%%%\begin{added}
%%%\item The close statement which leads to the closing of the data station
%%%defines the action concerning CAN and PRM.
%%%\end{added}
%%% added}
\item The dation must be opened before any access to its data.

\item Corresponding \code{OPEN} and \code{CLOSE} statements need not be executed by
the same task.
\item Upon lacking parameters, \code{PRM} is assumed, unless earlier
executed close or open statements have made explicit settings.
%%%\begin{added}
%%%\item executing an OPEN statement on an already opened dation ignores the
%%%IDF attribute. The attributes CAN or PRM must be identical 
%%%in all concurrent OPEN statements. Otherwise an error message is produced
%%%or the RST variable is set with the error number.

%%%% removed: OPEN allowed only once 
%%%\item
%%%Executing the OPEN statement on an already opened dation need identical 
%%%open parameters IDF, CAN and PRM. Otherwise an error message is produced
%%%or the RST variable is set with the error number.
%%%% added: OPEN allowed only once 
\item
Executing the \code{OPEN} statement on an already opened dation
will cause an error message or the \code{RST} variable is set with the error
number.
\item
Executing the \code{CLOSE} statement on an already closed dation
will cause an error message or the RST variable is set with the error
number.
%%%\end{added}
\end{itemize}

Example:
\label{example}

\begin{lstlisting}
MODULE(modulename);
SYSTEM;
   printer: DRUA;
   disc: Disc('/home' ,10);
 ... 
PROBLEM;
   SPC printer DATION OUT SYSTEM ALPHIC;
   DCL tabProt DATION OUT ALPHIC DIM($^*$, 50, 30) FORWARD GLOBAL
       CREATED(printer);
   SPC disc  DATION INOUT SYSTEM ALL;
   DCL table  DATION IN FLOAT DIM(300, 5) DIRECT GLOBAL CREATED(disc);

start: TASK MAIN;
   OPEN tabProt;
   OPEN table BY IDF('TAB-1'), OLD;
   ACTIVATE prot;
   ...
END; ! start

prot: TASK;
   ! Data transmission statements with table and tabProt
   ...
   CLOSE tabProt;
   CLOSE table; 
END; ! prot 
... 
MODEND;
\end{lstlisting}

\section{The Read and Write Statements (READ, WRITE)}   % 10.4
\label{sec_read_write}
%\index{READ@\textbf{READ}}
%\index{WRITE@\textbf{WRITE}}

The read statement serves for the input, the write statement for the
output of data without converting the computer internal representation
(binary input/output). Data can be transmitted to or from the connected
devices (e.g., a file on disk or magnetic tape). The corresponding data
stations must be declared with the class attribute
``TypeOfTransmissionData''.

Examples:

\begin{enumerate}
\item Columns 4 and 5 in the dation \code{table}  (cf. the example on page
\pageref{example}) are to be replaced by re-calculated values.

\begin{lstlisting}
...
DCL (x, y, z) FLOAT;
...
FOR line FROM 1 TO 300
REPEAT
   ! calculation of x, y, z
   WRITE x, SIN(y+z) TO table BY POS(line, 4);
END;
\end{lstlisting}

\item A task \code{measurement} periodically acquires 14 temperature values
(Procedure \code{getTemp}), processes them, and writes them sequentially in
blocks with 14 values in a logbook on the disk.

\begin{lstlisting}
PROBLEM;
   SPC getTemp PROC(i FIXED) RETURNS} (FIXED(15)) GLOBAL;
   SPC disc    DATION INOUT SYSTEM ALL;
   DCL logbook DATION OUT          FIXED(15) DIM(*, 14)
               TFU FORWARD CREATED (disc);
   DCL numTemp INV FIXED(15) INIT (14);

start: TASK MAIN;
   OPEN logbook;  ! Positioning to beginning
   ALL 10 SEC ACTIVATE measurement;
   ... 
END; ! start

measurement: TASK;
   DCL temperature (numTemp) FIXED(15); 
   FOR i TO numTemp
   REPEAT
      temperature(i) := getTemp(i);
   END;
   ! Processing the measured values
   WRITE temperature TO logbook; 
END; ! measurement
\end{lstlisting}
\end{enumerate}

The general forms of the read and write statements read:

\begin{grammarframe}
\input{ReadStatement.bnf}
\index{READ@\textbf{READ}!ReadStatement}
\index{FROM@\textbf{FROM}!\textbf{READ}}
\index{BY@\textbf{BY}!\textbf{READ}}
\index{RST@\textbf{RST}!\textbf{READ}}
\input{WriteStatement.bnf}
\index{WRITE@\textbf{WRITE}!WriteStatement}
\index{TO@\textbf{TO}!\textbf{WRITE}}
\index{BY@\textbf{BY}!\textbf{WRITE}}
\index{RST@\textbf{RST}!\textbf{WRITE}}
\end{grammarframe}

\OpenPEARL{} regards \hyperlink{StringSelection}{string selections} as
expressions. They are not accepted in the read statement.

A slice may be defined as range of the last index of an array.
\begin{grammarframe}
\input{ArraySlice.bnf}

\input{Position.bnf}
\index{BY@\textbf{BY}!Position}

\input{AbsolutePosition.bnf}
\index{COL@\textbf{COL}}
\index{LINE@\textbf{LINE}}
\index{POS@\textbf{POS}}
\index{SOP@\textbf{SOP}}

\input{RelativePosition.bnf}
\index{X@\textbf{X}}
\index{SKIP@\textbf{SKIP}}
\index{PAGE@\textbf{PAGE}}
\index{ADV@\textbf{ADV}}
\index{EOF@\textbf{EOF}}

\end{grammarframe}


Upon entry with the read statement, the addressed data elements are
read one after the other and correspondingly assigned to the variables
in the variable list. The variables are assigned according to the
general rules for assignments. If an element of the variable list is an
array, the addressed data are assigned by rows; if it is a structure, the
data are assigned to the structure components in the order determined by
the structure declaration.

For simplicity in writing, the elements (of the last dimension) of an
array following one another in the variable list can be declared in form
of a segment. Let list be an array with ten elements \code{list(1), ... ,
list(10);} then the two following statements are equivalent:

\kw{READ}\code{ list(2), list(3), list(4) ... ;}\\
\kw{READ}\code{ list (2 : 4), ... ;}

All position expressions are completely evaluated, before values are
assigned to the data elements.

Example:

Read \code{x} from position 3 and \code{y} from position 5 from the dation file:

\kw{READ}\code{ x }\kw{FROM}\code{ file }\kw{BY POS}\code{(3) ;}\\
\kw{READ}\code{ y }\kw{FROM}\code{ file }\kw{BY POS}\code{(5) ;}\\

The execution of the statement

\kw{READ}\code{ x,y }\kw{FROM}\code{ file }\kw{BY POS}\code{(3), }\kw{POS}\code{(5) ;}

however, results in that \code{x} being read from position 5 and \code{y} from the
subsequent position.

The \code{RST} attribute (cf. \ref{sec_dation_rst}) can occur anywhere in the
 position list.
However, it does not become effective until it is evaluated. A position
list is elaborated one after the other, starting from the left. If an
error occurs, the elaboration of the I/O statement is aborted in this
position, and the error reaction valid at this moment (error assignment
to an \code{RST} variable or system reaction) is executed.

These statements analogously hold for the write statement.

The type of the variables in the variable list of the read statement
must be compatible with the class attribute of the given data station;
this holds analogously for the results of the expressions in the
expression list of the write statement.

The variable or expression list of the read or write statement may be
lacking, if these statements shall only be used for positioning in the
denoted data stations. In this case, however, a position must be
given.

The values in the position list refer to the structure of the data
station and determine the data elements to be transmitted. Thus, the
values of the expressions must be of type \code{FIXED} and must be compatible
with the structure.

When using an absolute position, i.e., a position independent from the
actual data element, the data station must have the access attribute
\code{DIRECT}. A relative position denotes the distance of the data element to
be transmitted from the current data element; in this case, the data
station must have the access attribute \code{FORWARD}
%%%\removedtext{, FORBACK}
 or \code{DIRECT}.

In detail, the possible position attributes have the following
meanings:
\begin{itemize}
\item \code{COL}\index{COL@\textbf{COL}} (Expression)\\
refers to the first dimension (from the right) of the structure and
determines the i-th element in the current line of the data station, if
i equals the value of the expression.
\item \code{LINE} (Expression)\\
refers to the second dimension of the structure and determines the i-th
line of the current page of the data station, if i equals the value of
the expression.

Example:

A 2-dimensional array corresponds to the structure (5,10):

\setlength{\unitlength}{0.00087500in}%
\begin{picture}(5627,2781)(179,-2206)
\thicklines
\put(451,-1861){\framebox(4500,2250){}}
\thinlines
%\multiput(1891,-61)(-5.62500,-5.62500){17}{\makebox(6.3500,9.5250){\SetFigFont{7}{8.4}{rm}.}}
\put(1891,-61){\line(-1,-1){90}}
\put(1981,-61){\line(-1,-1){180}}
\put(2071,-61){\line(-1,-1){270}}
\put(2161,-61){\line(-1,-1){360}}
\put(2251,-61){\line(-1,-1){450}}
\put(2251,-151){\line(-1,-1){360}}
\put(2251,-241){\line(-1,-1){270}}
\put(2251,-331){\line(-1,-1){180}}
%\multiput(2251,-421)(-5.62500,-5.62500){17}{\makebox(6.3500,9.5250){\SetFigFont{7}{8.4}{rm}.}}
\put(2251,-421){\line(-1,-1){90}}
\put(1801,-61){\line( 1,-1){450}}
\put(1891,-61){\line( 1,-1){360}}
\put(1981,-61){\line( 1,-1){270}}
\put(2071,-61){\line( 1,-1){180}}
%\multiput(2161,-61)(5.62500,-5.62500){17}{\makebox(6.3500,9.5250){\SetFigFont{7}{8.4}{rm}.}}
\put(2161,-61){\line( 1,-1){90}}
\put(1801,-421){\line( 1,-1){90}}
\put(1801,-151){\line( 1,-1){360}}
\put(1801,-241){\line( 1,-1){270}}
\put(1801,-331){\line( 1,-1){180}}
%\multiput(1801,-421)(5.62500,-5.62500){17}{\makebox(6.3500,9.5250){\SetFigFont{7}{8.4}{rm}.}}
\put(1801,389){\line( 0,-1){2250}}
\put(2251,389){\line( 0,-1){2250}}
\put(451,-61){\line( 1, 0){4500}}
\put(451,-511){\line( 1, 0){4500}}
\put(676,479){\makebox(0,0)[b]{\smash{\SetFigFont{12}{14.4}{rm}1}}}
\put(1126,479){\makebox(0,0)[b]{\smash{\SetFigFont{12}{14.4}{rm}2}}}
\put(1576,479){\makebox(0,0)[b]{\smash{\SetFigFont{12}{14.4}{rm}3}}}
\put(2026,479){\makebox(0,0)[b]{\smash{\SetFigFont{12}{14.4}{rm}4}}}
\put(2476,479){\makebox(0,0)[b]{\smash{\SetFigFont{12}{14.4}{rm}5}}}
\put(2926,479){\makebox(0,0)[b]{\smash{\SetFigFont{12}{14.4}{rm}6}}}
\put(3376,479){\makebox(0,0)[b]{\smash{\SetFigFont{12}{14.4}{rm}7}}}
\put(3826,479){\makebox(0,0)[b]{\smash{\SetFigFont{12}{14.4}{rm}8}}}
\put(4276,479){\makebox(0,0)[b]{\smash{\SetFigFont{12}{14.4}{rm}9}}}
\put(4726,479){\makebox(0,0)[b]{\smash{\SetFigFont{12}{14.4}{rm}10}}}
\put(226,-1726){\makebox(0,0)[b]{\smash{\SetFigFont{12}{14.4}{rm}5}}}
\put(2026,-2176){\makebox(0,0)[b]{\smash{\SetFigFont{12}{14.4}{rm}COL(4)}}}
\put(5491,-376){\makebox(0,0)[b]{\smash{\SetFigFont{12}{14.4}{rm}LINE(2)}}}
\put(226,-331){\makebox(0,0)[b]{\smash{\SetFigFont{12}{14.4}{rm}2}}}
\put(226,119){\makebox(0,0)[b]{\smash{\SetFigFont{12}{14.4}{rm}1}}}
\put(226,-781){\makebox(0,0)[b]{\smash{\SetFigFont{12}{14.4}{rm}3}}}
\put(226,-1231){\makebox(0,0)[b]{\smash{\SetFigFont{12}{14.4}{rm}4}}}
\end{picture}

\item \code{POS} ( [ [ Expression$\S $Page , ] Expression$\S $Row , ] Expression$\S $Column )\\
gives the position of a data element in the n-dimensional structure
(n = 1, 2, 3) of a data station. Lacking expressions are replaced by the
respective current value.

Example:

With the statement

\kw{READ}\code{ x }\kw{FROM}\code{ file1 }\kw{BY POS}\code{(3,2,8);}

the eighth data element of the second row of the third page of \code{file1} is
read into \code{x}. If this is followed by the statement

\kw{READ}\code{ x }\kw{FROM}\code{ file1 }\kw{BY POS}\code{(4,5);}

the fifth data element of the fourth row of the third page of \code{file1} is
read into \code{x}.
\item \code{SOP} ( [ [ Name$\S $Page , ] Name$\S $Row , ] Name$\S $Column )\\
is the counterpart to \code{POS}. With \code{SOP}, the current positions of a dation
can be assigned to the given program variables. The number of names in
\code{SOP} may not exceed the number of dimensions in the addressed dation.
\end{itemize}

In the following, let i be the value of the respective expression.
\begin{itemize}
\item \code{X} [ (Expression) ]\\
refers to the first dimension of the structure and determines the i-th
element behind (i positive) or before (i negative), respectively, the
current element in the current row of the data station; i = 0 denotes
the current element.
\item \code{SKIP} [ (Expression) ]\\
refers to the second dimension of the structure and determines the start
(the first element) of the i-th row behind (i positive) or before (i
negative), respectively, the current row of the current page of the data
station; i = 0 denotes the current row.
\item \code{PAGE} [ (Expression) ]\\
refers to the third dimension of the structure and determines the start
of the i-th page behind (i positive) or before (i negative),
respectively, the current page of the data station; i = 0 denotes the
current page.
\end{itemize}

If the Expression denotation is not present for \code{X}, \code{SKIP}, or \code{PAGE}, respectively,
value 1 is assumed. The value of a given expression must be positive, if
the data station has the attribute \code{FORWARD}.
\begin{itemize}
\item \code{ADV} ( [ [ Expression$\S $Page , ] Expression$\S $Row , ] Expression$\S $Column )\\
gives the distance of the data element to be transmitted from the
current data element. Missing expressions are replaced by the value
zero. If the data station has the attribute \code{FORWARD}, the value for the
expression furthest to the left must be positive or zero.
%%%\begin{discuss}
\item \code{EOF}\\
refers to the end of the dation. If used in conjunction with fixed typology
and \code{CYCLIC} attribute the
compiler will emit a warning that the effective position will be the
beginning of the dation.
%%%\end{discuss}
\end{itemize}

Examples:

Let the corresponding data station \code{file1} have the structure (10, 10,
10), let the actual position be (5,3,8).

\begin{tabular}{l@{\x}l@{\x}l}
position statement   & new position &     \\ \hline
\code{X}                    & (5,3,9)      &     \\
\code{X (-5)}               & (5,3,3)      &     \\
\code{X (4)}                & (5,4,2)      & (1) \\
\code{SKIP (2)}             & (5,5,1)      &     \\
\code{SKIP (-1)}            & (5,2,1)      &     \\
\code{PAGE}                 & (6,1,1)      &     \\
\code{PAGE (6)}             & (1,1,1)      & (2) \\
\code{PAGE (-4)}            & (1,1,1)      &     \\
\code{ADV (2,5,1)}          & (7,8,9)      &     \\
\code{ADV (1,0)}            & (5,4,8)      &     \\
\code{ADV (-3,-2,1)}        & (2,1,9)      &     \\
\code{ADV (1,8,0)}          & (7,1,8)      & (3)
\end{tabular}

With a relative positioning, dimension boundaries may not be exceeded
(cf. the marked examples), unless the data station possesses the access
attribute \code{STREAM} or \code{CYCLIC}, respectively. \code{STREAM} allows to exceed the
internal dimension boundary (examples (1) and (3)), but not to exceed
the boundary of the highest dimension (example (2)). For this, a data
station must possess the attribute \code{CYCLIC}. If error reactions shall
occur upon exceeding the corresponding boundaries, the attributes
\code{NOSTREAM} or \code{NOCYCL}, respectively, must be given. By default, \code{STREAM} and
\code{NOCYCL} are assumed.

Example:

\kw{DCL}\code{ file1 }\kw{DATION INOUT FIXED DIM}\code{(10,10,10) }\kw{FORWARD CYCLIC}\\
\x \kw{GLOBAL CREATED}\code{(disc);}

Let each time the current position be (5,3,8). Then it holds:

\begin{tabular}{l@{\x}l}
position statement   & new position \\ \hline
\code{X (6)}                & (5,4,4)      \\
\code{SKIP (8)}             & (6,1,1)      \\
\code{PAGE (7)}             & (2,1,1)      \\
\code{ADV (9,0,3)}          & (4,4,1)
\end{tabular}

%%%\begin{accepted}
The positioning controls work diffently on \code{FORWARD} and \code{DIRECT}
dations. \code{FORWARD} dations does not allow to adjust the real position for read
and write directly. 
Thus the positioning discards input data in the \code{READ} statement and 
writes as may zero bytes as required to fill the amount of data elements
in the \code{WRITE} statements.
%%%\end{accepted}

%%%\begin{accepted}
For dations with the ClassAttribute \code{ALL}
%%%only one unlimited dimension DIM(*) is supported in OpenPEARL.
the positioning operates on byte level.
%%%Absolute positioning is allowed only to POS(1) for DIRECT dations.
%%%No relative positioning is allowed.
%%%\end{accepted}

\section{The Get and Put Statements (GET, PUT)}    % 10.5
\label{sec_get_put}
%\index{PUT@\textbf{PUT}}
%\index{GET@\textbf{GET}}

The get statement serves for the input, the put statement for the
output of data with conversion between computer internal and external,
character oriented representations on \code{ALPHIC} data stations. To control
this conversion, formats can be declared.

Examples:

\begin{enumerate}
\item The following text is to appear on the monitor of a storekeeper:

\begin{verbatim}
___article no:__4711
___stock:_______1281
\end{verbatim}

Program steps required:

\begin{lstlisting}
...
DCL stockMonitor DATION INOUT ALPHIC DIM(*,20,80)
                 FORWARD GLOBAL CREATED (monitor);
DCL (artno, stock) FIXED;
...
PUT 'article no:', artno, 'stock:', stock TO stockMonitor
    BY X(3), A(13), F(4), SKIP, X(3), A(13), F(4);
...
\end{lstlisting}

\item Output of two values in standard format on a new page of the
printer.

\begin{lstlisting}
...
DCL printer DATION OUT ALPHIC DIM(*,50,120) FORWARD
  GLOBAL CREATED (thermoPrint);
DCL a FIXED(15), x FLOAT(23);
...
a := 5;
x := 2.33;
...
PUT TO printer BY PAGE;
PUT a, x TO printer BY LIST;
\end{lstlisting}

The execution results in the following printer layout:

\begin{verbatim}
_____5___2.33000E+00
\end{verbatim}
\item Let data be stored in an input file on floppy disk in the
following form:

\begin{tabular}{l@{:}l}
column  1 -- 10 & article identifier (\code{CHARACTER})\\
column 12 -- 20 & quantity (\code{FIXED})\\
column 22 -- 30 & price per unit right-justified (e.g., \_\_\_124.57)
\end{tabular}

They are to be read into the variables \code{articleId}, \code{quantity}, \code{price}:

\begin{lstlisting}
...
DCL inputFile DATION IN ALPHIC DIM(*,80) TFU FORWARD
              GLOBAL CREATED (floppyDisk),
    articleId CHAR(10),
    quantity  FIXED,
    price     FLOAT;
...
GET articleId, quantity, price FROM inputFile BY
    A(10), X,   F(9), X,  E(9), SKIP;
\end{lstlisting}
\end{enumerate}

The general forms of the get and put statements read:

\begin{grammarframe}
\input{GetStatement.bnf}
\index{GET@\textbf{GET}!GetStatement}
\index{FROM@\textbf{FROM}!\textbf{GET}}
\index{BY@\textbf{BY}!\textbf{GET}}
\index{RST@\textbf{RST}!\textbf{GET}}
\input{PutStatement.bnf}
\index{PUT@\textbf{PUT}!PutStatement}
\index{TO@\textbf{TO}!\textbf{PUT}}
\index{BY@\textbf{BY}!\textbf{PUT}}
\index{RST@\textbf{RST}!\textbf{PUT}}
\input{FormatOrPosition.bnf}
\input{FormatFactor.bnf}
\input{Format.bnf}
\end{grammarframe}

\OpenPEARL{} regards \hyperlink{StringSelection}{string selections} as
expressions. They are not accepted in the get statement.

Upon input with the get statement, the addressed data elements are
read one after the other and correspondingly assigned to the variables
in the variable list (analogously to the read statement). The
assignment to the variables takes place according to the general
assignment rules.

The input is terminated, when the the variable list is worked off. If
there still are list elements, but no data elements, an error message is
displayed.

When executing the put statement, the values of the expressions
following \code{PUT} in the list are written out in the given order.

In the order of notation, a format in the format position list is
associated with each variable in the variable list of the get
statement, describing the external representation of the data on the
named data station and used for converting into computer internal
representation. 
%%%\begin{removed}
%%%The kind of format is determined by the type of the
%%%variable.
%%%\end{removed}
%%%\begin{added}
If no formats are specified, the kind of format is determined by the
type of the variable (see \code{LIST} \ref{sec_dation_list_format}).
%%%\end{added}
Besides formats, the list of the format positions can also
contain position statements (cf. \ref{sec_read_write})
for positioning in the data
station. If the variable list is not exhausted, yet, the further
positionings are executed and the just mentioned association is
continued with the next following format. If, on the contrary, the
variable list is already exhausted, the following position attributes
are elaborated, until a format is encountered or the list is worked off.

If the variable list contains an array or a segment, the following
formats are associated with the array elements one after the other.

The number of the transmitted data elements is only determined by the
variable list, not by the format position list. If there are more
formats than variables, the surplus formats are neglected. If there are
still variables left, when the format position list is worked off, it is
re-started with the first element of the format position list. In each
case, the transmission is terminated, when the variable list is
exhausted.

The above analogously holds for the \code{PUT} statement with ``variable''
replaced by ``expression''.

The data station must possess the class attribute \code{ALPHIC}, a structure,
and an access attribute. The selection of the access attribute possibly
limits the positioning possibilities (cf. \ref{sec_read_write}).

The format position list consists of format and position statements.
To simplify notation, repetition factors may be used in the list. For
instance, the format position list

\code{X(2), F(12,3), X(2), F(12,3), X(2), F(12,3)}

can be written in an easier way:

\code{(3) (X(2), F(12,3))}

The following table shows the permitted associations between formats and
types of the data elements to be transferred:

\begin{tabular}{ll}
format                  & data type \\ \hline
fixed format            & \code{FIXED}, \code{FLOAT} \\
float format            & \code{FIXED}, \code{FLOAT} \\
bit format              & \code{BIT}          \\
character string format & \code{CHARACTER}    \\
time format             & \code{CLOCK}        \\
duration format         & \code{DURATION}     \\
list format             & all given data types
\end{tabular}

In detail, the formats have the following forms and meanings (the
position statements were explained in \ref{sec_read_write}).

\begin{modified}
If no FormatOrPosition elements are specified the default format is \code{LIST}
 for each data element.
\end{modified}

%%%\begin{accepted}
% already stated at read/write
The positioning controls work differently on \code{FORWARD} and \code{DIRECT}
dations. \code{FORWARD} dations does not allow to adjust the real position for read
and write directly. 
Thus the positioning discards input data in the \code{GET} statement and 
writes as may space characters as required to fill the amount of data elements
in the \code{PUT} statements.
%%%\end{accepted}


\subsection{The Fixed Format (F)}   % 10.5.1
\label{sec_dation_f_format}
\index{F@\textbf{F}}

\begin{grammarframe}
\input{FixedFormat.bnf}

\input{FieldWidth.bnf}
\index{FORMAT@\textbf{FORMAT}!\textbf{F}}
\index{FieldWidth!F-Format}
\index{F@\textbf{F}!FieldWidth}

\input{DecimalPositions.bnf}
\index{DecimalPositions!F-Format}
\index{F@\textbf{F}!DecimalPositions}
\end{grammarframe}

The fixed format describes the external representation of decimal fixed
point numbers. The field width w is the complete number of characters
available for the decimal number; decimal positions d denote the number
of digits behind the decimal point. 
%%%\begin{accepted}
The behavior is described separatelly for input and output.
%%%\end{accepted}

%%%\begin{accepted}
\begin{enumerate}
\item {\bf Output}
\begin{enumerate}
\item The decimal number is stored right-justified in a field of length
w in the form

[ - ] pi [ . pi ]

where pi means positive integer. If the number does not occupy the
entire field, the left part is padded with spaces.
\item If 0 $>$ d or w $\leq$ d, the \code{FixedFormatSignal} is induced
\item In case w $\leq$ 0, the \code{FixedFormatSignal} is induced
\item If d = 0 or is not given, only the integer part of the decimal
number rounded without decimal point is written out.
\item Except for the zero immediately in front of the decimal point,
leading zeros are suppressed.
\item if the output value does not fit into the given field width 
   (also regarding
% p and
 negative sign), the \code{FixedValueSignal} is induced.
\item Example for output:

\begin{tabular}{lll}
value  & format   & output \\ \hline
13.5   & \code{F(7,-1)}   &  \x \x {\em FixedFormatSignal!}  \\
13.5   & \code{F(0)}   &  \x \x {\em FixedFormatSignal!}  \\
13.5   & \code{F(7,2)}   & \_\_13.50  \\
-13.5   & \code{F(7,2)}   & \_-13.50  \\
275.2  & \code{F(4,1)}   &  \x \x {\em FixedValueSignal!} \\
22.8   & \code{F(5)}     & \_\_\_23 \\
212.73 & \code{F(9,2)}   & \_\_\_212.73 \\
\end{tabular}
%%%\end{accepted}
\end{enumerate}
\item {\bf Input}
\begin{enumerate}
\item A field of length w is read, containing a decimal fixed point
number in the following representation:

[ [ + $\mid$ - ] pi [ . [ pi ] ] ]

\item Spaces preceding or following the number are ignored.
%%%%\item If the entire field is empty, value 0 is read in.
\item If the entire field is empty, the \code{FixedValueSignal} is induced.
\item If no decimal point occurs in the representation, the last d
digits are interpreted as positions following a decimal point.
\item If a decimal point occurs in front of the last b digits in the
representation, then it has priority over the specification by d. In
this case, the statement of d has no meaning. Nevertheless the value must
be reasonable in respect to the other rules. 
\item If w $\leq$ 0 or d $<$ 0 or w $\leq$ d, no assignment takes place;
    the \code{FixedFormatSignal} is induced
\item If the data in the input field does not match a valid fixed 
   number, the \code{FixedValueSignal} is induced


%%%\begin{accepted}
\item Example for input:

\begin{tabular}{llll}
external       &          & \multicolumn{2}{c}{input as}  \\ 
value       &          & \code{FIXED} & \code{FLOAT}  \\ \hline
13.5   & \code{F(7,1)}   &  14 & 13.5   \\
13.5   & \code{F(5)}     &  14 & 13.5   \\
275.2  & \code{F(4,1)}   & 275 & 275.2 \\
\_\_\_212.73 &  \code{F(9,2)}&  213 & 213,73 \\
\_\_\_\_\_ & \code{F(5)} & 0& 0.0 \\
\_3A\_\_\_ & \code{F(5)} & \multicolumn{2}{c}{\em FixedValueSignal!} \\
\end{tabular}
\end{enumerate}
\end{enumerate}

%%%\end{accepted}

\subsection{The Float Format (E)}   % 10.5.2
\label{sec_dation_e_format}
\index{E@\textbf{E}}
\index{E3@\textbf{E3}|see{E}}
\index{FORMAT@\textbf{FORMAT}!\textbf{E}}
\index{FORMAT@\textbf{FORMAT}!\textbf{E3}}

\begin{grammarframe}
\input{FloatFormat.bnf}
\index{FieldWidth!E-Format}
\index{E@\textbf{E}!FieldWidth}
\index{DecimalPositions!E-Format}
\index{E@\textbf{E}!DecimalPositions}
\input{Significance.bnf}
\index{Significance!E-Format}
\index{E@\textbf{E}!Significance}
\end{grammarframe}

The float format describes the external representation of decimal
floating point numbers of the form

[ + $\mid$ - ] FloatingPointNumber

where the exponent consists of two 
%%%\acceptedtext{
(E) or three (E3) %%%}
 digits.
 Field width and decimal
positions have the same meaning as in the fixed format; significance s
denotes the number of significant digits; i.e., the length of the
mantissa.

E(w,d) is of equivalent to E(w,d,d+1), and E(w) to E(w,0).

%%%\begin{accepted}
\begin{enumerate}
\item {\bf Output}\\
\begin{enumerate}
\item If w $<$ d or w $<=$ 0, the \code{ExpFormatSignal} is induced. 

\item If w is too small to mention a digit of the mantissa,
    the \code{ExpValueSignal} is induced

\item The floating point number is stored right-justified in a
 field of length w.
%%% Otherwise, 1(a) in \ref{sec_dation_f_format} holds.

\item In case w $>$ d $>$ s, the mantissa is chosen so that it holds:

\hspace*{1cm}$10^{s-d-1} \leq \mid$ mantissa $\mid < 10^{s-d}$

\item If d $>$ 0, the number has the form

\hspace*{1cm}[ - ] s-d digits . d digits E \{ + $\mid$ - \} exponent.

The exponent is determined in such a way that the leading digit of the
mantissa does not equal zero, if the number is different from zero.

\item If d = 0, the number has the form

\hspace*{1cm}[ - ] s digits E \{ + $\mid$ - \} exponent

\item the exponent field used two digits in E-format and 
      three digit in E3-format

\item If s $<$ d +1 the \code{ExpFormatSignal} is induced.
\item Example for output:

\begin{tabular}{llll}
value  & format    & output \\ \hline
-0.07  & \code{E(9,1)}    & \_-7.0E-02 \\
2713.5 & \code{E(11,2,4)} & \_\_27.14E+02 & rounding occurs\\
2721   & \code{E(8)}      & \_\_\_3E+03 & rounding occurs\\
2721   & \code{E3(8)}      & \_\_3E+003 & rounding occurs
\end{tabular}

\end{enumerate}
 
\item {\bf Input}\\
\begin{enumerate}
\item If w $<=$ 0, or d $<$ 0, or s $<$ 
%%%\removedtext{0}\addedtext{1}
d+1 , no assignment happens and  
         the \code{ExpFormatSignal} is induced
\item If the field  is empty, the \code{ExpValueSignal} is induced.
\item If the data in the field does not match a valid number, 
 no data is assigned and the \code{ExpValueSignal} is induced.
\item A field of length w is read,
    containing a decimal floating point number
    in one of the possible representations (cf. 5.3).
    Otherwise the \code{ExpValueSignal} is induced.
%\item The statements 2(b) to 2(f) in \ref{sec_dation_f_format}
%     hold analogously.
\end{enumerate}
\end{enumerate}

%%%\end{accepted}

%\subsection{The Character String Formats (A) and (S)}  % 10.5.3
\subsection{The Character String Formats (A)}  % 10.5.3
\label{sec_dation_a_format}

%%%\begin{front}
%%%CharacterStringFormat ::=\\
%%%\x {\bf A} [ (Expression$\S $NumberCharacters) ]
%%%  \removedtext{ $\mid$ {\bf S} (Name$\S $NumberCharactersVariableFixed) }
%%%\end{front}
\begin{grammarframe}
\input{CharacterStringFormat.bnf}
\index{A@\textbf{A}}
\index{FORMAT@\textbf{FORMAT}!\textbf{A}}
\end{grammarframe}


The character formats describe the external representation of character
strings (character quantities) of the form

Character$^{...}$

%\subsubsection{Character String Format (A)}

The value of the expression in character string format means the total
number w of the character positions available for the representation.
\begin{enumerate}
\item {\bf Output}\\
\begin{enumerate}
\item If the format has form ``\code{A} (Expression)'', the character string is
written out left-justified in the above presented form in a field of
length w.
\item  If the character string consists of more than w characters, it
is truncated on the right; if it consists of less than w characters, the
field is padded with spaces on the right. 
%%%\begin{removed}
%%%If w = 0, no characters are
%%%written out, and the expression in the expression list of the put
%%%statement is skipped.
%%%\end{removed}
%%%\begin{accepted}
\item If w $<=$ 0, the \code{CharacterFormatSignal} is induced
%%%\end{accepted}
\item If the expression is not given in the format, i.e., the format has the
form ``\code{A}'', the string is written out in a field, whose length equals
the string length.
\end{enumerate}
\item {\bf Input}\\
\begin{enumerate}
\item Characters up to the maximum of w or until encountering the next record
delimiter (e.g., CR) are read in. A record delimiter is not transferred
into the character string variable.
\item If w is smaller than length lg of the associated character string
variable, the right part is padded with spaces; in case w $>$ lg, the
right part is truncated.
%%%\begin{removed}
%%% If w = 0, a string of lg spaces is assigned to
%%%the variable.
%%%\end{removed}
%%%\begin{accepted}
\item  If w $<=$ 0, no assignment happens and the \code{CharacterFormatSignal} is 
   induced. 
%%%\end{accepted}
\end{enumerate}
\end{enumerate}

Examples:

The output of the character string 'PEARL' in format

\begin{itemize}
\item \code{A}    results in \code{PEARL} 
\item \code{A(5)} results in \code{PEARL} 
\item \code{A(7)} results in \code{PEARL\_\_}
\item \code{A(2)} results in \code{PE}
\end{itemize}

The input of the character string '\code{PEARL\_\_}' to a 
\code{CHAR(5)} variable text in format

\begin{itemize}
\item \code{A} is equivalent to \code{text := 'PEARL';}
\item \code{A(5)} is equivalent to \code{text := 'PEARL';}
\item \code{A(7)} is equivalent to \code{text := 'PEARL';}
\item \code{A(2)} is equivalent to \code{text := 'PE\_\_\_';}
\end{itemize}

%%%\begin{removed}
%%%\subsubsection{Character String Format (S)}
%%%
%%%The variable in character string format must be of type FIXED.
%%%\begin{enumerate}
%%%\item {\bf Output}\\
%%%Identical with A format: the value of the given variable determines the
%%%width of the output field.
%%%\item{\bf Input}\\
%%%Characters up to the maximum lg (length of the associated character
%%%string variables) or until encountering the next record delimiter are
%%%read in; otherwise, the same rules like for the A format hold.
%%%Additionally, the number of characters (without record delimiter) read
%%%is assigned to the variable of the S format. Which record delimiters are
%%%defined for which devices is to be found in the respective PEARL user
%%%manual.
%%%\end{enumerate}
%%%
%%%Example:
%%%
%%%Command lines are to be read in from the terminal.
%%%
%%%{\bf DCL} buffer {\bf CHAR}(80);\\
%%%{\bf DCL} length {\bf FIXED};\\
%%%...\\
%%%{\bf GET} buffer {\bf FROM} terminal {\bf BY S}(length);\\
%%%...
%%%
%%%After the entry ``abc$<$RETURN$>$'' on the keyboard (the key
%%%$<$RETURN$>$ creates a record delimiter), the variable ``buffer''
%%%contains the characters ``abc'' and the variable ``length'' the value 3.
%%%\end{removed}


\subsection{The Bit Format (B)}    % 10.5.4
\label{sec_dation_b_format}

\begin{grammarframe}
\input{BitFormat.bnf}
\index{FORMAT@\textbf{FORMAT}!\textbf{B},B1,B2,B3,B4}
\index{B@\textbf{B}}
\index{B1@\textbf{B1}|see {B}}
\index{B2@\textbf{B2}|see {B}}
\index{B3@\textbf{B3}|see {B}}
\index{B4@\textbf{B4}|see {B}}
\end{grammarframe}

The bit format describes the external representation of bit strings (bit
quantities), namely (cf. \ref{sec_bit_strings})
\begin{itemize}
\item in binary form by the format: \{ \code{B} $\mid$ \code{B1} \} 
		[ \code{(} Expression \code{)} ] ,
\item in form of tetrades by the format: \code{B2} [ \code{(} Expression \code{)} ] ,
\item in form of octades by the format: \code{B3} [ \code{(} Expression \code{)} ] ,
\item in hexadecimal form by the format: \code{B4} [ \code{(} Expression \code{}) ] .
\end{itemize}

The value of the expression in bit format means the total number w of
the character positions available for the representation.
\begin{enumerate}
\item {\bf Output}\\
\begin{enumerate}
\item If the expression is declared in the format, the bit string is written
out left-justified in the above represented form in a field of length
w. 
\item If the bit string consists of more than w characters, it is truncated
on the right; if it consists of less than w characters, the field is
padded with zeros on the right. 
\item If w is not given, and if the bit format
thus has the form \code{B} $\mid$ \code{B1} $\mid$ \code{B2} $\mid$
	 \code{B3} $\mid$ \code{B4}, the string
is written out in a field, whose length equals the length of the
string.
%%%\begin{accepted}
\item If w $<=$ 0, no data is written and the \code{BitFormatSignal} is induced.
%%%\end{accepted}
\end{enumerate}
\item {\bf Input}\\
\begin{enumerate}
\item The expression must be given.
\item A field of length w is read in, which must contain a bit string of the
above described form.
If the data field does not match a valid bit representation,
    or represents more bits than the input element, 
   or the field consists completely of spaces,
the \code{BitValueSignal} is induced.
\item Spaces preceding or following the string are ignored.

%%%\begin{accepted}
\item If w $<=$ 0, no data is assigned and the \code{BitFormatSignal} is induced.
%%%\end{accepted}
\item The B2/3/4 formats may inherently specify more bits as the input element contains. 
   These bits must be cleared. The \code{BitValueSignal} is induced else. 
\end{enumerate}
\end{enumerate}

Example:

The output of the bit string '\code{0101110}' in the format

\begin{itemize}
\item \code{B(5)} results in \code{01011} 
\item \code{B2(3)} results in \code{113} 
\item \code{B3(3)} results in \code{270} 
\item \code{B4(2)} results in \code{5C}
\end{itemize}

Let the variable bit string be of type \code{BIT(8)}; the input, there are the
following possibilities:

\begin{tabular}{lll}
data element to be entered & format & value of bit string \\ \hline
11111                      & \code{B(5)}   &         11111000    \\
201                        & \code{B2(3)}  &         10000100    \\
235                        & \code{B3(3)}  &   \textit{BitValueSignal} \\
AB                         & \code{B4(2)}  &         10101011
\end{tabular}

\subsection{The Time Format (T)}   % 10.5.5
\label{sec_dation_t_format}

\begin{grammarframe}
\input{TimeFormat.bnf}
\index{T@\textbf{T}}
\index{FieldWidth!T-Format}
\index{T@\textbf{T}!FieldWidth}
\index{FORMAT@\textbf{FORMAT}!\textbf{T}}

\index{DecimalPositions!T-Format}
\index{T@\textbf{T}!DecimalPositions}
\end{grammarframe}

%%%\begin{removed}

The time format describes the external representation of time data. The
field width means the total number w of the character positions
available for the representation; decimal positions stands for the
number d of the digits for the fractional parts of seconds of the clock
time. 
%%%\addedtext{
The number of decimal positions defaults to 0.
%%%}
\begin{enumerate}
\item {\bf Output}\\
\begin{enumerate}

%%%\begin{accepted}
\item If w $<=$ 0 or d $<$ 0, no data is written an the \code{ClockFormatSignal}
   is induced.

\item If w is too small for the given clock value, the \code{ClockValueSignal}
   is induced.
%%%\end{accepted}

\item The time is written out right-justified in a field of length w in the
form

[ \code{Digit} ] \code{Digit : Digit Digit : Digit Digit} [ \code{. pi} ]

\item 
%If the first digit is zero, it is replaced by spaces. 
In case d = 0,
decimal point and fractional parts of seconds are not written out.

\item If the output value does not occupy the entire field, the left part is
padded up with spaces.
\end{enumerate}

\item {\bf Input}\\
\begin{enumerate}
\item A field of length w is read in, which must contain a time in a permitted
representation (see \ref{sec_type_clock}). Preceding and following spaces are ignored.

%%%\begin{accepted}
\item If w $<=$ 0 or d $<$ 0, no data is read and the \code{ClockFormatSignal}
   is induced.

\item If the input field does not match a valid clock constant, the 
   \code{ClockValueSignal} is induced.

%%%\end{accepted}

%%%\begin{discuss}
%%%The current implementation accepts automatic decimals if no decimal point is given and $d > 0$. This behavior is defined in the F-format. It was removed 
%%%when the D-format was relaxed.
%%%
%%%Should this behavior be removed here and from the F format?
%%%\end{discuss}

\item The parameter DecimalPosition is not regarded in input
 --- the decimal point must be given explicitly.
\end{enumerate}
\end{enumerate}

Examples:

\begin{tabular}{lll}
value               & format  & output \\ \hline
12.30 hours 5.2 sec & \code{T(12,1)} & \_\_12:30:05.2 \\
8 hours             & \code{T(8)}    & \_8:00:00
\end{tabular}

\subsection{The Duration Format (D)}   % 10.5.6
\label{sec_dation_d_format}

\begin{grammarframe}
\input{DurationFormat.bnf}
\index{D@\textbf{D}}
\index{FieldWidth!D-Format}
\index{D@\textbf{D}!FieldWidth}
\index{FORMAT@\textbf{FORMAT}!\textbf{D}}

\index{DecimalPositions!D-Format}
\index{D@\textbf{D}!DecimalPositions}
\end{grammarframe}

The duration format describes the external representation of durations.
The value of the field width means the total number w of the character
positions available for representation, the value of decimal positions
means the number d of the digits for the fractional parts of seconds of
the duration.

\begin{enumerate}
\item {\bf Output}\\
\begin{enumerate}
\item  The duration is written out right-justified in a field of length w in
the form

[ \code{-} ] [ [ [ [ \code{Digit} $^{...}$ \kw{HRS} ]
          \code{Digit}] \code{Digit} \kw{MIN} ] 
	 \code{Digit} ] \code{Digit} [ \kw{.} \code{pi} ] \kw{SEC}

%%%\begin{accepted}
\item If w $<=$ 0 or d $<$ 0, no data is written an the \code{DurationFormatSignal}
   is induced.

\item If w is too small for the given duration value, the \code{DurationValueSignal}
   is induced.

%%%\end{accepted}
%%%\begin{added}
\item Leading fields with hours and minutes may be suppressed by 
choosing a small value of w. 

\item Examples:

\begin{tabular}{lll}
value               & format  & output \\ \hline
11 hours 15 minutes & \code{D(20)}   & 11\_HRS\_15\_MIN\_00\_SEC \\
100 milliseconds    & \code{D(24,3)} & \_0\_HRS\_00\_MIN\_00.100\_SEC \\
100 milliseconds    & \code{D(9,3)} & 0.100\_SEC \\
-100 milliseconds    & \code{D(9,3)} & DurationValueSignal \\
-100 milliseconds    & \code{D(9,2)} & -0.10\_SEC \\
\end{tabular}

The character \_ means a space.
%%%\end{added}
\end{enumerate}

%%%\begin{removed}
%%%The rules from \ref{sec_dation_t_format} (1) are valid.
%%%\end{removed}
\item {\bf Input}\\
\begin{enumerate}
\item If w $<=$ 0 or d $<$ 0, no data is read and the \code{DurationFormatSignal}
   is induced.

%%%\begin{accepted}
\item A field of length w is read in, which must contain a duration in a
permitted representation (see \ref{sec_type_duration}).
 Preceding and following spaces are
ignored.

\item If the input field does not match a valid duration constant, the 
   \code{DurationValueSignal} is induced.

%%%\end{accepted}
%%%\begin{added}
\item A valid duration constant from chapter \ref{sec_type_duration} is accepted.
No delimiting spaces are required.
\item OpenPEARL accepts a leading sign \code{[+  $\mid$ -]} 
\item The value of d is not used in input. A decimal point is required in the
input field for precisions better than \code{1 SEC}.

\item Examples:

\begin{tabular}{lll}
input               & format  & value \\ \hline
11 HRS 15 MIN & \code{D(20)}   & 11\_HRS\_15\_MIN\_00\_SEC \\
0.100 SEC    & \code{D(24,3)} & \_0\_HRS\_00\_MIN\_00.100\_SEC \\
100 SEC    & \code{D(24,3)} & DurationValueSignal \\
10MIN1SEC    & \code{D(24,3)} & \_0\_HRS\_10\_MIN\_01\_SEC \\
-10MIN1SEC    & \code{D(24,3)} & -\_0\_HRS\_10\_MIN\_01\_SEC \\
\end{tabular}

%%%\end{added}
The character \_ means a space.

\end{enumerate}
\end{enumerate}

%%%\begin{removed}
%%%Examples:
%%%
%%%\begin{tabular}{lll}
%%%value               & format  & output \\ \hline
%%%11 hours 15 minutes & D(20)   & 11\_HRS\_15\_MIN\_00\_SEC \\
%%%100 milliseconds    & D(24,3) & \_0\_HRS\_00\_MIN\_00.100\_SEC \\
%%%100 milliseconds    & D(9,3) & 0.100\_SEC \\
%%%\end{tabular}
%%%
%%%The character \_ means a space.
%%%\end{removed}


\subsection{The List Format (LIST)}    % 10.5.7
\label{sec_dation_list_format}

\begin{grammarframe}
\input{ListFormat.bnf}
\index{L@\textbf{L}}
\index{LIST@\textbf{LIST}|see {L}}
\index{FORMAT@\textbf{FORMAT}!\textbf{L}}
\end{grammarframe}

The list format serves for the input/output of fixed, float, bit, char,
clock, and dur quantities.
\begin{enumerate}
\item {\bf Output}\\
Subsequent output data are separated by two spaces, each. The data are
written out in such a way, as if for a quantity of type

\begin{tabular}{lll}
\code{CHAR}(k)  & format & \code{A}(k)\\
\code{BIT}(k)   & format & \code{B}(k)\\
\code{FIXED}(k) & format & \code{F}(n)\\
\code{FLOAT}(k) & format & \code{E}(m,m-7,m-6)\\
\code{CLOCK}    & format & \code{T}(8)\\
\code{DUR}      & format & \code{D}(20)
\end{tabular}

with n = \code{ENTIER}(k/3.32) + 2, m = \code{ENTIER} (k/3.32) + 3 were
declared.
\item {\bf Input}\\
The input data can have any form permitted for the representation of
constants. 
\begin{added}
The same format replacements as on output are used.
Only one data element should be read in one input line.
\end{added} 
\begin{removed}
They are separated by a comma or at least two spaces. If no
constant is between two commas, the corresponding element of the
variable list remains unchanged.
\end{removed}
\end{enumerate}

Examples:

\begin{tabular}{llll}
data type & value    & implicit format & output \\ \hline
\code{FIXED}(15) & 127      & \code{F}(6)            & \_\_\_127 \\
\code{FLOAT}(23) & 3.28E+28 & \code{E}(9,2,3)       & \_3.28E+28\\
\code{BIT}(8)    & 'EF'B4   & \code{B}(8)            & 11101111
\end{tabular}

\subsection{The R Format (R)}   % 10.5.8
\label{sec_dation_r_format}

Sometimes, the same format position lists are used in more than one get
or put statement. The R format serves to describe these lists only
once. For this, a list with the so-called format declaration is
introduced.

\begin{grammarframe}
\input{FormatDeclaration.bnf}
\index{FORMAT@\textbf{FORMAT}!Statement}
\end{grammarframe}

Example:

\code{fTab :} \kw{FORMAT}\code{ ( }\kw{X}\code{(2), }\kw{F}\code{(8,3), (3) ( }
\kw{X}\code{(2), }\kw{E}\code{(10,3) ) ) ;}

A format declared in such a way can be used in a get or put statement,
stating its identifier:

\begin{grammarframe}
\input{RFormat.bnf}
\index{R@\textbf{R}}
\index{FORMAT@\textbf{FORMAT}!\textbf{R}}
\end{grammarframe}

When transmitting data, the \code{R} format is replaced by the format position
list contained in the indicated format declaration.

Example:

\kw{PUT}\code{ a, x, y, z }\kw{TO}\code{ printer }\kw{BY R}\code{(fTab)} ;

The format position list in the format declaration must not contain any
\code{R} format referring directly or indirectly (via another format
declaration) to its own format declaration.

\section{The Convert Statement (CONVERT)}   % 10.6
\label{sec_convert}
\index{CONVERT@\textbf{CONVERT}}

The comfortable conversion of number values in character strings and
vice versa is very important for many applications, e.g., for the
creation of display masks or for data exchange via communication
interfaces which cannot transfer binary data. The \code{PUT} and \code{GET}
statements enable that in connection with data stations. Following
that, the \code{CONVERT} statement is defined, performing formatted data
exchange with a character string or character string variable,
respectively, instead of a data station.

The general forms of the convert statement are:

\begin{grammarframe}
\input{ConvertStatement.bnf}
\input{ConvertToStatement.bnf}
\index{BY@\textbf{BY}!\textbf{CONVERT}}
\index{TO@\textbf{TO}!\textbf{CONVERT}}
\input{ConvertFromStatement.bnf}
\index{FROM@\textbf{FROM}!\textbf{CONVERT}}
\input{FormatOrPositionConvert.bnf}
\input{PositionConvert.bnf}
\index{RST@\textbf{RST}!\textbf{CONVERT}}
\end{grammarframe}

\OpenPEARL{} regards \hyperlink{StringSelection}{string selections} as
expressions. They are not accepted in the convert-from statement.

All permitted formats have the same meaning as for the \code{PUT} and \code{GET}
statements 
%%%\addedtext{
on \code{FORWARD} \code{DATIONS}.
%%%}.
%%%\begin{removed}
%%%The only exception is the S format. In the convert
%%%statement, the number of characters, which at this point in time was
%%%read from Expression$\S $CharacterString or written into
%%%Name$\S $CharacterString, respectively, is assigned to the variable in S
%%%format.
%%%\end{removed}

Example:

%%%\begin{added}
\begin{lstlisting}
DCL (index, convError) FIXED,
    value              FLOAT,
    stringOut          CHAR(40),
    stringIn           CHAR(20);
...
CONVERT 'Index =', index TO stringOut BY A, F(4), A;
CONVERT index, value FROM stringIn BY RST(convError), F(4), E(10,2);
\end{lstlisting}

\section{The Take and Send Statements}    % 10.7
\label{sec_take_send}

The take statement serves for the input, the send statement for the
output of data. These statements are provided for the transmission of
process data, and for the data exchange with user specific drivers,
respectively. The data station must possess the class attribute BASIC.

\begin{grammarframe}
\input{TakeStatement.bnf}
\index{TAKE@\textbf{TAKE}}
\index{BY@\textbf{BY}!\textbf{TAKE}}
\index{FROM@\textbf{FROM}!\textbf{TAKE}}
\index{RST@\textbf{RST}!\textbf{TAKE}}
\input{SendStatement.bnf}
\index{SEND@\textbf{SEND}}
\index{BY@\textbf{BY}!\textbf{SEND}}
\index{TO@\textbf{TO}!\textbf{SEND}}
\input{RSTFormat.bnf}
\index{RST@\textbf{RST}!\textbf{SEND}}
\end{grammarframe}

\OpenPEARL{} regards \hyperlink{StringSelection}{string selections} as
expressions. They are not accepted in the take statement.

%%%\begin{accepted}
The types of the ``variable'' in the take statement or the
``expression'' in the send statement, respectively, must fit to 
the type in the dation declaration. If the dation declaration 
specifies \code{ALL}, no type checking occurs.
%%%\end{accepted}

%%%\begin{removed}
%%%The attribute definitions RST, S and CONTROL may be stated in any
%%%order, but only once, each. The meanings of the CONTROL and S formats
%%%are implementation dependent and, thus, to be found in the respective
%%%user manual of a PEARL implementation.
%%%\end{removed}

Example:

%%%\begin{removed}
%%%{\bf SYSTEM};\\
%%%\x motor: DIGEA $^*$1$^*$1,4;
%%%\end{removed}

\begin{lstlisting}
PROBLEM;
  SPC motor DATION OUT SYSTEM BASIC BIT(4);
  DCL m     DATION OUT        BASIC BIT(4) CREATED ( motor );
  DCL on    INV BIT(4) INIT('1010'B1);

  SEND on TO m;
\end{lstlisting}

\section{Error Handling in I/O Statements (RST)}    % 10.8
\label{sec_dation_rst}

Usually, errors are recognised when the execution of I/O statements lead to the
termination of the responsible task, and an error message is written
out. This standard reaction of the PEARL system is suppressed by
defining

\kw{RST} ( Name$\S $ErrorVariable-FIXED )
\index{RST@\textbf{RST}}

in the parameter list of the \code{OPEN}/\code{CLOSE} statement, or as format or
position element in the other data transmission statements,
respectively.

``Name'' must denote a variable of type \code{FIXED}, in which an error number
not equal to zero is written in the case of error. Upon error free
execution of the I/O statement, the variable is set to zero. Possible
errors and their identification are to be found in the PEARL user manual
of the respective computer system.

The \code{RST} definition can be located in any position of the format or
position list of \code{PUT}, \code{GET}, \code{WRITE}, \code{READ} and \code{CONVERT} statements; also
multiple definitions with different variables are permitted. The \code{RST}
definition does not change the error reaction before the \code{RST} element is
evaluated in the format or position list. Upon recognition of an error,
the I/O statement is immediately aborted, and the error reaction valid
at this point in time (assignment of an error number to an \code{RST} variable
or the system reaction) is carried out.

\section{Interface for Additional Drivers}   % 10.9
\label{sec_driver_interface}

The diversity of --- particularly in the world of PCs 
and micro controllers --- existing I/O
controllers and devices does not allow the compiler supplier to account
for all I/O devices in the system part and create system names for them.
To be able to address special I/O devices from PEARL in spite of that
fact, a driver interface can be provided for certain operating systems,
to which the PEARL programmer himself can connect drivers. This
interface is described in the respective user manual (in the chapter
``Open Driver Interfaces'').

