\chapter{Signals}   % 11
\label{sec_signals}

When executing certain statements, internal events, so-called signals,
can occur, leading to an interruption of the running task; such signals
may, e.g., be an overflow in course of an arithmetic operation, a
division by zero, or reaching the end of a file.

A program abortion can be avoided by enabling a corresponding error
handling upon the occurrence of a signal.

\section{System Part Entries}
The signals needed for a program are declared in the system part, where
freely selectable user names can be assigned to them.
%%%%%\begin{removed} Additionally, an
%%%%%error list can be defined to limit the signal scheduling to one or
%%%%%several particular error numbers.
%%%%%
%%%%%UserNameDeclaration$\S $For-SIGNAL ::= \\
%%%%%\x Identifier$\S $UserName : Identifier$\S $SIGNAL-SystemName\\
%%%%%\x \x [ ( Identifier$\S $ErrorNumber [ , Identifier$\S $ErrorNumber ] $^{...}$ ) ]
%%%%%\end{removed}

\begin{grammarframe}
\inputBNF{SystemSignalDeclaration.bnf}
\end{grammarframe}


The signals and their associated error numbers possible on a certain
computer are described in the respective user manual, giving their
system names and meanings.

\section{Declaration and Specification of Signals} 

Before signals are used, they must be specified under their user names
in the problem part at module level.

Example:

%%%\begin{removed}
%%%IO\_SIGNAL and ENDF are system names; c\_error\_open is a pre-defined
%%%constant.
%%%
%%%{\bf MODULE};\\
%%%\x {\bf SYSTEM};\\
%%%\x \x OPEN\_ERR : IO\_SIGNAL (c\_error\_open);\\
%%%\x \x EOF : ENDF;\\
%%%\x \x ...\\
%%%\x {\bf PROBLEM};\\
%%%\x \x {\bf SPC} ( OPEN\_ERR, EOF ) {\bf SIGNAL};\\
%%%\x \x ... \\
%%%{\bf MODEND};
%%%\end{removed}
\code{FixedOverflowSignal} and \code{DationEOFSignal} are system names.

\begin{lstlisting}
MODULE(moduleName);
SYSTEM;
   overflow : FixedOverflowSignal;
   eof : DationEOFSignal;
...
PROBLEM;
   SPC ( overflow, eof ) SIGNAL;
 ... 
MODEND;
\end{lstlisting}

The general form of the specification of signals reads:

%%%\begin{front}
%%%SIGNAL-Specification ::=\\
%%%\x \{ {\bf SPECIFY  $\mid$  SPC} \} Identifier\_or\_IdentifierList 
%%%{\bf SIGNAL} [ GlobalAttribute ] ;
%%%\end{front}
\begin{grammarframe}
\inputBNF{SignalSpecification.bnf}
\index{SIGNAL@\textbf{SIGNAL}}
\index{SIGNAL@\textbf{SIGNAL}!Specification}
\end{grammarframe}

%%%%%\begin{added}
%%%%%
%%%%%Proposal fuer UserSignals
%%%%%
%%%%%Dies ist Thema auf der AK-Sitzung im November 2016

%In larger software systems it may be useful to define own signals, e.g.
%for wrong usage of functions. 
%The declaration of a user defined signal is allowed on module level.
%
%\begin{grammarframe}
%\inputBNF{UserSignalDeclaration.bnf}
%\index{SIGNAL@\textbf{SIGNAL}!Declaration}
%\end{grammarframe}

Valid ranges for the error number are listed in the plattform specific manual.

%%%%%\end{added}

\section{Signal Reactions}
\label{sec_signal_reactions}
The reaction planned for the occurrence of a signal is scheduled with
the following statement:

%%%%%\begin{removed}
%%%%%SchedulingSignalReaction ::=\\
%%%%%\x {\bf ON} Name$\S $Signal \{ [ {\bf RST} ( Name$\S $ErrorVariable-FIXED ) ] :\\
%%%%%\x \x SignalReaction $\mid$ {\bf RST} ( Name$\S $ErrorVariable-FIXED ) \};
%%%%%
%%%%%SignalReaction ::=\\
%%%%%\x UnlabeledStatement
%%%%%\end{removed}

%%%%%\begin{discuss}
\begin{grammarframe}
\inputBNF{SchedulingSignalReaction.bnf}
\index{SIGNAL@\textbf{SIGNAL}!Reaction}
\index{SIGNAL@\textbf{SIGNAL}!\textbf{ON}}
\index{ON@\textbf{ON}}
\index{RST@\textbf{RST}!Signal Reaction}
%%%%\inputBNF{SignalStatement.bnf}
\inputBNF{SignalReaction.bnf}
\inputBNF{SignalFinalStatement.bnf}
%\inputBNF{InduceStatement.bnf}
%\index{INDUCE@\textbf{INDUCE}}
\index{SIGNAL@\textbf{SIGNAL}!\textbf{INDUCE}}
\end{grammarframe}

\textbf{Note:} OpenPEARL does not support the \kw{INDUCE} with \kw{RST}, which is defined in PEARL90.

\textbf{Note:} OpenPEARL enforces for easy reading that each signal reaction is terminated with a SignalFinalStatement.

If the syntax variant 

\x \kw{ON} \hlink{Name}$\S $Signal \kw{RST} \kw{(} \hlink{Name}$\S $ErrorVariable-FIXED \kw{)} \kw{;}

is used the scheduling task (or procedure) will not be terminated (or aborted)
but continued after setting the error variable upon the occurance 
of the corresponding signal with the statement following the 
statement, which caused the failure.

%%%%%\end{discuss}

% obsolete 2024-02-02 (rm)
%Instead of UnlabeledStatement, all statements besides the
%statement SchedulingSignalReaction are permitted, particularly
%%%%\removedtext{blocks or procedure calls}\addedtext{blocks,  which  end with a SignalFinalStatement}.
%blocks,  which  end with a SignalFinalStatement.

The statement SchedulingSignalReaction is not permitted within \code{BEGIN}
and \code{REPEAT} blocks, or as SignalReaction.

\begin{itemize}
\item Validity range of scheduling a signal reaction:\\
The validity range of scheduling a signal reaction reaches from the
execution of the corresponding \code{ON} statement to the end of the
scheduling task or procedure, respectively.\\
If another scheduling for the same signal (e.g., in a procedure called
by the scheduling task) is run, it hides the preceding scheduling until
the end of the validity range of the new schedule (in the example until
the end of the scheduling procedure).
\item Leaving a procedure (task) after the execution of a signal
reaction:\\
If a signal occurs within the validity range of scheduling a signal
reaction, the procedure (task) in which the signal reaction is scheduled
is left by 
%%%\removedtext{an implicit RETURN(TERMINATE)} \addedtext{
the explicit \code{INDUCE}, \code{RETURN} or \code{TERMINATE}
%%%}
 after having executed the 
corresponding signal reaction, unless the signal reaction is left by a 
\code{GOTO} statement.
\item Validitiy of signal schedulings during the execution of a signal
reaction:\\
If a signal reaction is scheduled at procedure level, only those \code{ON}
schedulings are valid during the execution of this signal reaction which
were already executed before calling this procedure.\\
If the signal reaction is scheduled at task level, no \code{ON} schedulings are
valid during the execution of this signal reaction.
\item Validity of signal schedulings after leaving a signal reaction by
\code{GOTO}:\\
If the execution of a signal reaction scheduled at procedure or task
level is left by \code{GOTO}, the signal schedulings which were already
executed before triggering the signal, particularly those of the
triggered signal, are valid again.
\end{itemize}

To test the reaction scheduled for a signal, the occurrence of a signal
can be simulated analogously to the occurrence of an interrupt:

%%%\begin{removed}
%%%InduceStatement ::=\\
%%%\x {\bf INDUCE} Name$\S $Signal [ {\bf RST} ( Expression$\S $ErrorNumber ) ];
%%%\end{removed}
%%%\begin{added}
\begin{grammarframe}
\inputBNF{InduceStatement.bnf}
\index{INDUCE@\textbf{INDUCE}}
\end{grammarframe}


The variant of the \code{INDUCE} statement without a specified signal is 
only allowed inside a signal reaction. This will trigger the signal handler
of the superior function or task with the identical signal.
%%%\end{added}


Whereas it is possible to react to asynchronous exception situations
(i.e., influences effecting from outside) by interrupts, the signal
treatment serves exclusively for reacting to synchronous error states
(i.e., cause and treatment of the error state originate from the same
task).

Example:

The procedure \code{analysis} shall sequentially analyse a logbook created
in the course of a day; the single data elements of the \code{logBook} are of
type \code{Event}

\begin{lstlisting}
...
PROBLEM;
   SPC eof SIGNAL,
       tape  DATION INOUT SYSTEM ALL;
   TYPE Event ...;
   DCL logBook DATION IN Event DIM(*) FORWARD CREATED(tape);

analysis: PROC;
   DCL input Event;
   ...
   OPEN logBook;
   ON eof:
      BEGIN
         CLOSE logBook;
         RETURN;
      END; ! ON eof
   ...
   REPEAT
      READ input FROM logBook;
      ...
   END;
END; ! analysis
\end{lstlisting}

For testing, the statement

\kw{INDUCE} \code{eof;}

could be executed sporadically instead of the read statement.


%%%\begin{modified}
%%%fixiert, dass die Systemreaktion zur Terminierung des ausfuehrenden Task
%%%fuehrt.

If a signal is induced, whether by an error state or an induce
statement, and no signal reaction is scheduled for it, the system
reaction is triggered (i.e., generally an error message of the run time
system), which terminates the causing task.
%%%\end{modified}

By stating a variable after RST in the signal scheduling, the programmer
gets access to the error number (error cause). 
%In this case, the signal
%SignalName can be induced for the error with the number ErrorNumber by
%
%\kw{INDUCE} SignalName \kw{RST} \kw{(} ErrorNumber \kw{)} \kw{;}

%%%%%\begin{modified}
%%%%%Im letzten Satz des vorherigen Abschnitts muss "'In this case"' gestrichen 
%%%%%werden, oder?

%%%%%\begin{added}
%In case of inducing a signal with an error number, the predefined number
%is superseeded for only this action with the given value. This does not change
%the identity of a signal, when setting the actual number to the
%system number of another signal.  


%%%%%\end{added}
%%%%%\end{discuss}
OpenPEARL  uses groups of signals for tasking (\code{TaskSignal}), 
operations upon simple data types (\code{DataValueSignal}), 
dations (\code{DationSignal}) and formatting (\code{FormatSignal}).
 Additional there exist the signal \code{AnySignal}, which covers all
application relevant signals.


Example:

The signal \code{taskSignal}, which covers all error situations of 
tasking operations, may be simulated with a special situation of 
starting a currently active task by \code{INDUCE taskRunningSignal;}

% shall be simulated with the 
%superseeded error number ``1010`` and not superseeded error number
%The causing task shall react by writing out the error
%number on the console:

\begin{lstlisting}
SYSTEM;
 taskSignal: TaskSignal;
 prioOutOfRange: PrioOutOfRangeSignal;
 taskRunningSignal: TaskRunningSignal;
  
PROBLEM;
 SPC taskSignal SIGNAL; 	! assume error number 1100 
 SPC prioOutOfRange SIGNAL;     ! assume error number 1101
 SPC taskRunningSignal SIGNAL;	! assume error number 1102
 DCL errorNumber FIXED;

regulator: TASK PRIO 20;
restart:
   ...
   ON taskSignal RST (errorNumber):
      BEGIN 
         PUT errorNumber TO console;
         GOTO restart;
      END;  ! of error handler 
   ...
   CALL test;
   ...
END; ! regulator

start: TASK MAIN;
   ALL ...  ACTIVATE regulator;
END; ! start

test: PROC;
   IF ... 
   THEN
      ...
      INDUCE taskRunningSignal; ! prints 1102
   ELSE
      ...
      INDUCE prioOutOfRangeSignal; ! prints 1101
   FIN;
END; ! test
\end{lstlisting}

