\chapter{Preprocessor}

\OpenPEARL{} requires the specification of the global interface of
other modules in each module which wants to use other modules.  In
some cases conditional compilation is needed.  To support this
efforts, the OpenPEARL{} preprocessor (PP) is introduced and
automatically called by the compiler. PP transforms the source code
before compilation.

\section{Command-Line Interface}
\label{sec:commandline_interface}

\section{Include Statement}

The import of a different source file is triggered with:
\\
\begin{lstlisting}
#INCLUDE path/filename ;
\end{lstlisting}

Please note the terminating semicolon, which is mandatory and is
inherited from RTOS-UH.

The extension of the included file can be freely choosen. It is
recommended to use the extension \texttt{.inc}.

The filename can be prefixed with an absolute or relative path.  In
case of a relative path the search for the include file starts in the
current working directory of the OpenPEARL{} source code file.

The search path can be adapted by setting the includepath as a
command-line argument(see \ref{sec:commandline_interface}).

The content of the specified source file is inserted at the place of
the include statement. Error messages from the compiler and the
runtime system will be given relative to the included file.

The base location for relative path names is the directory containing
the file, doing the \texttt{INCLUDE}.

If an included file contains further \texttt{\#INCLUDE} statements
with relative path names, they included relative to the location of
the including file.

Example:

\begin{lstlisting}
/* declaration of a variable switch of type bit string of length 1 */ 
DECLARE switch BIT;

/* specification of a global variable status of type bit string 
of length 16 */ 
SPECIFY status BIT(16) GLOBAL(otherModule);
\end{lstlisting}


\section{Conditional Compilation}

\subsection{Definition of Macros Constants}
\#DEFINE ConstantIdentifier {\bf = } ConstantIntegerExpression ;

ConstantIntegerExpression :== \\
\x [ ConstantMonadicOperator ] ConstantOperand  [ 
 ConstantDyadicOperator ConstantIntegerExpression  ] $^{...}$\\

ConstantMonadicOperator:== \\
\x {\bf +} $\mid$ {\bf -}

ConstantDyadicOperator:== \\
\x {\bf +} $\mid$ {\bf -} $\mid$ {\bf *} $\mid$ {\bf // }

ConstantOperand :== \\
\x Integer $\mid$ Identifier$\S $Constant $\mid$ 
{\bf (} ConstantIntegerExpression {\bf )}

Example:\\
\#DEFINE lines = 10;			\\
\#DEFINE lineLength = 80;		\\
\#DEFINE size = 2*(lines*lineLength);	\\

Example usage:

MODULE(condCompile);\\
\\
SYSTEM:\\
\ \\
PROBLEM;\\
\#INCLUDE config.inc    ! defines target\\
\ \\
\#IFUDEF target;\\
\x \#ERROR 'target is not set';\\
\#ENDIF;\\
\\
\#IF target == 2;\\
\x  \#INCLUDE 'module2.inc' ;\\
\#ELSE IF target == 3 ;\\
\x   \#INCLUDE 'module3.inc' ;\\
\#ELSE ;\\
\x   \#ERROR 'unknown target';\\
\#ENDIF;\\

Zulaessige Vergleiche: $<, <= >, >=, ==, /= $

Macros: DEFINE, INCLUDE, IF, IFUDEF, IFDEF, (?? auch UNDEF?? )

'\#' und Makroname muessen zusammen stehen. Vor einem Makro ist nur
Whitespace erlaubt. Makroanweisungen enden mit ';'.  Nach dem ';' sind
normal PEARL-Anweisungen zulaessig.

\subsection{Undefining of Macros Constants}

\section{Pragmas}

\section{Diagnostics}
\#error
\#warn
\#info




%%% Local Variables:
%%% TeX-master: "p-report"
%%% End:
